\chapter*{Introduction de la Partie II}
\addcontentsline{toc}{chapter}{Introduction de la Partie II}

La seconde partie traite l’étude systémique de la machine frigorifique à éjecteur au R718, en mettant l’accent sur le couplage des composants, l’implémentation logicielle et l’analyse des performances.

L’enjeu principal est de passer d’une modélisation « composant par composant » à une représentation cohérente du cycle complet, où les niveaux de pression, les débits massiques et les transferts thermiques sont interdépendants.
Le couplage est formulé selon une convention de numérotation unifiée des états thermodynamiques, utilisée à la fois dans la simulation et dans la représentation sur diagrammes \(P\text{-}h\) et \(T\text{-}s\).

Dans ce cadre, l’outil numérique développé en Python s’appuie sur une architecture modulaire orientée objet, permettant de tester chaque composant séparément puis de l’intégrer au système global.
L’éjecteur est considéré par défaut selon un modèle compressible amélioré (V2), intégrant les diagnostics de régime, la détection d’étranglement et la présence éventuelle d’une onde de choc.

L’analyse finale porte sur la performance énergétique (notamment le coefficient de performance, \(COP\)), l’influence des conditions opératoires, et les recommandations d’amélioration.
Une mini-analyse exergétique est également introduite afin d’identifier qualitativement les sources majeures d’irréversibilités, en cohérence avec la perspective d’approfondissement prévue en thèse.
