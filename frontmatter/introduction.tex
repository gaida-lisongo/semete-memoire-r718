% ==========================================================================
% introduction_generale.tex
% Introduction générale — Mémoire (4 à 6 pages)
% Structure en 6 sections, progressive et captivante
% À inclure via : \input{introduction_generale}
% ==========================================================================

\chapter*{Introduction générale}
\addcontentsline{toc}{chapter}{Introduction générale}
\markboth{Introduction générale}{Introduction générale}

% --------------------------------------------------------------------------
% NOTE D’INTÉGRATION
% - Tu peux insérer tes citations BibTeX avec \cite{...} là où c’est indiqué.
% - Les valeurs numériques (ex: pression de saturation à 10°C) sont conservées
%   et peuvent être sourcées (ASHRAE, IAPWS, CoolProp, ouvrages).
% --------------------------------------------------------------------------

\section*{I. Contexte énergétique mondial et enjeux de la réfrigération}
\addcontentsline{toc}{section}{I. Contexte énergétique mondial et enjeux de la réfrigération}

La demande mondiale en \textit{froid utile} connaît une croissance rapide, portée simultanément par l’augmentation du confort thermique (climatisation), l’urbanisation, l’essor des services, et la nécessité de sécuriser les chaînes de conservation des denrées et des médicaments. Dans de nombreuses régions, en particulier dans les zones intertropicales, le froid n’est plus seulement un élément de confort : il devient un enjeu de santé publique, de sécurité alimentaire et de productivité économique. Or, cette montée en puissance du besoin en réfrigération intervient dans un contexte où les systèmes énergétiques sont déjà fortement sollicités.

Les technologies dominantes reposent encore largement sur le \textit{cycle à compression mécanique de vapeur}, reconnu pour sa maturité industrielle et ses performances. Cependant, cette solution s’accompagne d’un coût énergétique élevé : les installations de climatisation et de réfrigération contribuent de manière significative à la consommation électrique, souvent aux heures de pointe, ce qui aggrave la contrainte sur les réseaux et augmente le recours aux moyens de production carbonés lorsque la demande excède l’offre renouvelable. À l’échelle macro-énergétique, la réfrigération devient ainsi un contributeur majeur aux émissions indirectes de \ce{CO2}, puisque l’électricité consommée est encore largement produite à partir de sources fossiles dans de nombreux pays.

Dans ce contexte, réduire la dépendance de la production de froid à l’électricité, tout en garantissant des performances satisfaisantes, constitue une priorité technique et sociétale. Cela conduit naturellement à explorer des architectures capables de valoriser des sources d’énergie thermique abondantes et disponibles localement, en particulier le rayonnement solaire, très favorable dans les régions intertropicales.

% --------------------------------------------------------------------------
% À CITER (exemples) :
% - Rapports énergie/climatisation, statistiques de consommation
% - Tendances sur la chaîne du froid, développement dans les zones chaudes
% --------------------------------------------------------------------------

\section*{II. Contraintes environnementales et transition vers les fluides naturels}
\addcontentsline{toc}{section}{II. Contraintes environnementales et transition vers les fluides naturels}

Au-delà de la dimension énergétique, la réfrigération est également au cœur d’une transition environnementale structurée par des contraintes réglementaires internationales. Historiquement, l’abandon progressif des CFC puis des HCFC a été imposé par la nécessité de protéger la couche d’ozone, notamment à travers le \textit{Protocole de Montréal}. Plus récemment, l’attention s’est déplacée vers l’impact climatique des fluides frigorigènes : de nombreux HFC, utilisés comme substituts, possèdent un potentiel de réchauffement global (\textit{GWP}) élevé. L’\textit{Amendement de Kigali} a ainsi accéléré la trajectoire de réduction des HFC, renforçant la recherche de solutions à faible ou nul impact climatique.

Dans ce cadre, les \textit{fluides naturels} (ammoniac, hydrocarbures, dioxyde de carbone, eau) apparaissent comme des alternatives crédibles, car ils offrent généralement un GWP négligeable. Cependant, chaque fluide naturel implique des compromis techniques : toxicité (NH\textsubscript{3}), inflammabilité (hydrocarbures), pressions élevées (\ce{CO2}), ou fonctionnement sous vide (eau). Le choix du fluide ne peut donc pas être dissocié de l’architecture du cycle et des contraintes d’intégration.

Parmi ces candidats, l’eau, désignée par R718, se distingue par un avantage environnemental radical : elle est non toxique, non inflammable, disponible et son GWP est nul. Cette évidence « écologique » se heurte toutefois à une réalité thermodynamique : l’eau comme réfrigérant impose, pour produire du froid à des températures usuelles, un fonctionnement à très basse pression. C’est précisément ce défi qui rend le sujet à la fois complexe et scientifique, mais aussi porteur d’innovation.

% --------------------------------------------------------------------------
% À CITER :
% - Protocole de Montréal, Kigali, rapports sur HFC/GWP
% - Panorama des fluides naturels et contraintes associées
% --------------------------------------------------------------------------

\section*{III. Réfrigération solaire thermique : principes et limites}
\addcontentsline{toc}{section}{III. Réfrigération solaire thermique : principes et limites}

L’idée d’une réfrigération alimentée par énergie solaire s’inscrit dans une logique physique simple : lorsqu’un besoin de froid est maximal (fortes températures ambiantes), l’irradiation solaire est souvent élevée. Le solaire devient alors une source d’énergie locale et simultanée au besoin. Deux grandes familles de cycles thermiques sont couramment envisagées : les cycles \textit{à absorption} et les cycles \textit{à éjection}.

Les systèmes à absorption, notamment le couple LiBr–\ce{H2O}, sont parmi les plus répandus dans la littérature. Ils permettent de produire du froid à partir d’une source thermique, mais présentent des contraintes technologiques importantes : risques de cristallisation, corrosion, nécessité de vide, sensibilité aux conditions opératoires et complexité d’exploitation. Pour des applications décentralisées ou dans des environnements où la maintenance spécialisée est limitée, ces contraintes peuvent devenir un verrou majeur.

Le cycle à éjecteur constitue une alternative attractive. Il repose sur un organe statique (l’éjecteur) capable d’assurer la recompression sans compresseur mécanique, grâce à l’énergie cinétique d’un jet primaire. En termes de robustesse et de simplicité, l’éjecteur offre des avantages structurels : absence de pièces en mouvement, maintenance réduite, bonne tolérance aux environnements sévères. De plus, sa compatibilité avec des sources thermiques de température modérée (typiquement 80–120\,$^\circ$C) en fait un candidat sérieux pour une intégration avec le solaire thermique, notamment via des concentrateurs cylindro-paraboliques ou des capteurs adaptés.

Cependant, l’intérêt de l’éjecteur ne supprime pas les difficultés. Au contraire, il déplace le problème vers un couplage thermodynamique et hydrodynamique délicat entre générateur, éjecteur, condenseur et évaporateur. Et lorsque l’eau est choisie comme fluide de travail, ce couplage se déroule sous des niveaux de pression très faibles, ce qui accentue les sensibilités et impose une modélisation rigoureuse.

% --------------------------------------------------------------------------
% À CITER :
% - Réfrigération solaire : absorption vs éjection
% - Limites LiBr-H2O, robustesse des cycles à éjecteur
% --------------------------------------------------------------------------

\section*{IV. Spécificités thermodynamiques du fluide R718 sous vide}
\addcontentsline{toc}{section}{IV. Spécificités thermodynamiques du fluide R718 sous vide}

L’utilisation de l’eau (R718) comme réfrigérant conduit à un fait central : pour obtenir des températures d’évaporation compatibles avec des applications de froid \textit{à moyenne température}, la pression d’évaporation devient extrêmement faible. À titre indicatif, à $T_\mathrm{evap} \approx 10\,^\circ$C, la pression de saturation est de l’ordre de $1{,}2$\,kPa, soit un régime de vide prononcé. Cette caractéristique, loin d’être un détail, conditionne l’ensemble du dimensionnement et de la stabilité du système.

Premièrement, les volumes spécifiques de la vapeur sont élevés, ce qui implique des sections d’écoulement et des vitesses susceptibles d’augmenter les pertes de charge. Deuxièmement, le fonctionnement sous vide rend le système très sensible aux entrées de gaz non condensables, qui perturbent la condensation, modifient les échanges et dégradent les performances. Troisièmement, de faibles variations de pression ou de température peuvent produire des changements significatifs de qualité ($x$) et déplacer les états thermodynamiques au voisinage du dôme de saturation, introduisant des risques de comportements diphasiques non désirés dans certaines zones.

Surtout, l’éjecteur introduit un ensemble d’effets non linéaires : détente supersonique dans la tuyère primaire, aspiration du secondaire, mélange, et récupération de pression dans le diffuseur, avec apparition possible d’ondes de choc. Sous ces conditions, les équations gouvernantes deviennent sensibles aux hypothèses de modélisation (gaz parfait vs propriétés réelles) et aux paramètres d’efficacité. Ainsi, même si le concept d’un cycle R718 à éjecteur est séduisant, son étude ne peut être conduite sérieusement sans un outillage numérique et une stratégie de validation qui prennent en compte ces particularités.

% --------------------------------------------------------------------------
% À CITER :
% - Pressions de saturation R718, tables vapeur, CoolProp/IAPWS
% - Sensibilités aux pertes de charge, non-condensables, vide
% --------------------------------------------------------------------------

\section*{V. Lacunes scientifiques et nécessité d’une modélisation numérique}
\addcontentsline{toc}{section}{V. Lacunes scientifiques et nécessité d’une modélisation numérique}

Dans les conditions décrites, une approche purement empirique ou basée sur des dimensionnements isolés atteint rapidement ses limites. D’une part, l’éjecteur est un composant dont le comportement dépend fortement du couplage global : une variation du condenseur (température, capacité d’échange) modifie la pression de condensation, ce qui modifie la pression de refoulement admissible, ce qui rétroagit sur la capacité d’entrainement et la stabilité du mélange. D’autre part, les échangeurs (évaporateur, condenseur, générateur) imposent des contraintes thermiques réelles (via $K$, $A$ et $\Delta T_\mathrm{lm}$) qui peuvent rendre un cycle théoriquement cohérent \textit{technologiquement irréalisable} si les surfaces ou les coefficients d’échange ne sont pas adaptés.

Une difficulté scientifique majeure est donc la suivante : il ne suffit pas d’obtenir des états thermodynamiques « corrects » ; il faut aussi assurer la \textit{cohérence énergétique} entre le bilan massique (lié aux débits et aux enthalpies) et la capacité d’échange réellement disponible. En pratique, cela exige des mécanismes de contrôle capables de détecter automatiquement des situations de sous-dimensionnement ou d’incohérence, afin d’éviter des interprétations erronées.

Par ailleurs, la littérature reste relativement hétérogène sur l’outillage numérique dédié aux puissances intermédiaires (par exemple autour de 12\,kW) en contexte intertropical, et sur les simulateurs capables d’intégrer à la fois des diagnostics physiques (régimes de l’éjecteur, chocs, états diphasiques) et des contraintes d’échangeurs (mismatch thermique). Ces constats justifient le développement d’un modèle numérique robuste, spécifiquement conçu pour explorer le compromis entre performance, faisabilité technologique et stabilité.

Il devient ainsi nécessaire de mettre en place un simulateur qui couple de manière contrôlée les équations thermodynamiques (bilans d’enthalpie, états de saturation, COP) et les mécanismes hydrodynamiques clés (aspiration, mélange, compression par diffuseur), tout en étant suffisamment instrumenté pour rendre visibles ses limites.

% --------------------------------------------------------------------------
% À CITER :
% - Lacunes sur outils adaptés, études de cycles à éjecteur R718
% - Limites des approches empiriques, couplage éjecteur-condenseur
% --------------------------------------------------------------------------

\section*{VI. Objectifs et organisation du mémoire}
\addcontentsline{toc}{section}{VI. Objectifs et organisation du mémoire}

\subsection*{Objectif général}

L’objectif général de ce travail est de \textbf{modéliser, implémenter et analyser} les performances d’un cycle frigorifique à éjecteur utilisant l’eau (R718) et alimenté par une source thermique solaire, en tenant compte des contraintes spécifiques liées au fonctionnement sous vide et à la non-linéarité du couplage entre composants.

\subsection*{Objectifs spécifiques}

De manière plus détaillée, il s’agit de :
\begin{enumerate}
    \item établir un cadre méthodologique clair définissant le périmètre du modèle, les hypothèses globales, les variables d’entrée et les critères de cohérence ;
    \item développer un simulateur numérique modulaire capable de représenter les composants principaux (détendeur, évaporateur, éjecteur, condenseur, pompe, chaudière) et de faire circuler les états thermodynamiques ;
    \item intégrer des mécanismes de diagnostic (drapeaux/flags) permettant de détecter les incohérences physiques, les risques de fonctionnement (vide poussé, diphasique inattendu, mismatch thermique) et les limites de validité ;
    \item réaliser une analyse de performance sur un cas nominal et proposer des pistes de dimensionnement et de recommandations conceptuelles.
\end{enumerate}

\subsection*{Approche et outils}

Le simulateur est implémenté en \textbf{Python}, avec calcul des propriétés thermodynamiques via \textbf{CoolProp}. Une architecture logicielle inspirée du \textbf{patron MVC} est adoptée afin de séparer la logique de calcul, la gestion des données et la présentation (interface et visualisation). Les résultats sont exploités au moyen de représentations thermodynamiques (diagrammes $P$--$h$ et $P$--$s$) et d’indicateurs synthétiques (COP, débits, bilans énergétiques), renforcés par des diagnostics internes.

\subsection*{Organisation du mémoire}

Le mémoire est structuré en deux parties principales :
\begin{itemize}
    \item \textbf{La première partie} présente le cadre théorique et technologique : principes de la réfrigération solaire, description des composants du cycle à éjecteur et enjeux propres au fluide R718 sous vide.
    \item \textbf{La deuxième partie} développe l’étude systémique : méthodologie de modélisation, architecture du simulateur, modélisation des composants, puis analyse des performances et recommandations.
\end{itemize}

Ainsi, cette étude vise à fournir à la fois un éclairage scientifique sur les mécanismes physiques dominants et un outil numérique structuré permettant d’explorer, de manière reproductible, les compromis de conception d’une machine frigorifique à éjecteur R718 destinée à des applications réelles.

% ==========================================================================
% FIN DU FICHIER
% ==========================================================================
