% ==========================================================================
% introduction_generale.tex
% Introduction générale — Mémoire (mise à jour)
% + Ajout : Problématique / Questions de recherche / Hypothèses / Méthodologie
% ==========================================================================

\chapter*{Introduction générale}
\addcontentsline{toc}{chapter}{Introduction générale}
\markboth{Introduction générale}{Introduction générale}

% --------------------------------------------------------------------------
% NOTE D’INTÉGRATION
% - Insérer tes citations BibTeX avec \cite{...} là où c’est indiqué.
% - Les commentaires %TODO te guident pour compléter avec tes choix finaux.
% --------------------------------------------------------------------------

La demande mondiale en \textit{froid utile} connaît une croissance rapide, portée simultanément par l’augmentation du confort thermique (climatisation), l’urbanisation, l’essor des services, et la nécessité de sécuriser les chaînes de conservation des denrées et des médicaments. Dans de nombreuses régions, en particulier dans les zones intertropicales, le froid n’est plus seulement un élément de confort : il devient un enjeu de santé publique, de sécurité alimentaire et de productivité économique. Or, cette montée en puissance du besoin en réfrigération intervient dans un contexte où les systèmes énergétiques sont déjà fortement sollicités.

Les technologies dominantes reposent encore largement sur le \textit{cycle à compression mécanique de vapeur}, reconnu pour sa maturité industrielle et ses performances. Cependant, cette solution s’accompagne d’un coût énergétique élevé : les installations de climatisation et de réfrigération contribuent de manière significative à la consommation électrique, souvent aux heures de pointe, ce qui aggrave la contrainte sur les réseaux et augmente le recours aux moyens de production carbonés lorsque la demande excède l’offre renouvelable. À l’échelle macro-énergétique, la réfrigération devient ainsi un contributeur important aux émissions indirectes de \ce{CO2}, puisque l’électricité consommée est encore largement produite à partir de sources fossiles dans de nombreux pays.
%TODO \cite{...} (rapports énergie/climatisation, statistiques de consommation)

Au-delà de la dimension énergétique, la réfrigération est également au cœur d’une transition environnementale structurée par des contraintes réglementaires internationales. Historiquement, l’abandon progressif des CFC puis des HCFC a été imposé par la nécessité de protéger la couche d’ozone, notamment à travers le \textit{Protocole de Montréal}. Plus récemment, l’attention s’est déplacée vers l’impact climatique des fluides frigorigènes : de nombreux HFC possèdent un potentiel de réchauffement global (\textit{GWP}) élevé. L’\textit{Amendement de Kigali} a ainsi accéléré la trajectoire de réduction des HFC, renforçant la recherche de solutions à faible ou nul impact climatique.
%TODO \cite{...} (Montréal, Kigali, GWP/HFC)

Parmi les fluides naturels, l’eau (R718) se distingue par un avantage environnemental radical : elle est non toxique, non inflammable, disponible et son GWP est nul. Cette évidence « écologique » se heurte toutefois à une réalité thermodynamique : l’eau comme réfrigérant impose, pour produire du froid à des températures usuelles, un fonctionnement à très basse pression. C’est précisément ce défi qui rend le sujet à la fois complexe et scientifique, mais aussi porteur d’innovation.

L’idée d’une réfrigération alimentée par énergie solaire s’inscrit dans une logique physique simple : lorsqu’un besoin de froid est maximal, l’irradiation solaire est souvent élevée. Deux grandes familles de cycles thermiques sont couramment envisagées : les cycles \textit{à absorption} et les cycles \textit{à éjection}.

Les systèmes à absorption (notamment LiBr--\ce{H2O}) sont largement étudiés, mais présentent des contraintes technologiques importantes : risques de cristallisation, corrosion, complexité d’exploitation et sensibilité aux conditions opératoires. Le cycle à éjecteur constitue une alternative attractive : il repose sur un organe statique, robuste, et compatible avec des sources thermiques de température modérée. Néanmoins, son fonctionnement dépend fortement du couplage entre générateur, éjecteur, condenseur et évaporateur, et ce couplage devient particulièrement délicat lorsque le fluide de travail est l’eau (R718) sous vide.
%TODO \cite{...} (absorption vs éjection, limites LiBr, avantages éjecteur)

L’utilisation de l’eau (R718) comme réfrigérant conduit à un fait central : pour obtenir des températures d’évaporation compatibles avec des applications de froid \textit{à moyenne température}, la pression d’évaporation devient extrêmement faible. À titre indicatif, à $T_\mathrm{evap} \approx 10\,^\circ$C, la pression de saturation est de l’ordre de $1{,}2$\,kPa, soit un régime de vide prononcé.
%TODO \cite{...} (tables vapeur/IAPWS/CoolProp)

Cette caractéristique conditionne l’ensemble du dimensionnement et de la stabilité du système : volumes spécifiques élevés, sensibilité aux pertes de charge, influence des gaz non condensables, apparition possible d’écoulements diphasiques non désirés, et non-linéarités marquées dans l’éjecteur (détente supersonique, mélange, onde de choc, récupération de pression). Dans ces conditions, la seule cohérence thermodynamique des états ne suffit pas : il est nécessaire de vérifier également la faisabilité énergétique et technologique, notamment via les capacités d’échange des échangeurs et des diagnostics internes.


Dans un cycle à éjecteur R718, les approches empiriques ou les dimensionnements isolés atteignent rapidement leurs limites. Une variation de la température de condensation modifie la pression de condensation, ce qui rétroagit sur l’éjecteur, puis sur le débit, puis sur les bilans énergétiques des échangeurs. Cette boucle de rétroaction rend le système intrinsèquement \textit{couplé} et potentiellement instable.

De plus, un cycle peut être thermodynamiquement cohérent tout en étant technologiquement irréalisable si les échangeurs ne peuvent fournir (ou rejeter) la puissance requise. Il devient donc indispensable de développer un outil numérique capable de (i) calculer les états et performances, (ii) instrumenter les résultats par des critères de cohérence (mismatch thermique, diphasique inattendu, vide poussé, etc.), et (iii) analyser la sensibilité des performances aux paramètres de conception.
%TODO \cite{...} (lacunes sur outils adaptés, couplage éjecteur-condenseur, etc.)

% ==========================================================================
% NOUVELLE SECTION : Problématique / Questions / Hypothèses / Méthodologie
% ==========================================================================

Malgré l’intérêt environnemental et la disponibilité de l’eau (R718), le développement d’une machine frigorifique à éjecteur fonctionnant sous vide soulève un ensemble de difficultés scientifiques et technologiques : forte sensibilité aux conditions de condensation et d’évaporation, non-linéarité de l’éjecteur, risques de fonctionnement diphasique non désiré, et dépendance critique aux capacités réelles des échangeurs (générateur, condenseur, évaporateur). Dans un tel contexte, l’enjeu principal n’est pas uniquement de \textit{calculer} un COP, mais de démontrer une cohérence globale du système en intégrant des mécanismes de contrôle et des indicateurs de validité.

La problématique peut alors être formulée comme suit :

\begin{quote}
    \textit{Comment modéliser et simuler de manière fiable un cycle frigorifique à éjecteur utilisant l’eau (R718) et alimenté par une source thermique solaire, en tenant compte du couplage non linéaire entre composants, des contraintes de fonctionnement sous vide, et de la faisabilité énergétique imposée par les échangeurs ?}
\end{quote}


À partir de cette problématique, ce travail s’articule autour des questions de recherche suivantes :

\begin{enumerate}
    \item \textbf{Q1.} Quels modèles mathématiques (0D/1D) permettent de représenter chaque composant du cycle (détendeur, évaporateur, éjecteur, condenseur, pompe, chaudière) de manière suffisamment fidèle pour une étude système ?
    \item \textbf{Q2.} Comment assurer la \textbf{cohérence} entre les bilans massiques/enthalpiques et les capacités d’échange réelles (via $K$, $A$, $\Delta T_\mathrm{lm}$), afin d’identifier automatiquement les cas technologiquement irréalistes ?
    \item \textbf{Q3.} Quelles conditions opératoires (notamment $T_\mathrm{gen}$, $T_\mathrm{cond}$, $T_\mathrm{evap}$) maximisent les performances tout en restant compatibles avec la stabilité sous vide (cavitation, non-condensables, pertes de charge) ?
    \item \textbf{Q4.} Dans quelle mesure l’intégration de diagnostics (\textit{flags}) améliore-t-elle la robustesse du simulateur et la qualité de l’interprétation des résultats ?
\end{enumerate}


Afin de rendre le problème traitable tout en conservant l’essentiel de la physique, les hypothèses suivantes sont adoptées :

\begin{enumerate}
    \item \textbf{H1 (régime permanent 0D)} : le cycle est étudié en régime permanent, avec des modèles globalisés (0D) pour les échangeurs et une représentation 1D simplifiée de l’éjecteur.
    \item \textbf{H2 (propriétés fiables)} : les propriétés thermodynamiques de l’eau (R718) sont calculées par une bibliothèque de référence (\textit{CoolProp}), garantissant la cohérence des états.
    \item \textbf{H3 (éjecteur modélisé)} : l’éjecteur peut être représenté par un modèle compressible 1D incluant la détente primaire, le mélange et une récupération de pression, avec prise en compte d’un choc normal lorsque nécessaire.
    \item \textbf{H4 (échangeurs paramétrés)} : les échangeurs (évaporateur, condenseur, chaudière) sont décrits par un modèle $K A \Delta T_\mathrm{lm}$, les paramètres $K$ et $A$ étant donnés ou estimés au stade de conception préliminaire.
    \item \textbf{H5 (cohérence instrumentée)} : des critères de cohérence et des drapeaux (\textit{flags}) permettent de détecter des écarts significatifs (mismatch thermique, états diphasiques inattendus, risque de cavitation, etc.) et d’encadrer l’interprétation des résultats.
\end{enumerate}

% ==========================================================================
% NOUVELLE SECTION : Méthodologie
% ==========================================================================
La démarche adoptée suit un protocole de recherche appliquée combinant (i) une construction progressive du modèle, (ii) une implémentation logicielle structurée, et (iii) une validation par critères de cohérence et analyses numériques. Elle se décline en cinq étapes complémentaires :


Dans un premier temps, une revue de littérature est menée afin de :
\begin{itemize}
    \item identifier les architectures de réfrigération solaire thermique et la place des cycles à éjecteur ;
    \item préciser les modèles de composants disponibles (échangeurs, éjecteur, pompe, détente) et leurs hypothèses ;
    \item établir les ordres de grandeur pertinents pour une application de puissance intermédiaire (environ 12\,kW).
\end{itemize}
Cette étape aboutit à la définition du \textit{périmètre} du modèle (circuit frigorifique et circuit moteur), des variables d’entrée et des grandeurs de sortie visées.

Ensuite, les hypothèses de modélisation sont fixées (régime permanent, modèles 0D pour échangeurs, représentation 1D de l’éjecteur). Surtout, un ensemble de critères de cohérence est défini afin de \textit{contrôler} la validité des calculs :
\begin{itemize}
    \item cohérence énergétique massique vs $K A \Delta T_\mathrm{lm}$ (mismatch thermique) ;
    \item détection de conditions de vide poussé ;
    \item détection de zones diphasiques inattendues ;
    \item vérification des hiérarchies de pression et des conditions d’aspiration ;
    \item indicateurs de stabilité (ex. risque de cavitation simplifié côté pompe).
\end{itemize}
Ces critères sont implémentés sous forme de \textit{flags}, ce qui permet une traçabilité des limites du modèle et une interprétation robuste des résultats.

Le simulateur est développé en \textbf{Python} autour d’une architecture modulaire inspirée du \textbf{MVC}. Chaque composant est implémenté comme un module indépendant :
\begin{itemize}
    \item un \textit{modèle} (équations, solveur, diagnostics),
    \item une structure de \textit{résultat} (états, bilans, flags),
    \item des services transversaux (propriétés via \textit{CoolProp}, objets d’état thermodynamique).
\end{itemize}
L’objectif est de garantir la maintenabilité du code, la réutilisabilité des composants et la possibilité d’étendre le modèle (ex. amélioration de l’éjecteur, ajout d’exergie).

À partir d’un cas nominal (défini par $T_\mathrm{gen}$, $T_\mathrm{cond}$, $T_\mathrm{evap}$ et une puissance frigorifique cible), des simulations sont réalisées pour :
\begin{itemize}
    \item déterminer les débits massiques (primaire/secondaire) et les états thermodynamiques,
    \item calculer les performances globales (COP, bilans énergétiques),
    \item produire des diagrammes thermodynamiques ($P$--$h$, $P$--$s$) et des graphiques d’analyse.
\end{itemize}
Les résultats sont systématiquement accompagnés des flags afin d’identifier les cas physiquement plausibles et technologiquement cohérents.

Enfin, une analyse de sensibilité est menée pour étudier l’influence des paramètres dominants (températures de fonctionnement, $K A$ des échangeurs, rendements de l’éjecteur) sur les performances et la stabilité. Cette étape permet de proposer des recommandations conceptuelles, notamment sur :
\begin{itemize}
    \item les compromis entre $T_\mathrm{gen}$ et les pertes thermiques/contraintes d’échange,
    \item le dimensionnement minimal des échangeurs,
    \item les précautions liées au fonctionnement sous vide (purge, étanchéité, pertes de charge).
\end{itemize}

% ==========================================================================
% RE-NUMÉROTATION : l’ancienne section VI devient VIII pour garder l’ordre
% ==========================================================================
L’objectif général de ce travail est de \textbf{modéliser, implémenter et analyser} les performances d’un cycle frigorifique à éjecteur utilisant l’eau (R718) et alimenté par une source thermique solaire, en tenant compte des contraintes spécifiques liées au fonctionnement sous vide et à la non-linéarité du couplage entre composants.

De manière plus détaillée, il s’agit de :
\begin{enumerate}
    \item établir un cadre méthodologique définissant le périmètre du modèle, les hypothèses globales, les variables d’entrée et les critères de cohérence ;
    \item développer un simulateur numérique modulaire représentant les composants principaux et assurant la circulation des états thermodynamiques ;
    \item intégrer des mécanismes de diagnostic (\textit{flags}) pour détecter incohérences physiques, risques de fonctionnement et limites de validité ;
    \item réaliser une analyse de performance (cas nominal, sensibilité) et formuler des recommandations conceptuelles.
\end{enumerate}


Le simulateur est implémenté en \textbf{Python}, avec calcul des propriétés thermodynamiques via \textbf{CoolProp}. Une architecture logicielle inspirée du \textbf{patron MVC} est adoptée afin de séparer la logique de calcul, la gestion des données et la présentation. Les résultats sont exploités au moyen de diagrammes $P$--$h$ et $P$--$s$ ainsi que d’indicateurs synthétiques (COP, débits, bilans énergétiques), renforcés par des diagnostics internes.


Le mémoire est structuré en deux parties principales :
\begin{itemize}
    \item \textbf{La première partie} présente le cadre théorique et technologique : principes de la réfrigération solaire, description des composants du cycle à éjecteur et enjeux propres au fluide R718 sous vide.
    \item \textbf{La deuxième partie} développe l’étude systémique : méthodologie de modélisation, architecture du simulateur, modélisation des composants, puis analyse des performances et recommandations.
\end{itemize}

% ==========================================================================
% FIN DU FICHIER
% ==========================================================================
