\chapter*{Résumé}
\addcontentsline{toc}{chapter}{Résumé}

Ce travail présente la conception, la modélisation et la simulation d’une machine frigorifique à éjecteur fonctionnant au fluide frigorigène naturel R718 (eau).

L’objectif principal est le dimensionnement inverse du système à partir de deux paramètres d’entrée : la puissance frigorifique de l’évaporateur et la température d’évaporation. 
Le modèle développé permet de calculer automatiquement les états thermodynamiques du cycle complet, les débits massiques, le coefficient de performance (COP), ainsi que les indicateurs de diagnostic.

Une modélisation avancée de l’éjecteur a été implémentée en intégrant les effets d’écoulement compressible supersonique, la détection d’onde de choc normale et l’analyse des régimes d’aspiration.

Une interface graphique interactive permet la visualisation dynamique du fonctionnement du système ainsi que l’affichage des diagrammes thermodynamiques P-h et T-s.

Les résultats obtenus démontrent la cohérence thermodynamique du modèle ainsi que la convergence du dimensionnement global du système.

\bigskip
\textbf{Mots-clés :} Éjecteur, R718, Cycle frigorifique, Écoulement compressible, Simulation thermodynamique, Dimensionnement inverse.
