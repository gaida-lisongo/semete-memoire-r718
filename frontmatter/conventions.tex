\chapter*{Notations et Conventions}
\addcontentsline{toc}{chapter}{Notations et Conventions}

\section*{Convention de numérotation des états}

Le cycle thermodynamique est défini selon la convention suivante :

\begin{itemize}
    \item 1 → 2 : Détendeur
    \item 2 → 3 : Évaporateur
    \item 3 → 4 : Chambre de mélange de l’éjecteur
    \item 4 → 5 : Diffuseur de l’éjecteur
    \item 5 → 6 : Condenseur
    \item 1 → 7 : Pompe
    \item 7 → 8 : Chaudière (Générateur)
    \item 8 → 4 : Tuyère de l’éjecteur
\end{itemize}

\section*{Grandeurs thermodynamiques}

\begin{longtable}{p{3cm} p{10cm}}
    P       & Pression (Pa)                     \\
    T       & Température (K)                   \\
    h       & Enthalpie spécifique (J/kg)       \\
    s       & Entropie spécifique (J/kg.K)      \\
    x       & Titre vapeur                      \\
    $\dot{m}$ & Débit massique (kg/s)             \\
    $\mu$     & Taux d'entraînement de l'éjecteur \\
    M       & Nombre de Mach                    \\
\end{longtable}

Les diagrammes thermodynamiques P-h et T-s présentent uniquement la numérotation des états (1 à 8) sans légende descriptive.
