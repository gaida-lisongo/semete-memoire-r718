\chapter*{Introduction de la Partie I}
\addcontentsline{toc}{chapter}{Introduction de la Partie I}

Cette première partie est consacrée à l’étude détaillée des composants constituant la machine frigorifique à éjecteur fonctionnant au R718 (eau).
L’objectif est de construire une base scientifique et technologique robuste permettant, d’une part, de justifier les choix de configuration et, d’autre part, d’établir les modèles nécessaires à la simulation numérique du cycle complet.

Dans ce travail, l’approche adoptée repose sur une modélisation principalement unidimensionnelle (1D) des différents organes, en s’appuyant sur les principes de conservation de la masse, de la quantité de mouvement et de l’énergie.
Chaque composant est étudié selon une démarche structurée comprenant : (i) son rôle dans le système, (ii) les phénomènes physiques dominants, (iii) les aspects technologiques, et (iv) l’établissement d’un modèle mathématique utilisable en simulation.

Une attention particulière est portée à l’éjecteur, organe clé du cycle, car il assure simultanément l’aspiration de la vapeur secondaire issue de l’évaporateur et la recompression du mélange vers le niveau de pression de condensation.
Les sections dédiées à l’éjecteur intègrent les notions d’écoulement compressible, de régime supersonique et de formation d’onde de choc, afin d’aboutir à un modèle représentatif des régimes de fonctionnement observés.

Enfin, l’étude des échangeurs (évaporateur, condenseur, générateur/chaudière) met en évidence le lien direct entre performances thermodynamiques et dimensionnement des surfaces d’échange, via l’utilisation du coefficient global de transfert thermique \(K\), conformément à la convention de notation adoptée dans ce mémoire.
