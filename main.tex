\documentclass[12pt,a4paper,oneside]{book}

% ------------------------
% Packages généraux
% ------------------------
\usepackage[french]{babel}
\usepackage[utf8]{inputenc}
\usepackage[T1]{fontenc}
\usepackage{geometry}
\geometry{margin=2.5cm}

\usepackage{graphicx}
\usepackage{amsmath,amssymb}
\usepackage{siunitx}
\usepackage{booktabs}
\usepackage{float}
\usepackage{hyperref}
\usepackage{caption}
\usepackage{longtable}
\usepackage{array}
\usepackage{titlesec}
% --- Packages pour les illustrations ---
\usepackage{forest}          % Pour dessiner l'arborescence
\usepackage{fontawesome5}    % Pour les icônes (dossiers, fichiers)
\usepackage{tikz}
\usetikzlibrary{shapes.geometric, arrows, positioning, shadows, patterns}
\usepackage{xcolor}          % Pour les couleurs

% --- Styles pour les diagrammes ---
\tikzstyle{class} = [rectangle, draw, fill=blue!10, 
    text width=10em, text centered, rounded corners, minimum height=3em, drop shadow]
\tikzstyle{service} = [rectangle, draw, fill=orange!20, 
    text width=10em, text centered, rounded corners, minimum height=3em, drop shadow]
\tikzstyle{attribute} = [rectangle, draw, fill=gray!10, text width=12em, text centered, minimum height=2.5em]
\tikzstyle{method} = [rectangle, draw, fill=green!10, text width=12em, text centered, minimum height=2.5em]
\tikzstyle{line} = [draw, -latex']
\tikzstyle{dashed_line} = [draw, dashed, -latex']
\tikzstyle{window} = [rectangle, draw, fill=gray!10, 
    text width=20em, rounded corners, minimum height=23em, drop shadow]
\tikzstyle{button} = [rectangle, draw, fill=blue!20, 
    text width=15em, text centered, rounded corners, minimum height=2.5em, drop shadow]
\tikzstyle{title} = [font=\Large\bfseries]
\tikzstyle{block} = [rectangle, draw, fill=blue!10, 
    text width=8em, text centered, rounded corners, minimum height=3em, drop shadow]
\tikzstyle{test_script} = [rectangle, draw, fill=purple!20, 
    text width=8em, text centered, rounded corners, minimum height=3em, drop shadow]
% Définition des nouveaux noms de styles
\tikzset{
    block-1/.style={
        rectangle, 
        rounded corners, 
        minimum width=3cm, 
        minimum height=1cm,
        text centered, 
        draw=black, 
        fill=blue!10
    },
    line-1/.style={
        draw, 
        -latex'
    }
}

% Style UML - Classe
\tikzstyle{umlclass} = [
    rectangle, draw=black, fill=white, drop shadow,
    text width=14em, minimum width=14em,
    font=\footnotesize\sffamily
]
\tikzstyle{umltitle} = [
    rectangle, draw=black, fill=blue!15,
    text width=14em, minimum width=14em,
    text centered, font=\footnotesize\sffamily\bfseries,
    minimum height=1.8em
]
\tikzstyle{umlsection} = [
    rectangle, draw=black, fill=white,
    text width=14em, minimum width=14em,
    font=\footnotesize\ttfamily,
    inner ysep=4pt
]
% Style UML - Dataclass (résultat)
\tikzstyle{umldatatitle} = [
    rectangle, draw=black, fill=green!15,
    text width=14em, minimum width=14em,
    text centered, font=\footnotesize\sffamily\bfseries,
    minimum height=1.8em
]
% Style UML - Service / Core
\tikzstyle{umlcoretitle} = [
    rectangle, draw=black, fill=orange!15,
    text width=14em, minimum width=14em,
    text centered, font=\footnotesize\sffamily\bfseries,
    minimum height=1.8em
]
% Flèches UML
\tikzstyle{uml_uses} = [draw, -latex, dashed]
\tikzstyle{uml_returns} = [draw, -latex, thick, blue!60]
\tikzstyle{uml_inherits} = [draw, -open triangle 60, thick]
% --- Définition des styles pour l'arbre ---
\forestset{
  % Style pour les dossiers
  dir/.style={
    font=\sffamily\bfseries,
    inner ysep=2pt,
    content={\faFolder\hspace{3pt}#1},
  },
  % Style pour les fichiers
  file/.style={
    font=\sffamily,
    inner ysep=2pt,
    content={\faFile\hspace{3pt}#1},
  },
  % Style pour les fichiers Python
  pyfile/.style={
    font=\sffamily,
    inner ysep=2pt,
    content={\faPython\hspace{3pt}#1},
  }
}
\usepackage{listings} % Pour les blocs de code

% --- Style pour les blocs de code ---
\lstset{
    backgroundcolor=\color{black!5},
    basicstyle=\ttfamily\small,
    breaklines=true,
    frame=single,
    rulecolor=\color{black!20},
    captionpos=b,
    keepspaces=true,
    showspaces=false,
    showstringspaces=false,
    showtabs=false,
    tabsize=2
}


\usepackage{apacite}
\bibliographystyle{apacite}

% ------------------------
% Numérotation des équations par chapitre
% ------------------------
\numberwithin{equation}{chapter}

% ------------------------
% Début document
% ------------------------
\begin{document}

% Page de titre
\title{Conception et Simulation d’une Machine Frigorifique à Éjecteur Fonctionnant au R718}
\author{LISONGO SEMETE Gabriel}
\date{\today}
\maketitle

% ------------------------
% Résumé
% ------------------------
\frontmatter
\chapter*{Résumé}
\addcontentsline{toc}{chapter}{Résumé}

Ce travail présente la conception, la modélisation et la simulation d’une machine frigorifique à éjecteur fonctionnant au fluide frigorigène naturel R718 (eau).

L’objectif principal est le dimensionnement inverse du système à partir de deux paramètres d’entrée : la puissance frigorifique de l’évaporateur et la température d’évaporation. 
Le modèle développé permet de calculer automatiquement les états thermodynamiques du cycle complet, les débits massiques, le coefficient de performance (COP), ainsi que les indicateurs de diagnostic.

Une modélisation avancée de l’éjecteur a été implémentée en intégrant les effets d’écoulement compressible supersonique, la détection d’onde de choc normale et l’analyse des régimes d’aspiration.

Une interface graphique interactive permet la visualisation dynamique du fonctionnement du système ainsi que l’affichage des diagrammes thermodynamiques P-h et T-s.

Les résultats obtenus démontrent la cohérence thermodynamique du modèle ainsi que la convergence du dimensionnement global du système.

\bigskip
\textbf{Mots-clés :} Éjecteur, R718, Cycle frigorifique, Écoulement compressible, Simulation thermodynamique, Dimensionnement inverse.

\chapter*{Abstract}
\addcontentsline{toc}{chapter}{Abstract}

This work presents the design, modeling, and simulation of an ejector refrigeration system operating with the natural refrigerant R718 (water).

The main objective is the inverse dimensioning of the system based on two input parameters: the evaporator cooling capacity and the evaporation temperature.
The developed model automatically computes thermodynamic states, mass flow rates, coefficient of performance (COP), and diagnostic indicators.

An advanced ejector model including supersonic compressible flow, normal shock detection, and suction regime analysis was implemented.

A graphical user interface enables dynamic visualization of system behavior and thermodynamic diagrams (P-h and T-s).

Results demonstrate thermodynamic consistency and successful convergence of the global system dimensioning.

\bigskip
\textbf{Keywords:} Ejector, R718, Refrigeration cycle, Compressible flow, Thermodynamic simulation, Inverse design.


% ------------------------
% Tables
% ------------------------
\tableofcontents
\listoffigures
\listoftables

% ------------------------
% Abréviations et Conventions
% ------------------------
\chapter*{Liste des abréviations}
\addcontentsline{toc}{chapter}{Liste des abréviations}

\begin{longtable}{p{4cm} p{9cm}}

    COP  & Coefficient de performance            \\

    R718 & Eau utilisée comme fluide frigorigène \\

    LMTD & Log Mean Temperature Difference       \\

    MVC  & Model View Controller                 \\

    UI   & User Interface                        \\

    P-h  & Diagramme Pression–Enthalpie          \\

    T-s  & Diagramme Température–Entropie        \\
\end{longtable}

\chapter*{Notations et Conventions}
\addcontentsline{toc}{chapter}{Notations et Conventions}

\section*{Convention de numérotation des états}

Le cycle thermodynamique est défini selon la convention suivante :

\begin{itemize}
    \item 1 → 2 : Détendeur
    \item 2 → 3 : Évaporateur
    \item 3 → 4 : Chambre de mélange de l’éjecteur
    \item 4 → 5 : Diffuseur de l’éjecteur
    \item 5 → 6 : Condenseur
    \item 1 → 7 : Pompe
    \item 7 → 8 : Chaudière (Générateur)
    \item 8 → 4 : Tuyère de l’éjecteur
\end{itemize}

\section*{Grandeurs thermodynamiques}

\begin{longtable}{p{3cm} p{10cm}}
    P       & Pression (Pa)                     \\
    T       & Température (K)                   \\
    h       & Enthalpie spécifique (J/kg)       \\
    s       & Entropie spécifique (J/kg.K)      \\
    x       & Titre vapeur                      \\
    $\dot{m}$ & Débit massique (kg/s)             \\
    $\mu$     & Taux d'entraînement de l'éjecteur \\
    M       & Nombre de Mach                    \\
\end{longtable}

Les diagrammes thermodynamiques P-h et T-s présentent uniquement la numérotation des états (1 à 8) sans légende descriptive.


% ------------------------
% Corps du mémoire
% ------------------------
\mainmatter
% ==========================================================
% PARTIE I — ÉTUDE DES COMPOSANTS
% ==========================================================
\part{Étude des composants}

% Introduction de la Partie I
\chapter*{Introduction de la Partie I}
\addcontentsline{toc}{chapter}{Introduction de la Partie I}

Cette première partie est consacrée à l’étude détaillée des composants constituant la machine frigorifique à éjecteur fonctionnant au R718 (eau).
L’objectif est de construire une base scientifique et technologique robuste permettant, d’une part, de justifier les choix de configuration et, d’autre part, d’établir les modèles nécessaires à la simulation numérique du cycle complet.

Dans ce travail, l’approche adoptée repose sur une modélisation principalement unidimensionnelle (1D) des différents organes, en s’appuyant sur les principes de conservation de la masse, de la quantité de mouvement et de l’énergie.
Chaque composant est étudié selon une démarche structurée comprenant : (i) son rôle dans le système, (ii) les phénomènes physiques dominants, (iii) les aspects technologiques, et (iv) l’établissement d’un modèle mathématique utilisable en simulation.

Une attention particulière est portée à l’éjecteur, organe clé du cycle, car il assure simultanément l’aspiration de la vapeur secondaire issue de l’évaporateur et la recompression du mélange vers le niveau de pression de condensation.
Les sections dédiées à l’éjecteur intègrent les notions d’écoulement compressible, de régime supersonique et de formation d’onde de choc, afin d’aboutir à un modèle représentatif des régimes de fonctionnement observés.

Enfin, l’étude des échangeurs (évaporateur, condenseur, générateur/chaudière) met en évidence le lien direct entre performances thermodynamiques et dimensionnement des surfaces d’échange, via l’utilisation du coefficient global de transfert thermique \(K\), conformément à la convention de notation adoptée dans ce mémoire.


% Chapitre 1 (I.1) : Description de la machine
\chapter{Description de la machine frigorifique à eau de 12 kW}

\section{Évolution historique et positionnement scientifique}

L’histoire de la réfrigération thermodynamique débute au XIX\textsuperscript{e} siècle avec les travaux fondateurs de Carnot sur les cycles idéaux.
Par la suite, les premières machines à compression mécanique utilisant l’ammoniac et le dioxyde de carbone ont été développées pour des applications industrielles.

Parallèlement aux cycles à compression, des solutions alternatives utilisant la chaleur comme énergie motrice ont émergé, notamment les cycles à absorption et les cycles à éjection.
Les premiers systèmes à jet de vapeur (\textit{steam-jet refrigeration}) apparaissent à la fin du XIX\textsuperscript{e} siècle et trouvent des applications industrielles majeures au XX\textsuperscript{e} siècle \cite{Sokolov1990}.

Avec les restrictions environnementales issues du Protocole de Montréal puis de l’amendement de Kigali, les fluides frigorigènes synthétiques ont progressivement été remplacés par des fluides naturels \cite{Herold2016}.
L’eau (R718) présente l’avantage d’être non toxique, non inflammable et universellement disponible.

La valorisation de l’énergie solaire thermique dans les régions intertropicales a renforcé l’intérêt pour les cycles frigorifiques trithermes \cite{Kalogirou2014}.

Le présent travail étudie une machine frigorifique simple effet, dimensionnée pour une puissance frigorifique nominale de :

\begin{itemize}
    \item Température d’évaporation : $T_{evap} = 10^\circ C$
    \item Température de condensation : $T_{cond} = 35^\circ C$
    \item Température du générateur : $T_{gen} = 100^\circ C$
    \item Puissance frigorifique nominale : $\dot{Q}_{evap} = 12~\mathrm{kW}$
\end{itemize}

Cette configuration est adaptée aux conditions climatiques de la région intertropicale.

\section{Organisation générale de la machine}

La machine étudiée comprend :

\begin{itemize}
    \item Une pompe d’alimentation
    \item Une chaudière solaire
    \item Un éjecteur (tuyère + chambre de mélange + diffuseur)
    \item Un condenseur
    \item Un détendeur
    \item Un évaporateur
\end{itemize}

Le cycle thermodynamique suit la convention de numérotation suivante :

\begin{align*}
    1 & \rightarrow 2 &  & \text{Détendeur}          \\
    2 & \rightarrow 3 &  & \text{Évaporateur}        \\
    3 & \rightarrow 4 &  & \text{Chambre de mélange} \\
    4 & \rightarrow 5 &  & \text{Diffuseur}          \\
    5 & \rightarrow 6 &  & \text{Condenseur}         \\
    1 & \rightarrow 7 &  & \text{Pompe}              \\
    7 & \rightarrow 8 &  & \text{Chaudière}          \\
    8 & \rightarrow 4 &  & \text{Tuyère primaire}
\end{align*}

\section{Fonctionnement détaillé du cycle}

\subsection{Détente isoenthalpique (1 → 2)}

La détente dans le détendeur est modélisée comme isoenthalpique :

\begin{equation}
    h_1 = h_2
    \label{eq:detente_isoenthalpique}
\end{equation}

La pression chute de $P_{cond}$ à $P_{evap}$, générant un mélange diphasique.

\subsection{Évaporation (2 → 3)}

Dans l’évaporateur, le fluide absorbe la puissance frigorifique :

\begin{equation}
    \dot{Q}_{evap} = \dot{m}_{sec} \left( h_3 - h_2 \right)
    \label{eq:evaporation}
\end{equation}

À $10^\circ C$, la pression de saturation est de l’ordre du kPa, impliquant un volume spécifique élevé \cite{Moran2014}.

\subsection{Pompage (1 → 7)}

La pompe élève la pression du liquide :

\begin{equation}
    W_p = \dot{m}_{pri} \left( h_7 - h_1 \right)
    \label{eq:pompe}
\end{equation}

Le travail de la pompe reste négligeable devant la puissance thermique.

\subsection{Génération vapeur motrice (7 → 8)}

La chaudière solaire fournit la chaleur :

\begin{equation}
    \dot{Q}_{gen} = \dot{m}_{pri} \left( h_8 - h_7 \right)
    \label{eq:generateur}
\end{equation}

La vapeur atteint l’état saturé à $T_{gen} = 100^\circ C$.

\subsection{Éjecteur (8 → 4 → 5)}

L’éjecteur réalise trois transformations :

\begin{enumerate}
    \item Accélération dans la tuyère primaire (8 → 4)
    \item Mélange adiabatique (3 → 4)
    \item Recompression dans le diffuseur (4 → 5)
\end{enumerate}

Le taux d’entraînement est défini par :

\begin{equation}
    \mu = \frac{\dot{m}_{sec}}{\dot{m}_{pri}}
    \label{eq:mu}
\end{equation}

Les équations de conservation utilisées sont celles de la masse, de la quantité de mouvement et de l’énergie \cite{Huang1999}.

\subsection{Condensation (5 → 6)}

Dans le condenseur, le fluide rejette la chaleur :

\begin{equation}
    \dot{Q}_{cond} = \dot{m}_{tot} \left( h_5 - h_6 \right)
    \label{eq:condensation}
\end{equation}

Le fluide redevient liquide saturé à la pression $P_{cond}$.

\section{Indicateur de performance énergétique}

Le coefficient de performance énergétique est défini par :

\begin{equation}
    COP = \frac{\dot{Q}_{evap}}{\dot{Q}_{gen}}
    \label{eq:COP}
\end{equation}

Contrairement aux cycles à compression mécanique, le travail de la pompe est négligeable.

Les études antérieures rapportent des valeurs typiques de $COP$ comprises entre 0.2 et 0.5 pour les cycles à éjection vapeur \cite{Chunnanond2004}.
Le modèle développé dans ce travail permet d’atteindre des performances supérieures dans des conditions favorables.

\section{Résumé du chapitre}

Ce chapitre a présenté une description détaillée de la machine frigorifique à éjecteur fonctionnant au R718, en mettant l’accent sur les aspects historiques, scientifiques et technologiques. Les éléments clés abordés comprennent :

\begin{itemize}
    \item L’évolution historique des cycles à éjection
    \item Le positionnement environnemental du R718
    \item L’architecture détaillée de la machine
    \item La convention de numérotation adoptée
    \item Les indicateurs énergétiques fondamentaux
\end{itemize}

Elle constitue le socle de la modélisation mathématique détaillée développée dans les chapitres suivants.

\chapter{Le fluide frigorigène : l’eau (R718)}

\section{Introduction partielle}

Le choix du fluide frigorigène constitue l’un des déterminants majeurs des performances thermodynamiques, environnementales et technologiques d’un système frigorifique.
Historiquement, les premiers cycles frigorifiques exploitaient des fluides naturels tels que l’ammoniac (\(\mathrm{NH_3}\)), le dioxyde de carbone (\(\mathrm{CO_2}\)) ou l’eau, avant que les chlorofluorocarbures (CFC) ne dominent le marché au milieu du XX\textsuperscript{e} siècle.
Toutefois, les impacts environnementaux associés aux CFC puis aux hydrofluorocarbures (HFC) ont conduit à une transition progressive vers des fluides à faible potentiel de réchauffement global (GWP) \cite{Herold2016}.

Dans ce contexte, l’eau — référencée \textbf{R718} selon la nomenclature ASHRAE — présente un intérêt renouvelé.
Elle possède un GWP nul, un ODP nul, une non-toxicité, une ininflammabilité et une disponibilité universelle.
Cependant, son utilisation comme fluide frigorigène impose des contraintes physiques majeures liées à ses propriétés thermodynamiques intrinsèques, notamment une pression de saturation très faible aux températures usuelles de production de froid.

L’analyse approfondie de ces caractéristiques est indispensable pour comprendre les défis liés au fonctionnement sous vide poussé de la machine étudiée.

\section{Conditions thermodynamiques requises pour un fluide frigorigène}

Un fluide frigorigène adapté doit satisfaire plusieurs critères thermodynamiques fondamentaux, parmi lesquels :

\begin{itemize}
    \item une pression d’évaporation modérée (limitant le vide poussé) ;
    \item une chaleur latente élevée ;
    \item un faible volume spécifique en phase vapeur (limitant les débits volumiques) ;
    \item une température critique suffisamment élevée (élargissant le domaine opératoire).
\end{itemize}

D’un point de vue énergétique, l’effet frigorifique spécifique associé à un changement de phase à l’évaporateur peut s’exprimer, en première approximation, par :

\begin{equation}
    q_{\mathrm{evap}} = h_v - h_l,
    \label{eq:q_evap_def}
\end{equation}

où \(h_v\) est l’enthalpie de la vapeur saturée et \(h_l\) celle du liquide saturé.
Plus la différence \((h_v - h_l)\) est grande, plus la puissance frigorifique spécifique est élevée pour un débit massique donné.

L’eau présente une chaleur latente de vaporisation particulièrement élevée (de l’ordre de \(\approx 2400~\mathrm{kJ\,kg^{-1}}\) à \(10^\circ\mathrm{C}\)), ce qui constitue un avantage énergétique majeur \cite{Moran2014}.
La contrepartie se manifeste néanmoins par la relation pression--température, particulièrement défavorable au voisinage de \(10^\circ\mathrm{C}\).

\section{Relation pression--température et fonctionnement sous vide}

La pression de saturation d’un fluide pur est régie par l’équation de Clausius--Clapeyron :

\begin{equation}
    \frac{dP}{dT} = \frac{L}{T\left(v_v - v_l\right)},
    \label{eq:clausius_clapeyron}
\end{equation}

où \(L\) est la chaleur latente de changement de phase, et \(v_v\) et \(v_l\) sont respectivement les volumes spécifiques de la vapeur et du liquide.

Pour l’eau à \(10^\circ\mathrm{C}\), la pression de saturation est typiquement :

\begin{equation}
    P_{\mathrm{sat}}(10^\circ\mathrm{C}) \approx 1.23~\mathrm{kPa}.
    \label{eq:psat_10C}
\end{equation}

Ce niveau de pression est environ \(800\) fois inférieur à la pression atmosphérique standard, ce qui implique :

\begin{itemize}
    \item un fonctionnement sous vide profond ;
    \item un volume spécifique vapeur très élevé ;
    \item des vitesses d’écoulement \(c\) potentiellement élevées dans les conduites et l’éjecteur ;
    \item une sensibilité accrue aux pertes de charge et aux infiltrations d’air.
\end{itemize}

Ainsi, la principale difficulté associée au R718 en réfrigération n’est pas la capacité frigorifique intrinsèque, mais les contraintes mécaniques, hydrauliques et d’étanchéité qu’impose le vide poussé.

\section{Analyse du volume spécifique vapeur}

En première approche, le volume spécifique vapeur peut être estimé par le modèle du gaz parfait :

\begin{equation}
    v \approx \frac{R\,T}{P},
    \label{eq:ideal_gas_specific_volume}
\end{equation}

où \(R\) est la constante spécifique du gaz et \(T\) la température absolue.
Pour la vapeur d’eau, \(R \approx 461~\mathrm{J\,kg^{-1}\,K^{-1}}\).

À \(T \approx 283~\mathrm{K}\) et \(P \approx 1.23\times 10^3~\mathrm{Pa}\), on obtient :

\begin{equation}
    v \approx \frac{461 \times 283}{1230} \approx 106~\mathrm{m^3\,kg^{-1}}.
    \label{eq:vapor_specific_volume_approx}
\end{equation}

Cet ordre de grandeur met en évidence un débit volumique important pour des débits massiques pourtant modérés.
Les conséquences directes incluent :

\begin{itemize}
    \item un dimensionnement critique des sections d’écoulement ;
    \item un accroissement du risque de pertes de charge (et donc de dégradation des performances) ;
    \item un comportement fortement non linéaire de l’éjecteur vis-à-vis des variations de pression secondaire.
\end{itemize}

Des travaux de référence montrent en particulier que la performance d’un éjecteur devient très sensible aux variations de conditions lorsqu’on opère avec de grands volumes spécifiques vapeur \cite{Huang1999}.

\section{Avantages thermodynamiques du R718}

Malgré les contraintes associées au vide poussé, l’eau présente plusieurs avantages thermodynamiques et environnementaux :

\begin{enumerate}
    \item \textbf{Chaleur latente élevée :} elle permet de réduire le débit massique nécessaire pour une puissance frigorifique donnée.
    \item \textbf{Température critique élevée :} \(T_c \approx 374^\circ\mathrm{C}\), autorisant un large domaine d’exploitation thermique.
    \item \textbf{Capacité calorifique liquide élevée :} favorisant une stabilité thermique des phases liquides.
    \item \textbf{Compatibilité environnementale :} GWP = 0 et ODP = 0.
\end{enumerate}

Selon \citeauthor{Chunnanond2004} (\citeyear{Chunnanond2004}), les cycles à éjection vapeur utilisant l’eau sont particulièrement adaptés aux applications de froid solaire, notamment en raison de la compatibilité entre chaleur disponible à moyenne température et génération de vapeur motrice.

\section{Contraintes spécifiques du fonctionnement sous vide}

Le fonctionnement sous vide impose plusieurs exigences de conception et d’exploitation :

\begin{itemize}
    \item étanchéité renforcée de l’ensemble des composants ;
    \item gestion des infiltrations d’air et des gaz non condensables ;
    \item sélection de matériaux compatibles avec le vide et les gradients thermiques ;
    \item surveillance du risque de cavitation et des conditions d’aspiration de la pompe (NPSH).
\end{itemize}

À très basse pression, les problèmes de désamorçage, d’instabilité hydraulique et de cavitation peuvent compromettre la continuité de fonctionnement.
Ces contraintes justifient une modélisation rigoureuse des pertes de charge et des bilans énergétiques, ainsi qu’un dimensionnement attentif des conduites et des échangeurs.

\section{Comparaison avec un fluide HFC : analyse critique}

Le tableau~\ref{tab:r718_vs_r134a} présente une comparaison qualitative entre l’eau (R718) et un HFC courant (R134a), à titre illustratif.

\begin{table}[H]
    \centering
    \caption{Comparaison qualitative entre l’eau (R718) et le R134a.}
    \label{tab:r718_vs_r134a}
    \begin{tabular}{@{}p{4cm}p{4cm}p{4cm}@{}}
        \toprule
        \textbf{Propriété}                              & \textbf{Eau (R718)}           & \textbf{R134a}               \\ \midrule
        GWP                                             & 0                             & \(\approx 1430\)             \\
        Pression de saturation à \(10^\circ\mathrm{C}\) & \(\approx 1.23~\mathrm{kPa}\) & \(\approx 300~\mathrm{kPa}\) \\
        Chaleur latente                                 & Très élevée                   & Moyenne                      \\
        Volume spécifique vapeur                        & Très élevé                    & Faible                       \\ \bottomrule
    \end{tabular}
\end{table}

Il apparaît que le principal handicap du R718 est principalement \textbf{mécanique et hydraulique} (vide poussé et forts débits volumiques) plutôt qu’énergétique.
Ainsi, son adoption dépend davantage de la conception du système (étanchéité, dimensionnement, pertes de charge) que de ses performances thermiques intrinsèques.

\section{Conclusion partielle}

L’eau (R718) constitue un fluide frigorigène écologiquement irréprochable et thermodynamiquement performant en termes d’effet frigorifique spécifique.
Cependant, son utilisation à \(10^\circ\mathrm{C}\) impose un fonctionnement sous vide profond, entraînant des volumes spécifiques élevés et une forte sensibilité aux pertes de charge.

Ces particularités rendent indispensable :

\begin{itemize}
    \item une modélisation fine de l’éjecteur ;
    \item une attention particulière au dimensionnement hydraulique ;
    \item une analyse détaillée des phénomènes d’évaporation et de condensation.
\end{itemize}

La section suivante abordera le cœur dynamique du système : l’éjecteur de fluide, dont le comportement conditionne directement la stabilité et le \(COP\) de la machine.

\chapter[L'éjecteur de fluide]{L'éjecteur de fluide : revue technologique, phénoménologique et modélisation 1D}
\label{chap:ejecteur}


Dans une machine frigorifique à éjecteur, l’éjecteur remplace la compression mécanique par une recompression aérodynamique fondée sur la conversion d’énergie de pression en énergie cinétique, suivie d’une récupération partielle de pression statique dans un diffuseur. Dépourvu de pièces mobiles, ce dispositif présente des avantages importants en termes de robustesse, simplicité et coût de maintenance. En contrepartie, sa plage de fonctionnement est généralement étroite et sa performance est fortement sensible au couple de pressions \((P_{\mathrm{gen}},P_{\mathrm{cond}})\), ainsi qu’aux pertes de charge côté aspiration \cite{Chunnanond2004,Huang1999}.

Dans le cas du fluide frigorigène R718, la difficulté est renforcée par le fonctionnement sous vide profond côté évaporateur, impliquant de très grands volumes spécifiques vapeur. Les vitesses d’écoulement \(c\) peuvent alors devenir suffisamment élevées pour conduire localement à des régimes compressibles, voire supersoniques, avec possibilité de formation d’ondes de choc dans la section de mélange ou le diffuseur \cite{Moran2014,Anderson2016}. Une revue technologique et théorique solide constitue donc un préalable indispensable à la modélisation retenue dans ce mémoire.

\section{Technologie et typologie des éjecteurs}

\subsection{Architecture constructive}

Un éjecteur de vapeur simple effet utilisé en réfrigération comporte classiquement :
\begin{itemize}
    \item une \textbf{tuyère primaire} (souvent convergente-divergente de type Laval) alimentée par la vapeur motrice issue du générateur ;
    \item une \textbf{chambre d’aspiration} connectée à l’évaporateur ;
    \item une \textbf{section de mélange}, fréquemment assimilée à une section à aire constante (\emph{constant-area mixing section}) ;
    \item un \textbf{diffuseur} (souvent divergent) assurant la récupération de pression statique ;
    \item éventuellement une \textbf{géométrie variable} (aiguille, col réglable) permettant d’adapter la section critique et d’élargir la plage de fonctionnement \cite{Chunnanond2004}.
\end{itemize}

Les paramètres géométriques structurants incluent le diamètre au col \(d_t\), le rapport d’expansion de la tuyère, le diamètre de mélange \(d_m\), la longueur de mélange \(L_m\), ainsi que l’angle et la longueur du diffuseur. La littérature montre notamment que le ratio \(L_m/d_m\) et la position relative de la sortie de tuyère (nozzle exit position) influencent fortement le rapport d’entraînement \(\mu\) et la pression critique de refoulement \cite{Huang1999,Eames1995}.

\subsection{Typologie fonctionnelle}

Les éjecteurs peuvent être classés selon :
\begin{itemize}
    \item la géométrie : éjecteur à géométrie fixe vs. géométrie variable ;
    \item le schéma de mélange : mélange à pression constante (\emph{constant-pressure mixing}) vs. mélange à aire constante (\emph{constant-area mixing}) ;
    \item le régime : régime critique (double étranglement) vs. régime subcritique (dépendant de \(P_{\mathrm{cond}}\)) \cite{Chunnanond2004}.
\end{itemize}

Pour une approche de modélisation système, la formulation « aire constante » est souvent retenue, car elle conduit à une structure 1D robuste et comparable aux formulations de référence \cite{Huang1999}.

\subsection{Contraintes technologiques spécifiques au R718}

Avec l’eau, les contraintes suivantes dominent :
\begin{itemize}
    \item côté aspiration, \(P_{\mathrm{evap}}\) est typiquement de l’ordre du kPa à \(10^\circ\mathrm{C}\), imposant une excellente étanchéité et une gestion des non-condensables ;
    \item les pertes de charge doivent être minimisées, car elles dégradent directement le rapport de pression disponible pour l’éjecteur ;
    \item la conception doit tenir compte des effets de corrosion, de dégazage et de purge des gaz résiduels \cite{Sokolov1990}.
\end{itemize}

\section{Rôle systémique et repérage des états du cycle}

Dans la convention adoptée dans ce mémoire, l’éjecteur est localisé entre la sortie évaporateur et l’entrée condenseur, et ses états internes sont explicitement distingués :
\begin{itemize}
    \item \(3 \rightarrow 4\) : aspiration et mélange (chambre de mélange),
    \item \(4 \rightarrow 5\) : recompression (diffuseur),
    \item \(8 \rightarrow 4\) : tuyère primaire (expansion du fluide motrice).
\end{itemize}

L’éjecteur assure la fonction : aspirer la vapeur secondaire à \(P_{\mathrm{evap}}\), l’entraîner, puis recomprimer le mélange jusqu’à un niveau de pression compatible avec le condenseur.

Le paramètre de performance le plus utilisé est le \textbf{rapport d’entraînement} :
\begin{equation}
    \mu = \frac{\dot{m}_{\mathrm{sec}}}{\dot{m}_{\mathrm{pri}}}.
    \label{eq:mu_def}
\end{equation}

À l’échelle de la machine (simple effet), et en négligeant le travail de pompe devant la puissance thermique, une écriture compacte du \(COP\) énergétique est :
\begin{equation}
    COP \simeq \frac{\dot{Q}_{\mathrm{evap}}}{\dot{Q}_{\mathrm{gen}}}
    = \frac{\dot{m}_{\mathrm{sec}}(h_3-h_2)}{\dot{m}_{\mathrm{pri}}(h_8-h_7)}
    = \mu \,\frac{(h_3-h_2)}{(h_8-h_7)}.
    \label{eq:cop_mu}
\end{equation}

Cette relation met en évidence que, toutes choses égales par ailleurs, l’augmentation de \(\mu\) améliore le \(COP\). Toutefois, \(\mu\) est fortement affecté par \(P_{\mathrm{cond}}\) et par l’existence d’une pression critique au-delà de laquelle l’éjecteur décroche \cite{Eames1995,Huang1999}.

\section{Écoulement compressible dans la tuyère primaire : étranglement et relations isentropiques}

\subsection{Hypothèses de base}

On considère un écoulement quasi-1D, stationnaire, adiabatique, sans travail de paroi. La tuyère convertit une partie de l’enthalpie en énergie cinétique, conduisant à des vitesses \(c\) élevées. En formulation d’énergie totale :
\begin{equation}
    h + \frac{c^2}{2} = h_0 = \text{constante} \qquad (\text{adiabatique, stationnaire}).
    \label{eq:energy_total_nozzle}
\end{equation}

\subsection{Étranglement au col (\emph{choking})}

Lorsque le rapport de pression amont/aval dépasse une valeur critique, l’écoulement atteint la condition sonique au col (\(M=1\)) et le débit devient pratiquement indépendant de la pression aval : c’est l’étranglement. Pour un gaz parfait, la relation aire--Mach s’écrit \cite{Anderson2016} :
\begin{equation}
    \frac{A}{A^\star} =
    \frac{1}{M}
    \left[
        \frac{2}{\gamma+1}\left(1+\frac{\gamma-1}{2}M^2\right)
        \right]^{\frac{\gamma+1}{2(\gamma-1)}}.
    \label{eq:area_mach}
\end{equation}

Dans le cas du R718, \(\gamma\) n’est ni constant ni parfaitement adapté au gaz parfait sous vide et proche de la saturation. En pratique, la stratégie retenue consiste à conserver la structure quasi-1D mais à \textbf{fermer le modèle par des propriétés réelles} (enthalpie/entropie/densité) calculées avec CoolProp \cite{Bell2014}.

\subsection{Rendement isentropique de tuyère}

Pour représenter les pertes (viscosité, non-uniformités, chocs faibles), on introduit un rendement isentropique de tuyère :
\begin{equation}
    \eta_{\mathrm{tuy}} =
    \frac{h_8 - h_{4s}}{h_8 - h_4},
    \label{eq:eta_tuy}
\end{equation}
où \(4s\) est l’état « isentropique » à la pression de sortie réelle \(P_4\). Conformément à la littérature 1D, des valeurs typiques de \(0.8\) à \(0.9\) sont couramment adoptées \cite{Huang1999,Chunnanond2004}. Dans la simulation, on retient une valeur nominale \(\eta_{\mathrm{tuy}}=0.85\).

\section{Modélisation du mélange : bilans 1D en section à aire constante}

\subsection{Conservation de la masse}

Dans la chambre de mélange (état \(3 \rightarrow 4\)), la conservation de la masse impose :
\begin{equation}
    \dot{m}_{\mathrm{pri}} + \dot{m}_{\mathrm{sec}} = \dot{m}_{\mathrm{mix}}.
    \label{eq:mass_mixing}
\end{equation}

\subsection{Conservation de la quantité de mouvement}

Sur un volume de contrôle 1D englobant la section de mélange (aire constante), la forme intégrale stationnaire peut être écrite, en regroupant les efforts dissipatifs (couche limite, turbulence) sous une forme équivalente :
\begin{equation}
    \dot{m}_{\mathrm{pri}}\,c_{4,\mathrm{pri}} + \dot{m}_{\mathrm{sec}}\,c_{3}
    =
    \dot{m}_{\mathrm{mix}}\,c_{4}
    + \Delta F_{\mathrm{pertes}}.
    \label{eq:momentum_mixing}
\end{equation}

Dans les modèles 1D, \(\Delta F_{\mathrm{pertes}}\) est souvent représenté par un coefficient semi-empirique de mélange/dispersion, dont l’ajustement se fait par calibration ou analyse de sensibilité \cite{Huang1999,Chunnanond2004}.

\subsection{Bilan d’énergie (mélange adiabatique)}

Sous hypothèse adiabatique, le bilan d’énergie (en enthalpie) conduit à :
\begin{equation}
    \dot{m}_{\mathrm{pri}}\,h_{4,\mathrm{pri}} + \dot{m}_{\mathrm{sec}}\,h_{3}
    =
    \dot{m}_{\mathrm{mix}}\,h_{4}.
    \label{eq:energy_mixing}
\end{equation}

Cette relation donne l’enthalpie moyenne du mélange. La détermination cohérente des vitesses \(c\) mobilise également la conservation de l’énergie totale (Eq.~\ref{eq:energy_total_nozzle}) dans chaque sous-domaine.

\section{Ondes de choc et relation de Hugoniot : choc normal dans l’éjecteur}

\subsection{Origine physique du choc}

Lorsque la tuyère primaire produit un jet supersonique et que la récupération de pression imposée par le diffuseur devient trop importante, l’écoulement peut subir un choc normal (ou un train de chocs). Celui-ci ramène le Mach à une valeur subsonique, au prix d’une augmentation d’entropie et d’une perte de pression totale. Les conséquences principales sont :
\begin{itemize}
    \item limitation du rapport de pression atteignable ;
    \item apparition d’une pression critique au condenseur ;
    \item chute de \(\mu\) et dégradation du \(COP\) \cite{Anderson2016,Eames1995}.
\end{itemize}

\subsection{Équations de Rankine--Hugoniot (forme générale)}

Pour éviter toute confusion avec les états du cycle numérotés \(1\) à \(8\), on note ici les grandeurs amont/aval du choc par les indices \(\mathrm{am}\) (amont) et \(\mathrm{av}\) (aval). Pour un choc normal stationnaire, les lois de conservation donnent :

\paragraph{Conservation de la masse}
\begin{equation}
    \rho_{\mathrm{am}}\,c_{\mathrm{am}} = \rho_{\mathrm{av}}\,c_{\mathrm{av}}.
    \label{eq:shock_mass}
\end{equation}

\paragraph{Conservation de la quantité de mouvement}
\begin{equation}
    P_{\mathrm{am}} + \rho_{\mathrm{am}}\,c_{\mathrm{am}}^{2}
    =
    P_{\mathrm{av}} + \rho_{\mathrm{av}}\,c_{\mathrm{av}}^{2}.
    \label{eq:shock_momentum}
\end{equation}

\paragraph{Conservation de l’énergie}
\begin{equation}
    h_{\mathrm{am}} + \frac{c_{\mathrm{am}}^{2}}{2}
    =
    h_{\mathrm{av}} + \frac{c_{\mathrm{av}}^{2}}{2}.
    \label{eq:shock_energy}
\end{equation}

La relation de Hugoniot peut alors être mise sous la forme \cite{Anderson2016} :
\begin{equation}
    h_{\mathrm{av}} - h_{\mathrm{am}}
    =
    \frac{1}{2}\left(P_{\mathrm{av}}-P_{\mathrm{am}}\right)\left(v_{\mathrm{am}}+v_{\mathrm{av}}\right),
    \label{eq:hugoniot}
\end{equation}
avec \(v = 1/\rho\). Cette relation met en évidence que le choc est adiabatique mais non isentropique (\(s_{\mathrm{av}} > s_{\mathrm{am}}\)), donc intrinsèquement dissipatif.

\subsection{Relations Mach (gaz parfait) et adaptation au R718 réel}

Pour un gaz parfait, les relations classiques donnent \cite{Anderson2016} :
\begin{equation}
    \frac{P_{\mathrm{av}}}{P_{\mathrm{am}}}
    =
    1+\frac{2\gamma}{\gamma+1}\left(M_{\mathrm{am}}^{2}-1\right),
    \qquad
    M_{\mathrm{av}}^{2}
    =
    \frac{1+\frac{\gamma-1}{2}M_{\mathrm{am}}^{2}}
    {\gamma M_{\mathrm{am}}^{2}-\frac{\gamma-1}{2}}.
    \label{eq:shock_mach_gp}
\end{equation}

Dans un modèle réaliste du R718 sous vide, l’hypothèse \(\gamma=\) constante est discutable. La stratégie robuste retenue dans ce mémoire consiste à :
\begin{itemize}
    \item conserver la structure de saut (Eqs.~\ref{eq:shock_mass}--\ref{eq:shock_energy}) ;
    \item fermer le problème avec les \textbf{propriétés réelles} via CoolProp (\(h(P,T)\), \(s(P,T)\), \(\rho(P,T)\), etc.) \cite{Bell2014}.
\end{itemize}

\section{Régime critique, double étranglement et pression critique}

\subsection{Étranglement primaire}

La tuyère primaire atteint typiquement \(M=1\) au col en régime étranglé, ce qui stabilise partiellement \(\dot{m}_{\mathrm{pri}}\) vis-à-vis de la pression aval. On peut alors écrire :
\begin{equation}
    \dot{m}_{\mathrm{pri}} = f\!\left(P_{\mathrm{gen}},T_{\mathrm{gen}},A_t\right).
    \label{eq:mdot_primary}
\end{equation}

\subsection{Étranglement secondaire et régime critique}

Dans certains régimes, la vapeur secondaire peut également atteindre une condition sonique au voisinage de l’entrée de la zone de mélange. Le double étranglement correspond au régime critique : \(\mu\) tend vers une valeur maximale et l’écoulement devient moins sensible à \(P_{\mathrm{cond}}\).

\subsection{Pression critique de condensation}

La pression critique \(P_{\mathrm{cond,crit}}\) est la pression maximale au condenseur pour laquelle le régime critique persiste. Au-delà, le choc se déplace en amont, l’étranglement secondaire disparaît et \(\mu\) chute fortement, ce qui explique le décrochage observé expérimentalement \cite{Eames1995}.

\section{Modèle mathématique 1D hybride retenu pour la simulation}

\subsection{Structure du modèle}

Le modèle retenu se compose de trois sous-modèles :
\begin{enumerate}
    \item \textbf{Tuyère primaire} (\(8 \rightarrow 4\)) : expansion quasi-1D + rendement \(\eta_{\mathrm{tuy}}\).
    \item \textbf{Mélange} (\(3 \rightarrow 4\)) : section à aire constante, bilans masse--moment--énergie, avec correction semi-empirique.
    \item \textbf{Diffusion} (\(4 \rightarrow 5\)) : récupération de pression + possibilité de choc normal + rendement diffuseur \(\eta_{\mathrm{diff}}\).
\end{enumerate}

\subsection{Fermeture par rendements et paramètres semi-empiriques}

Les modèles 1D ne résolvent ni turbulence ni non-uniformités. Des coefficients de performance sont donc nécessaires \cite{Chunnanond2004,Huang1999}. Dans ce travail, les valeurs nominales adoptées sont :
\begin{equation}
    \eta_{\mathrm{tuy}} = 0.85,
    \qquad
    \eta_{\mathrm{diff}} = 0.75.
    \label{eq:etas_nominal}
\end{equation}

\subsection{Propriétés réelles via CoolProp}

Toutes les propriétés thermodynamiques (par exemple \(h\), \(s\), \(\rho\)) sont calculées par CoolProp afin :
\begin{itemize}
    \item de respecter la thermodynamique réelle de l’eau sous vide ;
    \item d’éviter les incohérences d’un modèle gaz parfait ;
    \item de garantir la cohérence sur diagrammes \(P\text{-}h\) et \(T\text{-}s\) \cite{Bell2014}.
\end{itemize}

\section{Comparaison critique des principaux modèles 1D d’éjecteurs}

\begin{table}[H]
    \centering
    \small
    \caption{Comparaison synthétique de modèles 1D d'éjecteurs issus de la littérature.}
    \label{tab:comparaison_modeles_1D}
    \begin{tabular}{@{}p{2.5cm}p{2.5cm}p{1.5cm}p{1.5cm}p{3cm}p{3.2cm}@{}}
        \toprule
        \textbf{Auteur}                   & \textbf{Hypothèse mélange}   & \textbf{Choc}  & \textbf{Fluide réel} & \textbf{Points forts}               & \textbf{Limites}                            \\ \midrule
        Eames et al. (1995)               & Pression constante           & Oui            & Non                  & Validation expérimentale            & Hypothèse gaz parfait et simplifications    \\
        Huang et al. (1999)               & Aire constante               & Oui            & Non                  & Structure analytique robuste        & Calibration nécessaire                      \\
        Chunnanond \& Aphornratana (2004) & Variable                     & Oui            & Partielle            & Synthèse complète                   & Approche souvent semi-empirique             \\
        Approche présente                 & Aire constante + corrections & Oui (Hugoniot) & Oui (CoolProp)       & Cohérence thermodynamique sous vide & Complexité numérique, besoin de diagnostics \\ \bottomrule
    \end{tabular}
\end{table}

\section*{Résumé du chapitre}
\addcontentsline{toc}{section}{Résumé du chapitre}

Ce chapitre a établi une base technologique et théorique complète pour l’éjecteur, composant central de la machine frigorifique au R718. La revue a montré que son fonctionnement combine : (i) une détente accélératrice dans la tuyère primaire susceptible d’atteindre le régime supersonique, (ii) un mélange adiabatique fortement dissipatif et (iii) une recompression dans le diffuseur, potentiellement gouvernée par la formation d’ondes de choc.

Sur le plan de la modélisation, les équations de conservation en régime quasi-1D (masse, quantité de mouvement, énergie) constituent le socle, tandis que la relation de Hugoniot formalise la discontinuité associée au choc normal et explicite l’irréversibilité (augmentation d’entropie). Le chapitre a également discuté le régime critique et la notion de pression critique de condensation, qui expliquent le décrochage et la chute de performance observés dans la littérature.

Enfin, un modèle 1D hybride a été retenu : tuyère (avec rendement \(\eta_{\mathrm{tuy}}\)), mélange à aire constante (avec correction semi-empirique) et diffuseur (avec \(\eta_{\mathrm{diff}}\) et choc éventuel), le tout fermé par des propriétés thermodynamiques réelles calculées via CoolProp. Ce modèle constitue la base de l’implémentation numérique et du couplage système développés dans la Partie II.


% ==========================================================
% Fichier : chapters/part1/ch4_detendeur.tex
% Chapitre I.4 — Le détendeur
% Convention : vitesse = c (pas V) ; coefficient global échange = K (pas U)
% Numérotation/notation du cycle (convention validée) :
% 1->2 : Détendeur ; 2->3 : Evaporateur ; 3->4 : Mélange éjecteur ;
% 4->5 : Diffuseur éjecteur ; 5->6 : Condenseur ; 1->7 : Pompe ;
% 7->8 : Chaudière ; 8->4 : Tuyère éjecteur
% ==========================================================

\chapter[Le détendeur]{Le détendeur}
\label{chap:detendeur}

Dans un cycle frigorifique, le détendeur (ou organe de laminage) est classiquement présenté comme un composant passif assurant une chute de pression entre le niveau de condensation et le niveau d’évaporation, sans production de travail utile. Cette représentation est correcte d’un point de vue énergétique global, mais elle devient \emph{insuffisante} dès lors que l’on s’intéresse à la dynamique du cycle, à la stabilité hydraulique et aux régimes diphasiques. Dans le cas spécifique d’une machine à eau (R718) opérant sous \emph{vide profond} côté évaporateur (pression typiquement de l’ordre du kPa à \(T_{\mathrm{evap}}\approx 10^\circ\mathrm{C}\)), le détendeur cesse d’être un simple « abaisseur de pression » : il conditionne la qualité diphasique en entrée d’évaporateur, influence la capacité d’aspiration de l’éjecteur, et peut devenir une source d’instabilités (oscillations de débit/pression, flash-boiling, cavitation, étranglement diphasique) \cite{Moran2014,CollierThome1994,Whalley1987}.

Sur le plan thermodynamique, le détendeur impose la transformation \emph{isoenthalpique} reliant le liquide haute pression (sortie condenseur) au mélange diphasique basse pression (entrée évaporateur). Or, pour le R718, la chute de pression est très élevée en ratio (par exemple \(P_{\mathrm{cond}}\sim 5.6~\mathrm{kPa}\) vers \(P_{\mathrm{evap}}\sim 1.2~\mathrm{kPa}\)), et la température de saturation varie fortement, favorisant un flash de vaporisation immédiat. Cette vaporisation instantanée accroît l'entropie, modifie brutalement la densité, et peut induire des vitesses locales élevées dans l'orifice (donc un risque d'étranglement) \cite{Thome2004,Whalley1987}.

Cette section propose ainsi un \textbf{état de l’art approfondi} (typologies, phénomènes, modélisation) puis une \textbf{formalisation mathématique} cohérente avec l’objectif du mémoire : un modèle 1D robuste, intégrable au simulateur Python (propriétés réelles via CoolProp), et compatible avec les diagrammes \(P\text{-}h\) et \(T\text{-}s\).
% saut de pae
\newpage
\section{Rôle systémique et interaction avec le cycle à éjecteur}
\label{sec:detendeur:role}

Dans la convention de numérotation adoptée, le détendeur réalise la transformation \((1\rightarrow 2)\) et sert d’interface entre :
\begin{itemize}
    \item le \textbf{niveau de condensation} (liquide saturé ou légèrement sous-refroidi en 1) ;
    \item le \textbf{niveau d’évaporation} (mélange diphasique en 2).
\end{itemize}

Ses fonctions systémiques sont donc :
\begin{enumerate}
    \item \textbf{Imposer \(P_{\mathrm{evap}}\)} (ou plus exactement, créer une perte de charge contrôlée qui fixe la pression amont/aval selon la boucle hydraulique).
    \item \textbf{Créer un état diphasique} compatible avec l’évaporateur \((2\rightarrow 3)\), caractérisé par un titre \(x_2\) déterminant pour le transfert thermique.
    \item \textbf{Conditionner le débit secondaire} aspiré par l'éjecteur, puisque le débit évaporateur et l'état vapeur en 3 gouvernent la capacité d'entraînement \(\mu\) et la stabilité du mélange \((3\rightarrow 4)\) \cite{Chunnanond2004,Huang1999}.
\end{enumerate}

Dans un cycle à éjecteur, l’équilibre global résulte d’un couplage non linéaire :
\[
    \text{détendeur} \Rightarrow (P_2,x_2,\dot{m}_{\mathrm{sec}})\Rightarrow \text{évaporateur} \Rightarrow \text{éjecteur} \Rightarrow P_{\mathrm{cond}} \Rightarrow \text{détendeur}.
\]
Ainsi, une variation de perte de charge au détendeur peut déplacer le point de fonctionnement du cycle (et potentiellement provoquer un décrochage de l’éjecteur si la récupération de pression devient insuffisante).

\section{Typologie technologique des détendeurs et choix pertinent pour le R718}
\label{sec:detendeur:typologie}

\subsection{Grandes familles d’organes de détente}
\label{subsec:detendeur:families}

La littérature distingue plusieurs familles principales \cite{ASHRAEHandbook2018,Thome2004} :
\begin{itemize}
    \item \textbf{Orifice fixe} (plaque à trou, buse, gicleur) : simplicité, robustesse, mais régulation limitée.
    \item \textbf{Tube capillaire} : perte de charge distribuée, sensible aux conditions, historiquement utilisé en petites puissances.
    \item \textbf{Détendeur thermostatique (TXV)} : régulation du surchauffe via bulbe, large plage mais mécanique plus complexe.
    \item \textbf{Détendeur électronique (EEV)} : commande fine, adapté aux systèmes variables, nécessite instrumentation et contrôle.
\end{itemize}

\subsection{Contraintes spécifiques du R718 (vide, non-condensables, volumes spécifiques)}
\label{subsec:detendeur:r718_constraints}

Pour l’eau comme fluide frigorigène, plusieurs contraintes orientent le choix technologique :
\begin{enumerate}
    \item \textbf{Faibles pressions absolues côté évaporateur} : l'infiltration d'air et la présence de non-condensables peuvent dégrader la performance et perturber la détente \cite{Sokolov1990}.
    \item \textbf{Flash-boiling intense} après l’orifice : le mélange peut présenter un fort gradient de densité et de fraction volumique vapeur, rendant la dynamique instable.
    \item \textbf{Sensibilité aux pertes de charge} : toute perte additionnelle sur la ligne basse pression pénalise l’aspiration et le rapport de pression de l’éjecteur.
\end{enumerate}

Dans un prototype de machine solaire à éjecteur, un \textbf{orifice fixe} ou une \textbf{géométrie simple} est souvent retenu pour assurer robustesse et facilité de modélisation, quitte à effectuer un dimensionnement/ajustement ultérieur (c’est l’approche suivie dans l’outil développé ici).

\section{Fondements thermodynamiques du laminage}
\label{sec:detendeur:thermo}

\subsection{Démonstration de l’isoenthalpie à partir du premier principe (écoulement stationnaire)}
\label{subsec:detendeur:isoenthalpie}

Considérons un volume de contrôle englobant le détendeur. L’équation d’énergie pour un écoulement stationnaire (en négligeant la variation d’énergie potentielle) s’écrit :
\begin{equation}
    \dot{Q} - \dot{W} + \dot{m}\left(h_1 + \frac{c_1^2}{2}\right)
    =
    \dot{m}\left(h_2 + \frac{c_2^2}{2}\right).
    \label{eq:detendeur_steady_energy}
\end{equation}

Dans un détendeur idéal :
\begin{itemize}
    \item \(\dot{Q}\simeq 0\) (adiabatique),
    \item \(\dot{W}=0\) (aucun travail d’arbre),
    \item les variations de vitesse restent souvent secondaires au regard des variations thermodynamiques macroscopiques (on les discute toutefois à la Section~\ref{sec:detendeur:hydraulics}).
\end{itemize}
On obtient alors l’approximation classique :
\begin{equation}
    h_1 \approx h_2.
    \label{eq:detendeur_isenthalpic}
\end{equation}

\subsection{Irréversibilité et entropie}
\label{subsec:detendeur:entropy}

Le laminage est fortement irréversible : la pression chute sans production de travail utile, ce qui correspond à une dissipation interne. Thermodynamiquement, cela se traduit par :
\begin{equation}
    s_2 > s_1.
    \label{eq:detendeur_entropy_increase}
\end{equation}
Sur un diagramme \(T\text{-}s\), la transformation \((1\rightarrow 2)\) apparaît donc comme une évolution vers des entropies plus élevées, en cohérence avec la production d'entropie (perte d'énergie disponible) \cite{Moran2014}.

\section{Formation du mélange diphasique en sortie}
\label{sec:detendeur:flash}

\subsection{Expression du titre vapeur en sortie (équilibre thermodynamique)}
\label{subsec:detendeur:quality}

Si l’état aval est diphasique à la pression \(P_2=P_{\mathrm{evap}}\), on peut relier le titre \(x_2\) à l’enthalpie isoenthalpique :
\begin{equation}
    x_2 =
    \frac{h_2 - h_\ell(P_2)}{h_v(P_2) - h_\ell(P_2)},
    \qquad \text{avec } h_2 = h_1.
    \label{eq:detendeur_quality}
\end{equation}
où \(h_\ell(P_2)\) et \(h_v(P_2)\) sont respectivement les enthalpies saturées liquide et vapeur à \(P_2\).

Cette relation est essentielle car \(x_2\) conditionne :
\begin{itemize}
    \item la part de chaleur latente à fournir dans l’évaporateur \((2\rightarrow 3)\),
    \item l’hydrodynamique (fraction volumique vapeur, densité moyenne),
    \item la stabilité (sensibilité aux pertes et aux fluctuations de pression).
\end{itemize}

\subsection{Équilibre vs non-équilibre : limites du modèle « homogène »}
\label{subsec:detendeur:noneq}

Dans un détendeur réel, la détente est rapide et peut conduire à des états transitoires de non-équilibre (retard de nucléation, surchauffe métastable, vaporisation localisée), particulièrement lors de \emph{flash-boiling}. L'état d'équilibre (utilisé dans l'équation~\ref{eq:detendeur_quality}) demeure néanmoins une approximation de référence pour un modèle système 1D, à condition de reconnaître ses limites \cite{CollierThome1994,Whalley1987}. Dans l'outil de simulation, cette approximation est retenue pour garantir :
\begin{itemize}
    \item une cohérence thermodynamique robuste,
    \item une stabilité numérique,
    \item une intégration simple au couplage global.
\end{itemize}

\section{Hydraulique du détendeur}
\label{sec:detendeur:hydraulics}

\subsection{Approche orifice : débit monophasique (référence)}
\label{subsec:detendeur:orifice_single}

Pour un orifice court et un écoulement liquide (amont), l’expression de base issue de Bernoulli avec coefficient de décharge \(C_d\) est :
\begin{equation}
    \dot{m} = C_d\,A\,\sqrt{2\,\rho_1\,(P_1 - P_2)},
    \label{eq:detendeur_orifice_single}
\end{equation}
où \(A\) est la section de passage et \(\rho_1\) la masse volumique amont.

Cette expression est utile comme \emph{repère}, mais elle devient insuffisante lorsque la détente génère un mélange diphasique et/ou lorsque l'écoulement devient critique \cite{Whalley1987}.

\subsection{Cavitation et nombre de cavitation}
\label{subsec:detendeur:cavitation}

La cavitation correspond à l’apparition de bulles lorsque la pression locale chute sous la pression de saturation. Un critère classique (défini pour les écoulements internes) s’exprime via le nombre de cavitation :
\begin{equation}
    \sigma = \frac{P_{\mathrm{am}} - P_{\mathrm{sat}}}{\tfrac{1}{2}\rho\,c^2}.
    \label{eq:detendeur_sigma}
\end{equation}

Dans un système au R718 sous vide, \(P_{\mathrm{sat}}\) est faible et la marge de pression disponible est réduite : \(\sigma\) peut devenir très faible, ce qui augmente la probabilité d'instabilités hydrodynamiques (oscillations de débit, bruit, fluctuations de pression). D'un point de vue système, ces instabilités peuvent perturber l'évaporateur et le fonctionnement de l'éjecteur \cite{CollierThome1994}.

\subsection{Écoulement critique en détente diphasique : notion et implications}
\label{subsec:detendeur:critical}

Lorsque le gradient de pression est important, un écoulement diphasique peut atteindre une condition \emph{critique} : l'augmentation de la baisse de pression aval n'augmente plus le débit massique. Cette notion (bien connue en sécurité des dépressurisations et en détente de fluides) est particulièrement importante pour les détentes avec flash-boiling \cite{Whalley1987}.

Les modèles de débit critique diphasique sont nombreux. Deux approches dominantes en modélisation système :
\begin{itemize}
    \item \textbf{HEM} (Homogeneous Equilibrium Model) : phases en équilibre, même vitesse (\emph{slip} nul), propriétés mélangées ; robuste et simple.
    \item \textbf{Modèles à glissement} (non-homogènes) : vitesse vapeur \(\neq\) vitesse liquide ; plus réalistes, mais nécessitent corrélations supplémentaires.
\end{itemize}

Dans un mémoire orienté « simulation système 1D », le HEM est souvent privilégié comme premier niveau de modélisation car il garantit la cohérence thermodynamique et reste numériquement stable \cite{Whalley1987}.

\section{Instabilités dynamiques et couplage avec les composants aval/amont}
\label{sec:detendeur:instabilities}

Même si le bilan énergétique du détendeur est simple (\(h_1\simeq h_2\)), son comportement dynamique peut être non linéaire. Les mécanismes typiques incluent :
\begin{itemize}
    \item \textbf{oscillations de pression} liées à la compressibilité apparente du mélange diphasique ;
    \item \textbf{instabilités de type relaxation} (nucleation retardée puis flash brusque) ;
    \item \textbf{couplage avec pertes de charge aval} : une variation de \(\Delta P\) sur la ligne d’évaporation modifie le débit et la qualité, ce qui modifie à son tour la charge thermique et la pression d’évaporation.
\end{itemize}

Dans une machine à éjecteur, ces instabilités peuvent se propager vers l’éjecteur via l’état 3 (vapeur secondaire), et déplacer le point de fonctionnement global (\(\mu\), pression de mélange, récupération). C’est pourquoi, même si la modélisation retenue est « simple », le détendeur doit être traité comme un organe critique de stabilité.

\section{Modèle mathématique retenu pour la simulation (niveau mémoire)}
\label{sec:detendeur:model}

Conformément au positionnement du mémoire (modélisation 1D + option semi-empirique), le détendeur est modélisé par :
\begin{enumerate}
    \item \textbf{Transformation isoenthalpique} : équation~\ref{eq:detendeur_isenthalpic}.
    \item \textbf{Calcul de l'état aval} à \((P_2,h_2)\) via propriétés réelles (CoolProp).
    \item \textbf{Option débit « orifice »} (si activée) de type équation~\ref{eq:detendeur_orifice_single}, avec \(C_d\) et \(A\) paramétrables.
    \item \textbf{Diagnostics} (cohérence physique) : \(\Delta P>0\), détection vide profond, indicateur sortie diphasique, etc.
\end{enumerate}

Ce choix constitue un compromis : suffisamment physique pour garantir une cohérence thermodynamique et permettre l’intégration au couplage global, tout en restant compatible avec l’objectif principal du projet (développement d’un outil de simulation modulaire et testable).

\section{Résumé du chapitre}
\label{sec:detendeur:resume}

Ce chapitre a établi que, dans une machine frigorifique à éjecteur fonctionnant au R718, le détendeur ne se réduit pas à une simple chute de pression. Il impose la transformation isoenthalpique \((1\rightarrow 2)\), détermine la qualité diphasique \(x_2\), et influence la stabilité du couple évaporateur–éjecteur. L’état de l’art met en évidence l’importance du flash-boiling, des régimes critiques diphasiques et du risque d’instabilités sous vide profond. Pour la simulation système 1D, un modèle isoenthalpique cohérent thermodynamiquement, enrichi d’une option de débit de type orifice et de diagnostics, est retenu afin d’assurer robustesse numérique et intégration au couplage global.

% ==========================================================
% Références citées (clés BibTeX attendues)
% - ASHRAEHandbookRefrigeration
% - MoranShapiro2014
% - CollierThome1994
% - Whalley1987
% - Thome2004
% - ChunnanondAphornratana2004
% - Huang1999
% - SokolovHershgal1990
% ==========================================================

% ==========================================================
% Fichier : chapters/part1/ch5_condenseur.tex
% Chapitre I.5 — Le condenseur
% Norme citations : APA via \citep{}
% Convention notations : vitesse = c ; coefficient global = K ; surface = A
% Convention cycle (validée) :
% 1->2 : Détendeur ; 2->3 : Evaporateur ; 3->4 : Mélange éjecteur ;
% 4->5 : Diffuseur éjecteur ; 5->6 : Condenseur ; 1->7 : Pompe ;
% 7->8 : Chaudière ; 8->4 : Tuyère éjecteur
% ==========================================================

\chapter[Le condenseur]{Le condenseur}
\label{chap:condenseur}

Le condenseur constitue l’organe de fermeture thermique du cycle frigorifique à éjecteur. Il assure la transformation du fluide en sortie d’éjecteur (état 5) vers un liquide saturé (état 6), en rejetant vers l’environnement la puissance thermique totale du cycle. À l’échelle du système, on a classiquement :
\begin{equation}
    \dot{Q}_{\mathrm{cond}} \simeq \dot{Q}_{\mathrm{gen}} + \dot{Q}_{\mathrm{evap}} + \dot{W}_{\mathrm{pump}},
    \label{eq:qcond_balance}
\end{equation}
où \(\dot{W}_{\mathrm{pump}}\) est généralement négligeable devant les puissances thermiques, ce qui est cohérent avec les résultats de simulation obtenus.

Dans une machine à éjecteur, \(P_{\mathrm{cond}}\) et \(T_{\mathrm{cond}}\) ne sont pas seulement des sorties thermiques : ils gouvernent directement la recompression aérodynamique de l'éjecteur. Une hausse de \(P_{\mathrm{cond}}\) peut déplacer la zone de choc, réduire \(\mu\), voire provoquer un décrochage lorsque la pression critique est dépassée \cite{Huang1999,Chunnanond2004,Anderson2016}. L'étude du condenseur revêt ainsi une importance \textbf{systémique} : il est à la fois l'échangeur de rejet et un \textbf{verrou de stabilité}.

Dans le cas présent, le condenseur est envisagé côté air en \textbf{convection naturelle}, sans ventilation forcée. Cette option simplifie la conception et réduit la consommation auxiliaire, mais implique des coefficients de convection faibles, une dépendance forte au climat (température et vitesse d'air), et donc une sensibilité accrue de \(P_{\mathrm{cond}}\) \cite{Kalogirou2014}.
\newpage
\section{Rôle systémique dans le cycle à éjecteur}
\label{sec:cond:role}

Le condenseur fixe simultanément :
\begin{itemize}
    \item le \textbf{niveau de pression haute} du cycle : \(P_{\mathrm{cond}}\),
    \item la \textbf{température de rejet} \(T_{\mathrm{cond}}\),
    \item la \textbf{condition d’alimentation du détendeur} (état 6), donc l’état \((1\rightarrow 2)\),
    \item la \textbf{marge de recompression disponible} pour l’éjecteur \((3\rightarrow 4\rightarrow 5)\).
\end{itemize}

Dans un cycle à éjecteur, la performance globale (COP) dépend fortement de la capacité du condenseur à maintenir un \(P_{\mathrm{cond}}\) compatible avec le régime d'aspiration/recompression. La littérature montre que l'éjecteur présente une pression critique de refoulement : au-delà, le rapport d'entraînement chute brutalement \cite{Eames1995,Huang1999}. En conséquence, le dimensionnement thermique et hydraulique du condenseur n'est pas une simple question de rejet de chaleur : c'est une condition de faisabilité opérationnelle.

\section{Typologie des condenseurs et choix technologique}
\label{sec:cond:types}

\subsection{Familles technologiques (classification)}
\label{sec:cond:types_family}

Les condenseurs peuvent être classés selon le \textbf{fluide de refroidissement} et le \textbf{mode d'échange} \cite{ASHRAEHandbook2018} :

\begin{itemize}
    \item \textbf{Condenseurs à air} : convection naturelle (sans ventilateur) ou convection forcée (avec ventilateur).
    \item \textbf{Condenseurs à eau} : échangeur tubulaire/à plaques, souvent couplé à une tour de refroidissement.
    \item \textbf{Condenseurs évaporatifs} : refroidissement par évaporation d’eau sur batterie (forte efficacité, maintenance).
\end{itemize}

Selon la \textbf{géométrie interne} (côté fluide frigorigène), on trouve :
\begin{itemize}
    \item \textbf{tube lisse} (film externe ou interne selon configuration),
    \item \textbf{tube aileté} (augmentation de surface côté air),
    \item \textbf{micro-canaux} (fortes performances en convection forcée, plus complexes),
    \item \textbf{échangeurs à plaques} (côté eau, très compacts).
\end{itemize}

\subsection{Spécificités du R718 et implications sur le choix}
\label{sec:cond:r718_specific}

Avec le R718, la pression de condensation reste \textbf{faible en absolu} (quelques kPa à \(30\text{--}40^\circ\mathrm{C}\)), ce qui impose :
\begin{itemize}
    \item une \textbf{gestion rigoureuse des non-condensables} (infiltration d’air sous vide) qui dégradent fortement le transfert et augmentent \(P_{\mathrm{cond}}\),
    \item une \textbf{attention à l’étanchéité} et aux procédures de purge,
    \item une \textbf{vigilance sur les pertes de charge} (même faibles en valeur absolue, elles sont significatives en ratio).
\end{itemize}
Les systèmes de réfrigération à éjecteur utilisant l'eau rapportent un impact très marqué des non-condensables sur les performances \cite{Sokolov1990}.

Dans ce mémoire, le choix d’un condenseur \textbf{à air en convection naturelle} répond à une logique de simplicité technologique et de faible énergie auxiliaire, au prix d’une surface d’échange potentiellement élevée et d’une sensibilité climatique plus forte.

\section{Phénoménologie de la condensation : zones thermiques et mécanismes}
\label{sec:cond:phenomenology}

\subsection{Trois zones classiques : désurchauffe, condensation, sous-refroidissement}
\label{sec:cond:zones}

Dans un condenseur réel, l'écoulement côté frigorigène se décompose généralement en trois zones \cite{Thome2004} :
\begin{enumerate}
    \item \textbf{Désurchauffe} : si l’état 5 est surchauffé, le fluide doit d’abord revenir à \(T_{\mathrm{sat}}(P_{\mathrm{cond}})\).
    \item \textbf{Condensation} : transfert latent dominant, fraction vapeur décroissante \(x \to 0\).
    \item \textbf{Sous-refroidissement} (optionnel) : refroidissement du liquide en dessous de \(T_{\mathrm{sat}}\) pour stabiliser l’alimentation du détendeur.
\end{enumerate}

Dans ton simulateur, l’état 6 correspond à un liquide saturé (\(x_6=0\)). Le sous-refroidissement peut être ignoré au premier niveau (outil), mais sera utile plus tard en dimensionnement réel.

\subsection{Modes de condensation : film vs gouttes}
\label{sec:cond:modes}

Deux régimes idéaux de condensation existent \cite{Incroptera1996} :
\begin{itemize}
    \item \textbf{Condensation en film} (filmwise) : un film liquide continu couvre la surface ; c’est le cas le plus fréquent, mais avec une résistance thermique plus élevée.
    \item \textbf{Condensation en gouttes} (dropwise) : gouttes individuelles, transfert très élevé mais difficile à maintenir (condition de mouillage/surface).
\end{itemize}

La plupart des modèles 1D adoptent la condensation en film comme hypothèse de base.

\section{Condensation en film}
\label{sec:cond:nusselt}

\subsection{Hypothèses et cadre}
\label{sec:cond:nusselt_hyp}

La théorie de Nusselt fournit une solution analytique pour la condensation laminaire en film sur surface verticale, sous hypothèses : film mince, régime laminaire, gradient thermique unidimensionnel, interface à \(T_{\mathrm{sat}}\), cisaillement vapeur négligeable \cite{Incroptera1996,Moran2014}.

\subsection{Résultat : coefficient moyen de transfert par film laminaire}
\label{sec:cond:nusselt_result}

Le résultat classique (plaque verticale) pour le coefficient moyen s’écrit sous forme :
\begin{equation}
    h_{\mathrm{cond}} = 0.943\left[
        \frac{\rho_\ell(\rho_\ell-\rho_v)g\,h_{fg}\,k_\ell^3}
        {\mu_\ell\,L\,(T_{\mathrm{sat}}-T_w)}
        \right]^{1/4},
    \label{eq:nusselt_vertical}
\end{equation}
où \(h_{fg}\) est la chaleur latente, \(k_\ell\) la conductivité du liquide, \(\mu_\ell\) la viscosité, \(L\) la longueur caractéristique et \(T_w\) la température de paroi.

Cette expression montre que le transfert en condensation filmique est relativement peu sensible à \(\Delta T\) (exposant \(1/4\)), mais dépend fortement des propriétés du liquide.

\subsection{Limites pratiques et corrections}
\label{sec:cond:nusselt_limits}

En pratique, plusieurs effets limitent l'application directe de l'équation~\ref{eq:nusselt_vertical} :
\begin{itemize}
    \item cisaillement vapeur (écoulement interne),
    \item turbulences du film (Reynolds du film élevé),
    \item géométries tubulaires/ailetées,
    \item présence de non-condensables.
\end{itemize}
Dans un modèle système, on regroupe souvent ces effets dans un coefficient global \(K\) calibré ou une corrélation appropriée \cite{Thome2004}.

\section{Convection naturelle côté air}
\label{sec:cond:natconv}

\subsection{Rayleigh et Nusselt}
\label{sec:cond:rayleigh}

Le transfert côté air en convection naturelle est gouverné par le nombre de Rayleigh :
\begin{equation}
    Ra = Gr\cdot Pr,
    \qquad
    Gr = \frac{g\beta(T_s-T_\infty)L^3}{\nu^2},
    \label{eq:rayleigh}
\end{equation}
où \(\beta\) est le coefficient de dilatation, \(\nu\) la viscosité cinématique et \(L\) une longueur caractéristique.

Une corrélation classique (plaque verticale) donne \cite{Kalogirou2014} :
\begin{equation}
    Nu = 0.68 + \frac{0.670\,Ra^{1/4}}
    {\left[1+(0.492/Pr)^{9/16}\right]^{4/9}}.
    \label{eq:nu_vertical_plate}
\end{equation}
Puis :
\begin{equation}
    h_{\mathrm{air}} = \frac{Nu\,k_{\mathrm{air}}}{L}.
    \label{eq:hair}
\end{equation}

\subsection{Conséquence majeure : résistance dominante côté air}
\label{sec:cond:dominant_air}

Dans un condenseur à air naturel, \(h_{\mathrm{air}}\) est typiquement faible comparé à \(h_{\mathrm{cond}}\). Ainsi, la résistance thermique dominante est souvent côté air :
\[
    h_{\mathrm{air}} \ll h_{\mathrm{cond}}.
\]
Cela explique les écarts importants observés en simulation lorsque \(K\) et \(A\) sont modestes : même si la condensation est « facile », l’évacuation vers l’air limite \(\dot{Q}\).

\section{Résistances thermiques globales et modèle $KA$-LMTD}
\label{sec:cond:overall}

\subsection{Résistance équivalente et coefficient global}
\label{sec:cond:U}

Pour un échangeur, on écrit :
\begin{equation}
    \dot{Q} = K A\,\Delta T_{\mathrm{lm}},
    \label{eq:KA_lmtd}
\end{equation}
avec une résistance globale équivalente :
\begin{equation}
    \frac{1}{K} = \frac{1}{h_{\mathrm{cond}}} + \frac{e}{k_{\mathrm{paroi}}} + \frac{1}{h_{\mathrm{air}}},
    \label{eq:K_series}
\end{equation}
où \(e\) est l’épaisseur de paroi.

En convection naturelle, \(1/h_{\mathrm{air}}\) domine souvent l'équation~\ref{eq:K_series}. Ainsi, augmenter la surface \(A\) et favoriser l'ailage côté air sont les leviers principaux.

\subsection{Définition de la différence de température logarithmique moyenne}
\label{sec:cond:lmtd}

Dans un condenseur où le frigorigène est approximativement à température quasi-constante \(T_{\mathrm{sat}}\) (zone de condensation dominante), et où l’air se réchauffe de \(T_{\mathrm{air,in}}\) à \(T_{\mathrm{air,out}}\), le \(\Delta T_{\mathrm{lm}}\) est :
\begin{equation}
    \Delta T_{\mathrm{lm}}
    =
    \frac{\Delta T_1 - \Delta T_2}{\ln(\Delta T_1/\Delta T_2)},
    \qquad
    \Delta T_1 = T_{\mathrm{sat}} - T_{\mathrm{air,in}},
    \quad
    \Delta T_2 = T_{\mathrm{sat}} - T_{\mathrm{air,out}}.
    \label{eq:lmtd_cond}
\end{equation}
Le cas \(\Delta T_1 \approx \Delta T_2\) doit être traité numériquement avec précaution (limite \(\Delta T_{\mathrm{lm}}\to \Delta T_1\)).

\section{Non-condensables sous vide}
\label{sec:cond:noncond}

Sous vide, l'infiltration d'air et l'accumulation de non-condensables forment une couche diffusante à l'interface vapeur/film qui augmente fortement la résistance au transfert de masse et de chaleur. Le résultat macroscopique est double \cite{Sokolov1990} :
\begin{itemize}
    \item baisse de \(\dot{Q}\) évacuable à surface donnée ;
    \item augmentation de \(T_{\mathrm{cond}}\) et donc de \(P_{\mathrm{cond}}\) pour maintenir la fermeture énergétique.
\end{itemize}
Ces effets sont critiques dans les cycles à éjecteur, car ils réduisent la marge de recompression et peuvent provoquer un décrochage. Dans un outil de simulation orienté premier dimensionnement, ces effets peuvent être représentés par une pénalisation effective de \(K\) (ou une dégradation progressive simulée).

\section{Impact sur la pression critique et la stabilité de l’éjecteur}
\label{sec:cond:critical}

Le condenseur agit comme « charge » imposée au diffuseur de l’éjecteur. Si \(K A\) est insuffisant (ou si \(T_\infty\) est trop élevé), alors \(T_{\mathrm{cond}}\) doit augmenter pour rejeter \(\dot{Q}_{\mathrm{cond}}\), ce qui augmente \(P_{\mathrm{cond}}\). Or, une hausse de \(P_{\mathrm{cond}}\) :
\begin{itemize}
    \item réduit le rapport de pression disponible pour l’éjecteur,
    \item déplace la zone de choc et augmente les irréversibilités,
    \item peut conduire au dépassement de la pression critique et donc au décrochage \cite{Huang1999,Eames1995,Anderson2016}.
\end{itemize}
Ce mécanisme justifie que, dans la phase de dimensionnement global (Partie II), le condenseur soit traité comme un composant déterminant pour la faisabilité.

\section{Modèle mathématique retenu dans le simulateur}
\label{sec:cond:model}

Dans le cadre de l’outil développé, le condenseur est modélisé par :
\begin{enumerate}
    \item un \textbf{bilan enthalpique massique} (condensation de l’état 5 vers 6 à \(P_{\mathrm{cond}}\)) :
          \begin{equation}
              \dot{Q}_{\mathrm{mass}} = \dot{m}_{\mathrm{tot}}\,(h_5 - h_6),
              \label{eq:qmass_cond}
          \end{equation}
    \item un \textbf{modèle échangeur} de type \(K A\)--LMTD :
          \begin{equation}
              \dot{Q}_{K A} = K A\,\Delta T_{\mathrm{lm}},
              \label{eq:qka_cond}
          \end{equation}
    \item un diagnostic \textbf{thermal\_mismatch} basé sur l’écart relatif :
          \begin{equation}
              \varepsilon = \left|\frac{\dot{Q}_{\mathrm{mass}}-\dot{Q}_{K A}}{\dot{Q}_{\mathrm{mass}}}\right|.
              \label{eq:mismatch}
          \end{equation}
\end{enumerate}

Dans la phase actuelle, l’objectif est de développer l’outil et d’obtenir un couplage cohérent ; le dimensionnement (choix réaliste de \(K\) et \(A\), ailetage, géométrie, etc.) est reporté à la phase de dimensionnement global.

\section{Discussion critique}
\label{sec:cond:discussion}

Un condenseur à convection naturelle présente des avantages (simplicité, absence de consommation électrique auxiliaire), mais impose :
\begin{itemize}
    \item une surface d’échange élevée,
    \item une sensibilité forte à \(T_\infty\) (climat tropical),
    \item une vulnérabilité aux non-condensables sous vide.
\end{itemize}

Dans l’outil développé, l’activation du diagnostic de mismatch permet d’identifier immédiatement un sous-dimensionnement thermique du condenseur (écart important entre \(\dot{Q}_{\mathrm{mass}}\) et \(\dot{Q}_{K A}\)). Cette approche est cohérente avec un simulateur destiné d’abord à la compréhension et à l’analyse de sensibilité.

\section{Résumé du chapitre}
\label{sec:cond:resume}

Le condenseur ferme thermiquement le cycle \((5\rightarrow 6)\) et pilote la pression haute \(P_{\mathrm{cond}}\), paramètre critique pour la stabilité de l’éjecteur. Une revue technologique a montré les principales familles de condenseurs et les contraintes spécifiques liées au R718 sous vide (non-condensables, sensibilité aux pertes de charge). Sur le plan phénoménologique, la condensation se structure en zones (désurchauffe, condensation, sous-refroidissement) et le transfert est souvent limité côté air en convection naturelle. Le modèle retenu combine un bilan massique \(\dot{Q}_{\mathrm{mass}}\) et un modèle échangeur \(K A\)--LMTD \(\dot{Q}_{K A}\), complété par un indicateur de mismatch destiné à la phase ultérieure de dimensionnement global.

% Références attendues (BibTeX) :
% ASHRAEHandbookRefrigeration
% MoranShapiro2014
% Kalogirou2014
% Thome2004
% IncroperaDeWitt
% Huang1999
% ChunnanondAphornratana2004
% Eames1995
% Anderson2016
% SokolovHershgal1990

% ==========================================================
% Fichier : chapters/part1/ch6_evaporateur.tex
% Chapitre I.6 — L’évaporateur à film ruisselant
% Norme citations : APA via \citep{}
% Références (clés BibTeX attendues) :
%   - MoranShapiro2014
%   - Bell2014CoolProp
% ==========================================================

\chapter[L'évaporateur à film ruisselant]{L'évaporateur à film ruisselant}
\label{chap:evaporateur}

L’évaporateur est l’organe où se produit l’effet frigorifique utile \(\dot{Q}_{\mathrm{evap}}=12~\mathrm{kW}\), par vaporisation du fluide R718 à basse pression. Dans la machine étudiée, l’évaporateur est de type \emph{film ruisselant} : le liquide détendu (état 2) est distribué en mince film sur une surface d’échange, tandis que la vaporisation se fait à faible pression, typiquement de l’ordre du \(\mathrm{kPa}\) pour une température d’évaporation proche de \(10^\circ\mathrm{C}\).

Ce mode d'évaporation est particulièrement adapté au R718 car il permet (i) de limiter l'inventaire liquide, (ii) de favoriser les échanges par réduction de la résistance thermique côté liquide, et (iii) d'améliorer la stabilité thermique si la distribution du film est maîtrisée. Néanmoins, sous vide profond, l'évaporateur devient un composant critique : les grands volumes spécifiques de la vapeur et la sensibilité aux pertes de charge imposent une analyse phénoménologique et un modèle thermique rigoureux \cite{Moran2014}.
\newpage
\section{Rôle systémique}
\label{sec:evap:role}

L’évaporateur assure :
\begin{itemize}
    \item la production de froid par absorption de chaleur à basse température ;
    \item la génération de vapeur secondaire aspirée par l’éjecteur (état 3) ;
    \item la stabilisation du débit secondaire \(\dot{m}_{\mathrm{sec}}\) qui conditionne le rapport d’entraînement \(\mu\) et le COP.
\end{itemize}

Du point de vue système, une dégradation de l’échange (mauvaise mouillabilité, assèchement local, pertes de charge élevées) se traduit par une baisse de \(\dot{Q}_{\mathrm{evap}}\), une modification de la qualité vapeur en sortie, et potentiellement un désamorçage de l’aspiration éjecteur.

\section{Étude phénoménologique du film ruisselant}
\label{sec:evap:phenomenology}

\subsection{Formation et dynamique du film}
\label{sec:evap:film_dynamics}

Dans un évaporateur à film ruisselant, le liquide est distribué en mince couche qui s’écoule sous l’effet de la gravité. Le régime d’écoulement du film (laminaire, transitionnel ou turbulent) influence directement :
\begin{itemize}
    \item l’épaisseur du film \(\delta\),
    \item le coefficient de transfert thermique côté liquide,
    \item le risque d’assèchement.
\end{itemize}

Une caractérisation classique repose sur un nombre de Reynolds de film, défini à partir du débit massique surfacique \(\Gamma\) (débit massique par unité de largeur de film) :
\begin{equation}
    Re_f = \frac{4\Gamma}{\mu_\ell},
    \label{eq:ref_film}
\end{equation}
où \(\mu_\ell\) est la viscosité dynamique du liquide. Lorsque \(Re_f\) est faible, le film est généralement laminaire et l’épaisseur augmente, ce qui tend à diminuer le transfert thermique. À mesure que \(Re_f\) augmente, l’apparition d’ondulations (\emph{wavy film}) peut au contraire améliorer l’échange par mélange interne du film.

\subsection{Vaporisation et résistances thermiques}
\label{sec:evap:resistances}

La vaporisation sur film ruisselant s’analyse comme une succession de résistances :
\begin{itemize}
    \item convection/conduction à travers le film liquide,
    \item résistance de paroi,
    \item résistance côté source froide (air/eau selon application).
\end{itemize}

Sous vide, la résistance côté vapeur peut devenir non négligeable si la vapeur est raréfiée ou si le transport est limité par les pertes de charge internes. C’est un point critique avec R718, car la vapeur à \(P_{\mathrm{evap}}\sim \mathrm{kPa}\) possède un volume spécifique élevé : des vitesses \(c\) élevées peuvent apparaître dans les conduites et les collecteurs, amplifiant les pertes.

\section{Bilan énergétique et définition de la puissance frigorifique}
\label{sec:evap:energy_balance}

La puissance frigorifique utile se traduit thermodynamiquement par :
\begin{equation}
    \dot{Q}_{\mathrm{evap}} = \dot{m}_{\mathrm{sec}}\,\left(h_3 - h_2\right),
    \label{eq:qevap_mass}
\end{equation}
où \(h_2\) est l’enthalpie en entrée d’évaporateur (après détente isoenthalpique) et \(h_3\) l’enthalpie en sortie (vapeur aspirée). Cette expression est fondamentale car elle lie directement la performance d’échange (via \(\dot{Q}_{\mathrm{evap}}\)), le débit secondaire \(\dot{m}_{\mathrm{sec}}\), et donc le fonctionnement de l’éjecteur.

\section{Modélisation thermique de l’évaporateur : approche détaillée}
\label{sec:evap:thermal_model}

\subsection{Écriture globale par coefficient global de transfert}
\label{sec:evap:KA}

À l’échelle échangeur, on écrit :
\begin{equation}
    \dot{Q}_{\mathrm{evap}} = K\,A\,\Delta T_{\mathrm{lm}},
    \label{eq:qevap_ka}
\end{equation}
où \(K\) est le coefficient global de transfert de chaleur, \(A\) la surface d’échange, et \(\Delta T_{\mathrm{lm}}\) la différence de température logarithmique moyenne.

Le coefficient global se décompose en résistances en série :
\begin{equation}
    \frac{1}{K} = \frac{1}{h_{\mathrm{film}}} + \frac{e}{k_{\mathrm{paroi}}} + \frac{1}{h_{\mathrm{source}}},
    \label{eq:k_series_evap}
\end{equation}
avec \(h_{\mathrm{film}}\) le coefficient côté film ruisselant, \(h_{\mathrm{source}}\) côté fluide à refroidir (ou côté air/eau), \(e\) l’épaisseur de paroi et \(k_{\mathrm{paroi}}\) sa conductivité.

\subsection{Modélisation du coefficient côté film ruisselant}
\label{sec:evap:hfilm}

Dans une démarche progressive, il est pertinent de présenter deux niveaux.

\subsubsection{Niveau théorique : film laminaire (base analytique)}
\label{sec:evap:hfilm_theory}

On relie l’épaisseur \(\delta\) à \(\Gamma\) via la dynamique du film, puis on met en évidence l’ordre de grandeur :
\begin{equation}
    h_{\mathrm{film}} \sim \frac{k_\ell}{\delta},
    \label{eq:hfilm_scale}
\end{equation}
où \(k_\ell\) est la conductivité thermique du liquide. Cette relation met en évidence le mécanisme clé : réduire \(\delta\) augmente \(h_{\mathrm{film}}\).

\subsubsection{Niveau semi-empirique : corrélations film ruisselant}
\label{sec:evap:hfilm_corr}

La littérature propose des corrélations reliant \(Nu\) à \(Re_f\), \(Pr\) et, lorsque l’ébullition intervient, à des nombres additionnels (p.\,ex. \(Bo\)). Dans ce mémoire, l’objectif n’est pas d’imposer une corrélation unique dès ce chapitre, mais de :
\begin{itemize}
    \item présenter la structure générale,
    \item justifier le choix final au chapitre d’implémentation et de simulation,
    \item montrer que les propriétés nécessaires (conductivité, viscosité, densité, chaleur latente, etc.) sont obtenues via CoolProp \cite{Bell2014}.
\end{itemize}

\section{Phénomènes critiques et stabilité sous vide}
\label{sec:evap:stability}

\subsection{Assèchement local (dry-out) et mouillage}
\label{sec:evap:dryout}

Un défaut de distribution du liquide peut provoquer un assèchement local, augmentant brutalement la température de paroi, réduisant \(\dot{Q}_{\mathrm{evap}}\) et modifiant la qualité vapeur en sortie. Sous vide, ce risque est accentué par la sensibilité du système aux fluctuations de débit.

\subsection{Pertes de charge et effet sur l’aspiration éjecteur}
\label{sec:evap:dp}

À faible pression, une faible perte absolue peut représenter une fraction importante de \(P_{\mathrm{evap}}\). Or l’aspiration de l’éjecteur dépend fortement de la pression au piquage d’aspiration. Le dimensionnement des collecteurs et conduites doit donc minimiser les pertes, notamment lorsque la vitesse vapeur \(c\) devient élevée.

\section{Modèle mathématique retenu pour la simulation}
\label{sec:evap:implemented_model}

Pour l’implémentation Python (Partie II), l’évaporateur sera modélisé par :
\begin{itemize}
    \item un bilan énergétique \(\dot{Q}_{\mathrm{evap}}=\dot{m}_{\mathrm{sec}}(h_3-h_2)\) ;
    \item une loi d’échange \(\dot{Q}_{\mathrm{evap}}=K A \Delta T_{\mathrm{lm}}\) ;
    \item un calcul des propriétés via CoolProp (enthalpies, viscosités, conductivités, etc.) \cite{Bell2014}.
\end{itemize}

Cette double écriture (thermodynamique + transfert) permet de dimensionner \(A\) à partir d’une charge de \(12~\mathrm{kW}\) et de vérifier la cohérence énergétique du cycle.

\section{Discussion critique}
\label{sec:evap:discussion}

L’évaporateur à film ruisselant offre une voie robuste pour évaporer le R718 sous vide, mais il impose une maîtrise de :
\begin{itemize}
    \item la distribution du film,
    \item le régime d’écoulement (via \(Re_f\)),
    \item les pertes de charge côté vapeur,
    \item la stabilité thermo-hydraulique.
\end{itemize}

Ces contraintes justifient, dans la partie résultats, une analyse de sensibilité de \(\dot{Q}_{\mathrm{evap}}\) et du COP aux variations de \(T_{\mathrm{cond}}\) et aux pertes de charge.

\section{Résumé du chapitre}
\label{sec:evap:resume}

L’évaporateur à film ruisselant est un composant déterminant pour la puissance frigorifique et la stabilité du cycle. Sa modélisation articule transfert thermique (via \(K\)) et thermodynamique (via \(\dot{m}_{\mathrm{sec}}(h_3-h_2)\)), tout en tenant compte des spécificités du R718 sous vide. La suite du mémoire abordera le générateur (chaudière), dont le niveau thermique fixé (\(T_{\mathrm{gen}}\)) pilote le débit motrice et donc la capacité de recompression de l’éjecteur.

% --- Références APA (commentaire pour insertion BibTeX) ---
% Bell, I. H., Wronski, J., Quoilin, S., & Lemort, V. (2014).
% CoolProp: An open-source reference-quality thermophysical property library.
% Industrial & Engineering Chemistry Research, 53(6), 2498–2508.
%
% Moran, M. J., & Shapiro, H. N. (2014).
% Fundamentals of Engineering Thermodynamics (8th ed.). Wiley.

% ==========================================================
% Fichier : chapters/part1/ch7_generateur.tex
% Chapitre I.7 — La chaudière solaire (générateur)
% Norme citations : APA via \citep{}
% Références (clés BibTeX attendues) :
%   - Kalogirou2014
%   - ChunnanondAphornratana2004   (si tu veux appuyer les valeurs typiques de COP)
% ==========================================================

\chapter[La chaudière solaire]{La chaudière solaire (générateur)}
\label{chap:generateur}


Dans une machine frigorifique à éjecteur alimentée par énergie solaire thermique, le générateur constitue la source motrice du cycle. Il assure la production de la vapeur primaire (état 8) à la température nominale \(T_{\mathrm{gen}}=100^\circ\mathrm{C}\), nécessaire au fonctionnement de la tuyère primaire de l’éjecteur (trajet \(8\rightarrow 4\)).

Le choix d'un concentrateur cylindro-parabolique s'inscrit dans une stratégie de valorisation d'un rayonnement solaire direct élevé, typique des régions intertropicales \cite{Kalogirou2014}. Ce type de concentrateur permet d'atteindre des températures comprises entre \(80^\circ\mathrm{C}\) et \(250^\circ\mathrm{C}\), compatibles avec les cycles à éjection.

Contrairement aux architectures à boucle intermédiaire (huile thermique), la présente configuration adopte un chauffage direct du R718. Cette option simplifie l’architecture globale, mais impose une maîtrise fine : (i) de la stabilité thermique, (ii) des phénomènes de vaporisation interne, et (iii) du contrôle de pression.
\newpage
\section{Principe du concentrateur cylindro-parabolique}
\label{sec:gen:ptc}

Un concentrateur cylindro-parabolique focalise le rayonnement solaire direct sur un tube absorbeur placé sur la ligne focale. La puissance solaire interceptée s’écrit :
\begin{equation}
    \dot{Q}_{\mathrm{sol,incident}} = G_b\,A_{\mathrm{aperture}},
    \label{eq:qsol_incident}
\end{equation}
où \(G_b\) est l’irradiance solaire directe normale \((\mathrm{W/m^2})\) et \(A_{\mathrm{aperture}}\) la surface d’ouverture du collecteur.

La puissance réellement absorbée est réduite par les pertes optiques :
\begin{equation}
    \dot{Q}_{\mathrm{opt}} = G_b\,A_{\mathrm{aperture}}\,\eta_{\mathrm{opt}},
    \label{eq:qopt}
\end{equation}
avec le rendement optique :
\begin{equation}
    \eta_{\mathrm{opt}} = \rho\,\tau\,\alpha\,\cos\theta,
    \label{eq:eta_opt}
\end{equation}
où \(\rho\) est la réflectivité du miroir, \(\tau\) une transmissivité éventuelle, \(\alpha\) l’absorptivité du tube, et \(\theta\) l’angle d’incidence.

\section{Bilan énergétique du générateur}
\label{sec:gen:balance}

La chaudière transforme l’énergie solaire absorbée en chaleur utile pour vaporiser le R718 :
\begin{equation}
    \dot{Q}_{\mathrm{gen}} = \dot{m}_{\mathrm{pri}}\left(h_8-h_7\right),
    \label{eq:qgen_mass}
\end{equation}
où \(\dot{m}_{\mathrm{pri}}\) est le débit primaire (branche haute pression \(1\rightarrow 7\rightarrow 8\)), \(h_7\) l’enthalpie en sortie de pompe (état 7) et \(h_8\) l’enthalpie en sortie de générateur (état 8).

Cette puissance est fournie par :
\begin{equation}
    \dot{Q}_{\mathrm{gen}}=\dot{Q}_{\mathrm{opt}}-\dot{Q}_{\mathrm{pertes}},
    \label{eq:qgen_opt_losses}
\end{equation}
où \(\dot{Q}_{\mathrm{pertes}}\) regroupe les pertes thermiques par convection naturelle, rayonnement et conduction.

\section{Pertes thermiques}
\label{sec:gen:losses}

\subsection{Convection naturelle}
\label{sec:gen:conv}

Le flux convectif externe s’écrit :
\begin{equation}
    \dot{Q}_{\mathrm{conv}}=h_{\mathrm{ext}}\,A_{\mathrm{tube}}\left(T_{\mathrm{tube}}-T_{\mathrm{amb}}\right),
    \label{eq:qconv}
\end{equation}
où \(h_{\mathrm{ext}}\) est le coefficient convectif externe, \(A_{\mathrm{tube}}\) la surface externe du tube absorbeur, \(T_{\mathrm{tube}}\) la température de surface, et \(T_{\mathrm{amb}}\) la température ambiante.

Le coefficient \(h_{\mathrm{ext}}\) dépend classiquement du nombre de Rayleigh :
\begin{equation}
    Ra=\frac{g\,\beta\left(T_{\mathrm{tube}}-T_{\mathrm{amb}}\right)L^3}{\nu^2}\,Pr,
    \label{eq:rayleigh_gen}
\end{equation}
où \(g\) est l’accélération gravitationnelle, \(\beta\) le coefficient de dilatation thermique, \(L\) une longueur caractéristique, \(\nu\) la viscosité cinématique et \(Pr\) le nombre de Prandtl.

\subsection{Rayonnement thermique}
\label{sec:gen:rad}

Le flux radiatif s’écrit :
\begin{equation}
    \dot{Q}_{\mathrm{rad}}=\varepsilon\,\sigma\,A_{\mathrm{tube}}\left(T_{\mathrm{tube}}^4-T_{\mathrm{amb}}^4\right),
    \label{eq:qrad}
\end{equation}
où \(\sigma\) est la constante de Stefan--Boltzmann et \(\varepsilon\) l’émissivité du tube.

\section{Rendement thermique du générateur}
\label{sec:gen:eta_th}

Le rendement thermique global du sous-système solaire est défini par :
\begin{equation}
    \eta_{\mathrm{th}}=\frac{\dot{Q}_{\mathrm{gen}}}{G_b\,A_{\mathrm{aperture}}}.
    \label{eq:eta_th_def}
\end{equation}

En pratique, on peut écrire :
\begin{equation}
    \eta_{\mathrm{th}}=\eta_{\mathrm{opt}}-\frac{\dot{Q}_{\mathrm{pertes}}}{G_b\,A_{\mathrm{aperture}}},
    \label{eq:eta_th_practical}
\end{equation}
ce qui montre que \(\eta_{\mathrm{th}}\) diminue lorsque \(T_{\mathrm{gen}}\) augmente, lorsque \(T_{\mathrm{amb}}\) est élevée, et lorsque la convection naturelle est peu efficace.

\section{Couplage thermodynamique avec l’éjecteur}
\label{sec:gen:coupling}

Le générateur détermine le débit primaire par :
\begin{equation}
    \dot{m}_{\mathrm{pri}}=\frac{\dot{Q}_{\mathrm{gen}}}{h_8-h_7}.
    \label{eq:mdot_pri_from_qgen}
\end{equation}

Or, à l’échelle du cycle, le COP énergétique s’écrit :
\begin{equation}
    COP=\frac{\dot{m}_{\mathrm{sec}}\left(h_3-h_2\right)}{\dot{m}_{\mathrm{pri}}\left(h_8-h_7\right)}.
    \label{eq:cop_cycle}
\end{equation}

Ainsi, une augmentation de \(T_{\mathrm{gen}}\) tend à augmenter la pression motrice et peut améliorer le rapport d’entraînement \(\mu\). Cependant, elle augmente aussi les pertes thermiques du collecteur. Il existe donc un compromis thermodynamique entre puissance motrice disponible, rendement solaire et stabilité.

\section{Dimensionnement préliminaire}
\label{sec:gen:pre_dim}

Si la machine doit produire \(\dot{Q}_{\mathrm{evap}}=12~\mathrm{kW}\) et si l’on suppose un ordre de grandeur \(COP\simeq 0.35\) (valeur typique rapportée pour des cycles à éjection vapeur, selon la configuration et les conditions), alors :
\begin{equation}
    \dot{Q}_{\mathrm{gen}}\approx \frac{\dot{Q}_{\mathrm{evap}}}{COP}\approx \frac{12}{0.35}\approx 34~\mathrm{kW}.
    \label{eq:qgen_estimate}
\end{equation}

Si \(G_b\simeq 800~\mathrm{W/m^2}\) et \(\eta_{\mathrm{th}}\simeq 0.5\), la surface d’ouverture nécessaire est :
\begin{equation}
    A_{\mathrm{aperture}}\approx \frac{\dot{Q}_{\mathrm{gen}}}{G_b\,\eta_{\mathrm{th}}}
    \approx \frac{34\,000}{800\times 0.5}\approx 85~\mathrm{m^2}.
    \label{eq:aperture_area_estimate}
\end{equation}
Ce résultat sera affiné lors du dimensionnement global (Partie II) à partir des performances simulées et des pertes effectives.

\section{Problèmes spécifiques au chauffage direct du R718}
\label{sec:gen:direct_heating}

Le chauffage direct implique :
\begin{itemize}
    \item une vaporisation interne progressive,
    \item une possible instabilité de bouillonnement,
    \item des variations locales de pression.
\end{itemize}

Le contrôle du débit primaire et la gestion des conditions de saturation sont essentiels pour éviter : surpression, vaporisation brutale, et fluctuations de débit vers l’éjecteur.

\section{Modèle mathématique retenu}
\label{sec:gen:model}

Le modèle utilisé en simulation comportera :
\begin{itemize}
    \item un bilan énergétique solaire (\(\dot{Q}_{\mathrm{sol,incident}}, \dot{Q}_{\mathrm{opt}}, \dot{Q}_{\mathrm{pertes}}\)),
    \item un calcul du rendement optique (\(\eta_{\mathrm{opt}}\)),
    \item un calcul des pertes convectives et radiatives (\(\dot{Q}_{\mathrm{conv}}, \dot{Q}_{\mathrm{rad}}\)),
    \item le calcul du débit primaire \(\dot{m}_{\mathrm{pri}}\) et le couplage avec l’éjecteur.
\end{itemize}
Les propriétés thermodynamiques nécessaires à l’évaluation des enthalpies \(h_7\) et \(h_8\) seront fournies via CoolProp au niveau de l’implémentation.

\section{Résumé du chapitre}
\label{sec:gen:resume}

La chaudière solaire constitue la source énergétique du cycle et détermine le débit primaire ainsi que la pression motrice disponibles pour l’éjecteur. Son rendement thermique conditionne directement la puissance solaire requise et, indirectement, le COP global. Le dimensionnement du générateur introduit un compromis entre température motrice élevée, pertes thermiques accrues et stabilité du système.

% --- Références APA (commentaire pour insertion BibTeX) ---
% Kalogirou, S. A. (2014). Solar Energy Engineering. Academic Press.
%
% (Optionnel si tu le cites pour COP typiques)
% Chunnanond, K., & Aphornratana, S. (2004). Ejectors: Applications in refrigeration technology.
% Renewable and Sustainable Energy Reviews, 8(2), 129–155.

% ==========================================================
% Fichier : chapters/part1/ch8_pompe.tex
% Chapitre I.8 — Pompe alimentaire de la chaudière
% Convention : \section (pas \s), citations \citep{...}
%
% CONVENTION DE NUMÉROTATION (VALIDÉE) POUR LE CYCLE :
% 1->2 : Détendeur
% 2->3 : Evaporateur
% 3->4 : Chambre de mélange (éjecteur)
% 4->5 : Diffuseur (éjecteur)
% 5->6 : Condenseur
% 1->7 : Pompe
% 7->8 : Chaudière (générateur)
% 8->4 : Tuyère (éjecteur)
%
% IMPORTANT : Dans le texte, la pompe est bien le trajet (1 -> 7).
% L'état 1 est le liquide en sortie condenseur (liquide saturé ou sous-refroidi),
% et l'état 7 est le liquide comprimé à la pression du générateur.
% L'entrée chaudière est 7, la sortie chaudière (vapeur motrice) est 8.
%
% Références BibTeX attendues (exemples de clés) :
%  - MoranShapiro2014
%  - Bell2014CoolProp
%  - Karassik2001 (Pump Handbook) ou Gulich2010 (Centrifugal Pumps)
%  - ISO9906 (si tu ajoutes norme essais pompes)
% ==========================================================

\chapter[La pompe alimentaire]{La pompe alimentaire de la chaudière}
\label{chap:pompe}

Dans la machine frigorifique à éjecteur, la pompe assure l’élévation de pression du condensat issu du condenseur (état \(1\)) afin d’alimenter la chaudière solaire (état \(7\)) selon la convention \(1\rightarrow 7\). Dans un cycle à éjection, la puissance mécanique consommée par la pompe est généralement faible devant la puissance thermique fournie au générateur, ce qui conduit souvent à négliger \(\dot{W}_{\mathrm{p}}\) dans une première analyse énergétique. Toutefois, dans le cas du R718 fonctionnant sous vide profond, la question déterminante n’est pas uniquement la puissance mécanique, mais surtout la capacité de la pompe à fonctionner \emph{sans cavitation}.

En effet, la pression absolue disponible à l'aspiration peut être fortement contrainte par (i) la faible pression de condensation (liée à \(T_{\mathrm{cond}}\)), (ii) les pertes de charge dans la ligne d'aspiration, (iii) la présence de gaz non condensables, et (iv) la proximité de la pression de saturation du liquide. Dans ces conditions, une analyse rigoureuse du NPSH (\emph{Net Positive Suction Head}) est indispensable pour garantir la stabilité hydraulique du système \cite{Moran2014}.

\newpage
\section{Rôle systémique}
\label{sec:pump:role}

La pompe contribue à :
\begin{itemize}
    \item maintenir un débit primaire \(\dot{m}_{\mathrm{pri}}\) stable, nécessaire à la production de vapeur motrice dans la chaudière ;
    \item assurer l’élévation de pression de \(P_{\mathrm{cond}}\) (état \(1\)) à \(P_{\mathrm{gen}}\) (état \(7\)) ;
    \item stabiliser l’alimentation du générateur en réduisant les fluctuations de débit susceptibles de perturber la vaporisation ;
    \item permettre un fonctionnement continu sans cavitation, condition de fiabilité et de sécurité.
\end{itemize}

Toute variation de \(\dot{m}_{\mathrm{pri}}\) affecte directement la puissance disponible au générateur, la pression motrice à l’entrée de la tuyère (\(8\rightarrow 4\)) et, par conséquent, la performance de l’éjecteur et la stabilité globale du cycle.

\section{Étude phénoménologique}
\label{sec:pump:pheno}

\subsection{Travail de pompage : formulation thermodynamique}
\label{sec:pump:work}

Pour un liquide faiblement compressible, le travail spécifique de pompage peut être approximé par :
\begin{equation}
    w_{\mathrm{p}} \approx v_{\ell}\left(P_{\mathrm{gen}}-P_{\mathrm{cond}}\right),
    \label{eq:pump_wp_vdp}
\end{equation}
où \(v_{\ell}\) est le volume spécifique du liquide. Dans une formulation basée sur les états, la puissance de pompage s’écrit :
\begin{equation}
    \dot{W}_{\mathrm{p}}=\dot{m}_{\mathrm{pri}}\left(h_{7}-h_{1}\right).
    \label{eq:pump_Wdot}
\end{equation}

En introduisant le rendement isentropique de la pompe \(\eta_{\mathrm{p}}\), défini par :
\begin{equation}
    \eta_{\mathrm{p}}=\frac{h_{7s}-h_{1}}{h_{7}-h_{1}},
    \label{eq:pump_eta_def}
\end{equation}
on obtient :
\begin{equation}
    h_{7}=h_{1}+\frac{h_{7s}-h_{1}}{\eta_{\mathrm{p}}},
    \label{eq:pump_h7}
\end{equation}
où \(7s\) désigne l’état de sortie \emph{isentropique} à la même pression que l’état \(7\).
Dans un pré-dimensionnement, \(\eta_{\mathrm{p}}\) est typiquement pris entre \(0.6\) et \(0.8\) pour une petite pompe, puis ajusté selon la technologie retenue et les conditions d’exploitation.

\subsection{Pertes de charge et pression disponible à l’aspiration}
\label{sec:pump:losses}

L’analyse de la ligne d’aspiration peut s’appuyer sur une forme de l’équation de Bernoulli, en distinguant les pertes régulières et singulières. Les pertes régulières s’écrivent :
\begin{equation}
    h_f=\lambda\frac{L}{D}\frac{c^2}{2g},
    \label{eq:headloss_fric}
\end{equation}
et les pertes singulières :
\begin{equation}
    h_s=\sum\zeta\frac{c^2}{2g},
    \label{eq:headloss_sing}
\end{equation}
où \(\lambda\) est le facteur de frottement, \(L\) la longueur, \(D\) le diamètre hydraulique, \(c\) la vitesse moyenne, et \(\zeta\) les coefficients de pertes singulières.

Sous faible pression absolue, la contrainte de conception majeure est la réduction drastique des pertes en aspiration : un faible \(\Delta P\) peut représenter une fraction significative de \(P_{\mathrm{cond}}\), augmentant fortement le risque de cavitation.

\section{Cavitation et NPSH }
\label{sec:pump:npsh}

\subsection{Définitions}
\label{sec:pump:npsh_def}

La cavitation apparaît lorsque la pression locale dans la pompe (notamment à l’œil de roue) chute au voisinage ou en dessous de la pression de saturation du liquide. Le critère usuel en ingénierie est le NPSH (\emph{hauteur nette positive d’aspiration}). On distingue :
\begin{itemize}
    \item \(NPSH_{\mathrm{a}}\) : NPSH disponible (\emph{available}), déterminé par l’installation ;
    \item \(NPSH_{\mathrm{r}}\) : NPSH requis (\emph{required}), déterminé par la courbe constructeur.
\end{itemize}
La condition de non-cavitation est :
\begin{equation}
    NPSH_{\mathrm{a}}\geq NPSH_{\mathrm{r}}.
    \label{eq:npsh_condition}
\end{equation}

\subsection{Formulation du NPSH disponible}
\label{sec:pump:npsh_a}

Dans une écriture générique (aspiration depuis un réservoir ou un collecteur liquide), le NPSH disponible peut être exprimé sous forme de hauteur :
\begin{equation}
    NPSH_{\mathrm{a}}=
    \frac{P_{\mathrm{asp}}}{\rho g}
    -\frac{P_{\mathrm{sat}}(T)}{\rho g}
    +\left(z_{\mathrm{surf}}-z_{\mathrm{pump}}\right)
    -h_{\mathrm{pertes}},
    \label{eq:npsh_a}
\end{equation}
où \(P_{\mathrm{asp}}\) est la pression absolue au point d’aspiration, \(P_{\mathrm{sat}}(T)\) la pression de saturation à la température du liquide, \(\rho\) la masse volumique, \(z\) les altitudes, et \(h_{\mathrm{pertes}}=h_f+h_s\) la somme des pertes de charge en aspiration.

\paragraph{Point critique avec le R718.}
Si le liquide en état \(1\) est proche de la saturation (faible sous-refroidissement), alors \(P_{\mathrm{asp}}\approx P_{\mathrm{sat}}(T)\), et le terme \(\big(P_{\mathrm{asp}}-P_{\mathrm{sat}}\big)/(\rho g)\) devient très faible. Le NPSH disponible se réduit alors essentiellement à la charge statique utile et aux pertes, ce qui rend la cavitation probable même pour des vitesses modestes.

\section{Effets de la cavitation sur le cycle}
\label{sec:pump:effects}

La cavitation induit typiquement :
\begin{itemize}
    \item une chute du débit primaire \(\dot{m}_{\mathrm{pri}}\) et une dégradation du rendement de la pompe ;
    \item des fluctuations de pression à l’entrée du générateur (\(P_{\mathrm{gen}}\)), pouvant provoquer des instabilités de vaporisation ;
    \item une perturbation de la tuyère (\(8\rightarrow 4\)) via une alimentation motrice instable ;
    \item une baisse du rapport d’entraînement \(\mu\) et une diminution du COP ;
    \item un risque de décrochage du cycle si l’éjecteur entre en régime défavorable.
\end{itemize}
Dans une machine à éjecteur, ces instabilités sont particulièrement critiques, car l’éjecteur présente déjà une dynamique non linéaire vis-à-vis de \(P_{\mathrm{cond}}\) et des pertes internes.

\section{Stratégies de mitigation et choix technologiques}
\label{sec:pump:mitigation}

Pour sécuriser le fonctionnement de la pompe dans un cycle R718 sous vide, plusieurs leviers de conception sont classiquement recommandés :
\begin{itemize}
    \item \textbf{Sous-refroidissement en sortie condenseur} : augmenter la marge \(P_{\mathrm{asp}}-P_{\mathrm{sat}}(T)\) en réduisant la température du liquide ;
    \item \textbf{Réduction des pertes en aspiration} : diamètres généreux, longueurs minimisées, limitation des singularités ;
    \item \textbf{Positionnement hydraulique} : placer la pompe sous le niveau liquide (charge statique positive) lorsque possible ;
    \item \textbf{Gestion des non-condensables} : purge/dégazage afin d’éviter les poches de gaz et la dégradation du NPSH ;
    \item \textbf{Technologie de pompe adaptée} : pompe compatible faible pression absolue, éventuellement à faible \(NPSH_{\mathrm{r}}\), et matériaux compatibles avec l’eau et le vide.
\end{itemize}
Ces éléments seront repris en recommandations de conception dans la partie résultats.

\section{Modèle mathématique retenu pour la simulation}
\label{sec:pump:model}

Dans le modèle global (Partie II), la pompe est représentée par :
\begin{itemize}
    \item \textbf{Élévation de pression} : \(P_7 = P_{\mathrm{gen}}\) ;
    \item \textbf{Calcul thermodynamique} : détermination de \(h_7\) par \eqref{eq:pump_h7} en utilisant \(\eta_{\mathrm{p}}\) ;
    \item \textbf{Puissance de pompe} : \(\dot{W}_{\mathrm{p}}=\dot{m}_{\mathrm{pri}}(h_7-h_1)\) ;
    \item \textbf{Diagnostic cavitation} : calcul de \(NPSH_{\mathrm{a}}\) via \eqref{eq:npsh_a}, comparaison à un seuil (ou à \(NPSH_{\mathrm{r}}\) si disponible).
\end{itemize}

Les propriétés \(\rho\), \(P_{\mathrm{sat}}(T)\) et l'état isentropique \(7s\) sont calculés via CoolProp \cite{Bell2014}. Le diagnostic cavitation est ensuite exposé dans les indicateurs/flags du système.

\section{Résumé du chapitre}
\label{sec:pump:resume}

Ce chapitre a montré que la pompe, bien que faiblement contributrice en puissance mécanique, est un composant critique pour la stabilité d’un cycle R718 sous vide. Son rôle est d’assurer l’élévation de pression selon \(1\rightarrow 7\) et de garantir un débit primaire stable vers la chaudière (\(7\rightarrow 8\)). L’analyse NPSH met en évidence que la cavitation peut survenir facilement lorsque le liquide est faiblement sous-refroidi ou lorsque les pertes en aspiration sont élevées. Le modèle retenu intègre un calcul thermodynamique basé sur un rendement \(\eta_{\mathrm{p}}\) et un diagnostic cavitation, indispensables pour un couplage système robuste.

% --- Références APA (commentaire pour insertion BibTeX) ---
% Moran, M. J., & Shapiro, H. N. (2014). Fundamentals of Engineering Thermodynamics (8th ed.). Wiley.
% Bell, I. H., Wronski, J., Quoilin, S., & Lemort, V. (2014). CoolProp: An open-source reference-quality thermophysical property library.
% Industrial & Engineering Chemistry Research, 53(6), 2498–2508.
% (Optionnel, si tu enrichis la partie pompe avec un état de l'art pompe)
% Karassik, I. J., Messina, J. P., Cooper, P., & Heald, C. C. (2001). Pump Handbook (3rd ed.). McGraw-Hill.
% Gulich, J. F. (2010). Centrifugal Pumps (2nd ed.). Springer.

% ==========================================================
% PARTIE II — ÉTUDE SYSTÉMIQUE
% ==========================================================
\part{Étude systémique}

% Introduction de la Partie II
\chapter*{Introduction de la Partie II}
\addcontentsline{toc}{chapter}{Introduction de la Partie II}

La seconde partie traite l’étude systémique de la machine frigorifique à éjecteur au R718, en mettant l’accent sur le couplage des composants, l’implémentation logicielle et l’analyse des performances.

L’enjeu principal est de passer d’une modélisation « composant par composant » à une représentation cohérente du cycle complet, où les niveaux de pression, les débits massiques et les transferts thermiques sont interdépendants.
Le couplage est formulé selon une convention de numérotation unifiée des états thermodynamiques, utilisée à la fois dans la simulation et dans la représentation sur diagrammes \(P\text{-}h\) et \(T\text{-}s\).

Dans ce cadre, l’outil numérique développé en Python s’appuie sur une architecture modulaire orientée objet, permettant de tester chaque composant séparément puis de l’intégrer au système global.
L’éjecteur est considéré par défaut selon un modèle compressible amélioré (V2), intégrant les diagnostics de régime, la détection d’étranglement et la présence éventuelle d’une onde de choc.

L’analyse finale porte sur la performance énergétique (notamment le coefficient de performance, \(COP\)), l’influence des conditions opératoires, et les recommandations d’amélioration.
Une mini-analyse exergétique est également introduite afin d’identifier qualitativement les sources majeures d’irréversibilités, en cohérence avec la perspective d’approfondissement prévue en thèse.

\chapter{Méthodologie de modélisation}

Après avoir établi, dans la première partie, les fondements thermodynamiques, technologiques et phénoménologiques des composants du système, cette seconde partie adopte une approche systémique visant à formaliser le modèle mathématique global de la machine frigorifique à éjecteur utilisant le R718.

L’objectif de ce chapitre est de :

\begin{itemize}
    \item Définir le périmètre exact du modèle,
    \item Préciser les hypothèses globales retenues,
    \item Formaliser les variables d’entrée du dimensionnement,
    \item Établir les équations de couplage,
    \item Définir les critères de cohérence thermodynamique du simulateur.
\end{itemize}

Cette méthodologie constitue la base scientifique du développement du simulateur numérique présenté au chapitre suivant.
\newpage
\section{Périmètre du modèle}

Le modèle développé couvre l’ensemble du circuit frigorifique et du circuit moteur solaire selon la convention de numérotation adoptée :

\begin{itemize}
    \item 1 $\rightarrow$ 2 : Détendeur
    \item 2 $\rightarrow$ 3 : Évaporateur
    \item 3 $\rightarrow$ 4 : Chambre de mélange de l’éjecteur
    \item 4 $\rightarrow$ 5 : Diffuseur
    \item 5 $\rightarrow$ 6 : Condenseur
    \item 6 $\rightarrow$ 1 : Réservoir / fermeture de cycle
    \item 1 $\rightarrow$ 7 : Pompe
    \item 7 $\rightarrow$ 8 : Chaudière solaire
    \item 8 $\rightarrow$ 4 : Tuyère primaire
\end{itemize}

Le modèle est :

\begin{itemize}
    \item Stationnaire
    \item Unidimensionnel quasi-uniforme
    \item Basé sur des propriétés thermodynamiques réelles (CoolProp)
\end{itemize}

Ne sont pas inclus dans le périmètre :

\begin{itemize}
    \item Les effets transitoires,
    \item Les instabilités dynamiques fines,
    \item Les simulations CFD 2D/3D,
    \item Les effets vibratoires ou acoustiques.
\end{itemize}

Le modèle vise une représentation énergétique cohérente à l’échelle système.
\section{Couplage thermodynamique et massique}

Le fonctionnement du système repose sur un couplage fort entre les composants.

\subsection{Couplage massique}

La conservation de la masse dans l’éjecteur impose :

\begin{equation}
    \dot{m}_{mix} = \dot{m}_{pri} + \dot{m}_{sec}
\end{equation}

Le rapport d’entraînement est défini par :

\begin{equation}
    \mu = \frac{\dot{m}_{sec}}{\dot{m}_{pri}}
\end{equation}

\subsection{Couplage énergétique global}

À l’échelle du cycle complet, la conservation de l’énergie impose :

\begin{equation}
    \dot{Q}_{cond} = \dot{Q}_{gen} + \dot{Q}_{evap}
\end{equation}

Cette relation constitue une condition fondamentale de fermeture énergétique du modèle.
\section{Hypothèses globales du modèle}

Les hypothèses retenues sont les suivantes :

\begin{itemize}
    \item Écoulement stationnaire,
    \item Modèle quasi-1D pour l’éjecteur,
    \item Détente isoenthalpique : $h_1 = h_2$,
    \item Mélange adiabatique dans la chambre de mélange,
    \item Condensation en film laminaire (théorie de Nusselt),
    \item Convection naturelle côté air pour le condenseur,
    \item Chauffage direct du R718 dans le générateur,
    \item Propriétés thermodynamiques réelles obtenues via CoolProp.
\end{itemize}

Ces hypothèses constituent un compromis entre rigueur physique et tractabilité numérique.
\section{Grandeurs d'entrée du dimensionnement}

Le dimensionnement inverse du système repose sur quatre grandeurs imposées :

\begin{itemize}
    \item Puissance frigorifique : $\dot{Q}_{evap}$
    \item Température d’évaporation : $T_{evap}$
    \item Température de condensation : $T_{cond}$
    \item Température du générateur : $T_{gen}$
\end{itemize}

À partir de ces grandeurs, le simulateur détermine :

\begin{itemize}
    \item Les pressions d’équilibre $P_{evap}$, $P_{cond}$, $P_{gen}$,
    \item Les débits massiques $\dot{m}_{pri}$ et $\dot{m}_{sec}$,
    \item Les états thermodynamiques 1 à 8,
    \item Les performances globales (COP, $\mu$).
\end{itemize}
\section{Critères de cohérence thermodynamique et indicateurs de validité}

Le simulateur ne se limite pas au calcul des états thermodynamiques ; il intègre un système d’indicateurs booléens (flags) permettant de vérifier la validité physique et numérique de la solution obtenue.

Ces indicateurs sont évalués à chaque itération du solveur global.

\subsection{Hiérarchie des pressions}

Le fonctionnement correct du cycle impose :

\begin{equation}
    P_{gen} > P_{cond} > P_{evap}
\end{equation}

Toute violation de cette condition active le flag :

\begin{itemize}
    \item \texttt{pressure\_hierarchy\_error}
\end{itemize}

Ce flag indique une inversion thermodynamique non physique du cycle.
\subsection{Fermeture énergétique globale}

La conservation de l’énergie à l’échelle système impose :

\begin{equation}
    \dot{Q}_{cond} = \dot{Q}_{gen} + \dot{Q}_{evap}
\end{equation}

Le résidu énergétique est défini par :

\begin{equation}
    \Delta Q = \left| \dot{Q}_{cond} - (\dot{Q}_{gen} + \dot{Q}_{evap}) \right|
\end{equation}

Si :

\begin{equation}
    \Delta Q > \varepsilon
\end{equation}

alors le flag suivant est activé :

\begin{itemize}
    \item \texttt{energy\_mismatch}
\end{itemize}

où $\varepsilon$ représente la tolérance numérique fixée pour le solveur.
\subsection{Convergence du dimensionnement inverse}

Le dimensionnement inverse impose une convergence sur la puissance frigorifique cible :

\begin{equation}
    \left| \dot{Q}_{evap}^{calc} - \dot{Q}_{evap}^{cible} \right| < \varepsilon_{conv}
\end{equation}

En cas de non convergence après un nombre maximal d’itérations :

\begin{itemize}
    \item \texttt{convergence\_failure}
\end{itemize}
\subsection{Régime critique et pression critique}

Le fonctionnement stable de l’éjecteur impose :

\begin{equation}
    P_{cond} \leq P_{cond,crit}
\end{equation}

Si :

\begin{equation}
    P_{cond} > P_{cond,crit}
\end{equation}

le régime critique disparaît et le flag suivant est activé :

\begin{itemize}
    \item \texttt{critical\_regime\_lost}
\end{itemize}

Ce phénomène correspond au décrochage du cycle.
\subsection{Cohérence du régime supersonique}

Dans la tuyère primaire :

\begin{equation}
    M_{throat} \geq 1
\end{equation}

En cas d'absence d’étranglement attendu :

\begin{itemize}
    \item \texttt{no\_choking}
\end{itemize}

Lorsqu’un choc normal est détecté :

\begin{itemize}
    \item \texttt{normal\_shock\_detected}
\end{itemize}

Si le choc entraîne une récupération de pression insuffisante :

\begin{itemize}
    \item \texttt{poor\_pressure\_recovery}
\end{itemize}
\subsection{Production d’entropie}

Le second principe impose :

\begin{equation}
    \Delta s \geq 0
\end{equation}

Si :

\begin{equation}
    \Delta s < 0
\end{equation}

alors le flag :

\begin{itemize}
    \item \texttt{entropy\_violation}
\end{itemize}

est activé, indiquant une incohérence thermodynamique grave.
\subsection{Risque de cavitation dans la pompe}

La condition de non-cavitation est :

\begin{equation}
    NPSH_a \geq NPSH_r
\end{equation}

Si :

\begin{equation}
    NPSH_a < NPSH_r
\end{equation}

le flag suivant est activé :

\begin{itemize}
    \item \texttt{cavitation\_risk}
\end{itemize}
\subsection{Conditions physiques minimales}

Les conditions suivantes doivent être respectées :

\begin{itemize}
    \item $\mu > 0$
    \item $\dot{m}_{pri} > 0$
    \item $\dot{m}_{sec} > 0$
    \item $0 \leq x \leq 1$ pour les états diphasiques
\end{itemize}

En cas de violation :

\begin{itemize}
    \item \texttt{non\_physical\_state}
\end{itemize}
\subsection{Indicateur global de validité}

Lorsque tous les critères précédents sont satisfaits, le flag global :

\begin{itemize}
    \item \texttt{success = true}
\end{itemize}

est activé.

Dans le cas contraire, la simulation est considérée non valide.
\section*{Résumé du chapitre}

Ce chapitre a défini le cadre méthodologique de la modélisation systémique de la machine frigorifique à éjecteur R718.

Le périmètre du modèle, les hypothèses globales, les équations de couplage et les critères de cohérence ont été formalisés. Cette structuration constitue la base scientifique du développement du simulateur numérique présenté au chapitre suivant.

\chapter{Architecture et fonctionnement du simulateur}

Après avoir établi la méthodologie de modélisation et les critères de cohérence thermodynamique du système, il est nécessaire de décrire l’architecture logicielle permettant l’implémentation opérationnelle du modèle.

Le simulateur développé dans le cadre de ce travail constitue un outil numérique destiné à :

\begin{itemize}
    \item dimensionner la machine frigorifique à éjecteur R718,
    \item résoudre le couplage non linéaire entre composants,
    \item vérifier automatiquement la cohérence thermodynamique,
    \item fournir des indicateurs de performance énergétique et exergétique,
    \item permettre une interaction utilisateur via une interface graphique dédiée.
\end{itemize}

L’architecture du simulateur a été conçue selon une logique modulaire afin de :

\begin{itemize}
    \item séparer clairement les modèles physiques des composants,
    \item garantir la maintenabilité du code,
    \item faciliter les évolutions futures (ajout de modèles, raffinements),
    \item assurer la traçabilité des échanges thermodynamiques.
\end{itemize}

Le développement repose sur le langage Python, choisi pour :

\begin{itemize}
    \item sa robustesse scientifique,
    \item son écosystème thermodynamique (CoolProp),
    \item ses capacités de visualisation,
    \item son adaptabilité à une interface graphique interactive.
\end{itemize}

Ce chapitre présente successivement :

\begin{itemize}
    \item l’arborescence générale du projet,
    \item l’organisation modulaire des composants,
    \item le protocole d’échange des données thermodynamiques,
    \item le point d’entrée du programme,
    \item la stratégie de validation et de test unitaire,
    \item les outils et l’environnement de développement.
\end{itemize}

L’objectif n’est pas uniquement de décrire une implémentation informatique, mais de démontrer la cohérence entre la structure logicielle et la structure thermodynamique du système étudié.
\newpage

\section{Arborescence du projet}

L'organisation du code source est un pilier fondamental pour la maintenabilité, la lisibilité et l'évolutivité d'un projet logiciel. Le simulateur a été structuré en suivant les conventions modernes du développement Python, en séparant clairement le code source, les tests, la documentation et les scripts de déploiement. Cette organisation modulaire facilite non seulement la navigation dans le projet, mais elle établit également une base solide pour l'intégration continue et la validation unitaire.

La structure globale du projet est illustrée par l'arborescence suivante :

\begin{figure}[h]
    \centering
    \begin{forest}
        for tree={
        grow'=0,            % Croissance horizontale
        l sep=1.5em,         % Séparation entre les niveaux
        fit=band,            % Ajustement de la taille
        edge={-latex},       % Flèches sur les branches
        },
        [\texttt{App\_R718/}, dir
                    [\texttt{src/}, dir
                            [\texttt{app\_r718/}, dir
                                    [\texttt{core/}, dir, name=core]
                                    [\texttt{modules/}, dir, name=modules]
                                    [\texttt{ui/}, dir, name=ui]
                                    [\texttt{\_\_main\_\_.py}, pyfile]
                            ]
                    ]
                    [\texttt{test/}, dir
                            [\texttt{test\_component.py}, pyfile]
                    ]
                    [\texttt{reports/}, dir]
                    [\texttt{main.py}, pyfile]
                    [\texttt{build\_exe.py}, pyfile]
                    [\texttt{requirements.txt}, file]
                    [\texttt{*.md}, file]
            ]
    \end{forest}
    \vspace{1em}
    \caption{Arborescence générale du projet App\_R718}
    \label{fig:arborescence_projet}
\end{figure}

Chaque répertoire et fichier remplit un rôle spécifique dans le projet, contribuant à une architecture claire et fonctionnelle. Cette organisation permet de séparer les différentes préoccupations du projet, facilitant ainsi le développement collaboratif et la maintenance à long terme.:
\begin{itemize}
    \item \textbf{\texttt{src/}} : Ce répertoire constitue le cœur du projet. Il contient tout le code source de l'application, encapsulé dans un package principal (\texttt{app\_r718}) pour éviter les conflits de nommage et faciliter la distribution.

    \item \textbf{\texttt{test/}} : Dédié aux tests de validation. Il héberge les scripts de tests unitaires (utilisant le framework \textit{Pytest}) qui permettent de vérifier le bon fonctionnement de chaque composant de manière isolée. Cette approche garantit la robustesse et la fiabilité du code à chaque modification.

    \item \textbf{\texttt{reports/}} : Répertoire destiné à stocker les rapports générés, comme les résultats des campagnes de tests ou les analyses de performance.

    \item \textbf{\texttt{main.py}} : Point d'entrée principal de l'application. L'exécution de ce script lance l'interface graphique et initialise la simulation.

    \item \textbf{\texttt{build\_exe.py}} : Script utilitaire qui automatise le processus de compilation de l'application en un fichier exécutable autonome (\texttt{.exe}) pour Windows, en utilisant l'outil \textit{PyInstaller}.

    \item \textbf{\texttt{requirements.txt}} : Fichier essentiel qui liste toutes les bibliothèques Python externes nécessaires au projet (ex: \textit{CoolProp}, \textit{numpy}, \textit{matplotlib}). Il permet de recréer un environnement de développement identique et fonctionnel via une simple commande.

    \item \textbf{Fichiers \texttt{.md}} : Ensemble de fichiers de documentation au format Markdown qui décrivent le contexte scientifique (\texttt{context.md}), l'utilisation de l'interface (\texttt{UI\_USAGE.md}) et la procédure de build (\texttt{BUILD\_README.md}).
\end{itemize}

Cette organisation rigoureuse sépare les préoccupations (\textit{separation of concerns}), ce qui est une pratique exemplaire en génie logiciel.
% Contenu à insérer dans votre fichier LaTeX principal

\section{Architecture modulaire}

\subsection{Une approche modulaire et fiable}

Au cœur du simulateur se trouve une architecture logicielle rigoureuse, conçue pour garantir la fiabilité des calculs, la maintenabilité du code et son évolutivité future. Cette architecture repose sur deux principes fondamentaux : la modularité via le patron de conception \textbf{Modèle-Vue-Contrôleur (MVC)} et l'assurance de la \textbf{cohérence des opérations}, une philosophie inspirée des principes ACID (Atomicité, Consistance, Isolation, Durabilité) du monde des bases de données.

L'idée est de traiter chaque calcul de composant comme une transaction fiable : une opération qui est \textbf{atomique} (elle réussit complètement ou échoue sans altérer le système), qui préserve la \textbf{consistance} des lois thermodynamiques, et qui s'exécute en \textbf{isolation} des autres composants.

Pour ce faire, chaque composant physique du cycle frigorifique est encapsulé dans son propre module logiciel indépendant. Le projet est ainsi décomposé en :
\begin{itemize}
    \item Module \textbf{Pompe} (\texttt{pump/})
    \item Module \textbf{Générateur} (\texttt{generator/})
    \item Module \textbf{Éjecteur} (\texttt{ejector/})
    \item Module \textbf{Condenseur} (\texttt{condenser/})
    \item Module \textbf{Détendeur} (\texttt{expansion\_valve/})
    \item Module \textbf{Évaporateur} (\texttt{evaporator/})
\end{itemize}

Chacun de ces modules respecte scrupuleusement le pattern MVC, dont nous allons détailler l'implémentation à travers l'étude d'un cas concret.

\subsection{Étude de cas : Le module du Détendeur (\textit{Expansion Valve})}

Le module du détendeur, de par sa simplicité physique (processus isenthalpique), est un exemple parfait pour illustrer l'application de l'architecture MVC.

\subsubsection{Le Modèle (\texttt{model.py})}
Le fichier \texttt{src/app\_r718/modules/expansion\_valve/model.py} contient la classe \texttt{ExpansionValveModel}. Son unique responsabilité est d'exécuter la logique métier. Il expose une méthode principale, typiquement \texttt{simulate(state\_in, P\_out)}, qui :
\begin{enumerate}
    \item Prend en entrée un état thermodynamique initial (\texttt{state\_in}) et une pression de sortie (\texttt{P\_out}).
    \item Effectue le calcul de la détente à enthalpie constante ($h_{out} = h_{in}$). Pour cela, il s'appuie sur la bibliothèque \textit{CoolProp} pour déterminer le nouvel état thermodynamique à la pression \texttt{P\_out} et à l'enthalpie \texttt{state\_in.h}.
    \item Retourne un nouvel objet représentant l'état de sortie (\texttt{state\_out}).
\end{enumerate}
Ce modèle est totalement agnostique de l'interface ; il pourrait être utilisé dans un script, une API web ou une application de bureau sans aucune modification.

\subsubsection{La Vue (\texttt{view.py})}
La vue est double. D'une part, une vue console permet un usage en mode script pour les tests. D'autre part, la classe \texttt{ExpansionValveTkView} dans \texttt{view.py} définit l'interface graphique. Elle hérite d'une fenêtre \textit{Tkinter} et contient :
\begin{itemize}
    \item Des widgets de saisie (\texttt{Entry}) pour les paramètres d'entrée (Pression, Température).
    \item Un bouton "Simuler" qui déclenchera le calcul.
    \item Des zones de texte (\texttt{Label}) pour afficher les résultats de manière formatée.
    \item Une zone de dessin (\textit{canvas}) \textit{Matplotlib} pour tracer le diagramme P-h de la transformation.
\end{itemize}
La vue ne fait qu'afficher des informations et capturer les actions de l'utilisateur, qu'elle transmet au contrôleur.

\subsubsection{Le Contrôleur (\texttt{controller.py})}
Le fichier \texttt{controller.py} est le chef d'orchestre. La classe \texttt{Controller} est initialisée au lancement du module et effectue les actions suivantes :
\begin{enumerate}
    \item Elle instancie le Modèle et la Vue.
    \item Elle lie l'action du bouton "Simuler" de la Vue à une de ses méthodes (par exemple, \texttt{run\_simulation}).
    \item Lorsque \texttt{run\_simulation} est appelée, le contrôleur récupère les valeurs saisies dans la Vue.
    \item Il appelle la méthode \texttt{simulate()} du Modèle en lui passant ces valeurs.
    \item Une fois que le Modèle retourne l'état de sortie, le contrôleur appelle les méthodes de la Vue pour mettre à jour les champs de résultats et redessiner le graphique.
\end{enumerate}
Ce flux de contrôle unidirectionnel (Vue $\rightarrow$ Contrôleur $\rightarrow$ Modèle $\rightarrow$ Contrôleur $\rightarrow$ Vue) garantit que la logique métier reste isolée et que l'interface utilisateur est toujours une représentation fidèle de l'état du modèle.

\section{Protocole d'échange}

L'architecture modulaire présentée précédemment ne peut être efficace sans un protocole de communication standardisé et fiable entre les différents modules. Si chaque module devait gérer manuellement les propriétés thermodynamiques, le risque d'incohérence, d'erreurs d'approximation et de duplication de code serait immense. Pour pallier ce problème, un protocole d'échange robuste a été mis en place, formant le véritable moteur de calcul du simulateur. Il repose sur deux classes fondamentales du répertoire \texttt{core/} : \texttt{ThermoState} et \texttt{PropsService}.

\subsection{La classe \texttt{ThermoState} : Le "Passeport" Thermodynamique}

La classe \texttt{ThermoState} est l'objet de communication standardisé. Elle agit comme un "passeport" qui représente l'état complet du fluide de travail (R718) à un point précis du cycle. Plutôt que de manipuler une multitude de variables indépendantes (pression, température, etc.), les modules s'échangent une unique instance de \texttt{ThermoState}.

Sa structure peut être conceptualisée comme suit (voir figure \ref{fig:thermostate_class}) :

\begin{figure}[h]
    \centering
    \begin{tikzpicture}[node distance=0.5cm]
        \node (class_name) [class, fill=cyan!20, text width=15em] {\textbf{Classe : ThermoState}};
        \node (attributes) [attribute, below=of class_name] {\textbf{Attributs (Données)} \\ \small P, T, h, s, x, rho, ...};
        \node (methods) [method, below=of attributes] {\textbf{Méthodes (Comportement)} \\ \small \texttt{update(P, T)}, \texttt{update(h, s)}, ...};
    \end{tikzpicture}
    \caption{Structure de la classe ThermoState}
    \label{fig:thermostate_class}
\end{figure}

\begin{itemize}
    \item \textbf{Les Attributs} : Ils stockent les propriétés thermodynamiques fondamentales du fluide : Pression (P), Température (T), Enthalpie (h), Entropie (s), Titre de vapeur (x), Masse volumique (rho), etc.
    \item \textbf{Les Méthodes} : Elles constituent l'interface publique pour mettre à jour l'état. Par exemple, une méthode \texttt{update(P, T)} permet de recalculer toutes les autres propriétés à partir d'une pression et d'une température données.
\end{itemize}

Le rôle de \texttt{ThermoState} est donc de garantir l'intégrité et la cohérence d'un état. Dès qu'une de ses méthodes de mise à jour est appelée, l'objet devient responsable de recalculer en interne l'ensemble de ses attributs pour qu'ils soient physiquement consistants. Mais comment réalise-t-il ces calculs ? Il ne les fait pas lui-même ; il les délègue.

\subsection{La classe \texttt{PropsService} : La Source de Vérité Unique}

La classe \texttt{PropsService} est un service implémenté selon le patron de conception \textit{Singleton}, ce qui signifie qu'il n'existe qu'une seule et unique instance de ce service dans toute l'application. Son rôle est d'agir comme une passerelle centralisée et exclusive vers la bibliothèque de calcul thermodynamique \textit{CoolProp}.

Aucun autre composant de l'application n'a le droit d'appeler \textit{CoolProp} directement. Cette centralisation est cruciale car elle garantit que tous les calculs, où qu'ils soient dans l'application, sont effectués de la même manière et avec la même configuration.

\subsection{Interaction et fonctionnement du moteur}

Le génie de ce protocole réside dans l'interaction entre \texttt{ThermoState} et \texttt{PropsService}. Le flux de travail est le suivant (voir figure \ref{fig:interaction_flux}) :

\begin{figure}[h]
    \centering
    \begin{tikzpicture}[node distance=2.5cm]
        \node (module) [class] {Module X \\ (ex: Détendeur)};
        \node (state) [class, fill=cyan!20, right=of module] {Instance de \\ \texttt{ThermoState}};
        \node (service) [service, below=of state] {\texttt{PropsService} \\ (Singleton)};
        \node (coolprop) [rectangle, draw, fill=gray!30, pattern=north east lines, below=of service, text width=8em, text centered, minimum height=3em] {Bibliothèque \\ \textit{CoolProp}};

        \path [line] (module) -- node[midway, above] {1. Appelle \texttt{update()}} (state);
        \path [dashed_line] (state) -- node[midway, right] {2. Délègue le calcul} (service);
        \path [dashed_line] (service) -- node[midway, right] {3. Appelle \textit{CoolProp}} (coolprop);
        \path [dashed_line] (coolprop) -- node[midway, left] {4. Retourne la valeur} (service);
        \path [dashed_line] (service) -- node[midway, left] {5. Retourne le résultat} (state);
        \path [line] (state) -- node[midway, below] {6. État mis à jour} (module);
    \end{tikzpicture}
    \caption{Flux d'interaction entre ThermoState et PropsService}
    \label{fig:interaction_flux}
\end{figure}

\begin{enumerate}
    \item Un module (par exemple, le modèle du détendeur) a besoin de calculer un nouvel état. Il appelle une méthode de son objet \texttt{ThermoState}, par exemple \texttt{state.update(P=1227, h=420000)}.
    \item L'instance de \texttt{ThermoState} ne fait aucun calcul elle-même. Elle transmet immédiatement la demande à l'instance unique de \texttt{PropsService}.
    \item \texttt{PropsService} traduit cette demande en un appel spécifique à la bibliothèque \textit{CoolProp}.
    \item \textit{CoolProp} effectue le calcul complexe et retourne la valeur brute.
    \item Le service reçoit la réponse et la renvoie à l'objet \texttt{ThermoState}.
    \item L'objet \texttt{ThermoState} met à jour ses attributs internes. Le module dispose maintenant d'un objet d'état parfaitement cohérent, prêt à être utilisé ou transmis au module suivant.
\end{enumerate}

Ensemble, \texttt{ThermoState} et \texttt{PropsService} forment un moteur de calcul thermodynamique robuste qui garantit la \textbf{consistance}, l'\textbf{intégrité} et la \textbf{fiabilité} des données à travers toute la simulation, tout en masquant la complexité de la bibliothèque sous-jacente.

\section{Points d'entrée et Interface Principale}

Un point d'entrée est le "lieu" par lequel l'exécution d'un programme commence. Dans le cas de notre simulateur, une attention particulière a été portée à la flexibilité du lancement, tout en guidant l'utilisateur vers une interface d'accueil claire et fonctionnelle qui sert de hub central pour accéder aux différents modules de simulation.

\subsection{Les mécanismes de lancement}
L'application a été conçue pour être lancée de deux manières conventionnelles en Python, offrant ainsi une flexibilité d'usage que ce soit en phase de développement ou d'utilisation finale :
\begin{enumerate}
    \item \textbf{Lancement via le script principal (\texttt{main.py})} :
          La commande \texttt{python main.py} exécute le script situé à la racine du projet. Ce fichier contient le bloc standard \texttt{if \_\_name\_\_ == "\_\_main\_\_":}, qui est la porte d'entrée conventionnelle pour une application Python. Son rôle est simple : importer la classe de l'application principale et l'exécuter.

    \item \textbf{Lancement en tant que module (\texttt{python -m src.app\_r718})} :
          Cette méthode, plus robuste pour les projets packagés, utilise le fichier \texttt{\_\_main\_\_.py} situé à l'intérieur du package \texttt{src/app\_r718}. L'interpréteur Python exécute le contenu de ce fichier lorsque le package lui-même est appelé. Cela permet de lancer l'application sans se soucier du répertoire courant, ce qui est une pratique recommandée.
\end{enumerate}
Dans les deux cas, le code exécuté est fonctionnellement identique : il instancie et lance la fenêtre principale de l'application.

\subsection{L'interface d'accueil : La classe \texttt{MainApp}}
Quelle que soit la méthode de lancement, l'utilisateur est accueilli par la fenêtre principale, gérée par la classe \texttt{MainApp} définie dans \texttt{src/app\_r718/ui/app.py}. Cette fenêtre sert de tableau de bord central, à partir duquel l'utilisateur peut choisir quel composant du cycle frigorifique il souhaite simuler.

L'interface est volontairement épurée pour guider l'utilisateur. Elle se compose d'un titre et d'une série de boutons, chacun correspondant à un module de simulation (voir figure \ref{fig:mainapp_interface}).

\begin{figure}[h]
    \centering
    \begin{tikzpicture}[node distance=0.5cm]
        % Fenêtre principale
        \node (main_window) [window, minimum height=25em] {};

        % Titre de la fenêtre
        \node at (main_window.north) [above, yshift=-1cm, title] {Simulateur R718};

        % Boutons arrangés en une colonne
        \node (btn_detendeur) [button] at (main_window.center) [yshift=8em] {\faCogs\hspace{0.5em}Détendeur};
        \node (btn_evap) [button, below=of btn_detendeur, fill=gray!30] {\faThermometerHalf\hspace{0.5em}Évaporateur};
        \node (btn_cond) [button, below=of btn_evap, fill=gray!30] {\faTint\hspace{0.5em}Condenseur};
        \node (btn_eject) [button, below=of btn_cond, fill=gray!30] {\faRandom\hspace{0.5em}Éjecteur};
        \node (btn_gen) [button, below=of btn_eject, fill=gray!30] {\faFire\hspace{0.5em}Générateur};
        \node (btn_pump) [button, below=of btn_gen, fill=gray!30] {\faCog\hspace{0.5em}Pompe};
    \end{tikzpicture}
    \caption{Interface principale de l'application MainApp}
    \label{fig:mainapp_interface}
\end{figure}

\subsection{Fonctionnement et flux de contrôle}
Le fonctionnement de cette interface d'accueil est direct et illustre bien le rôle d'un contrôleur central :
\begin{enumerate}
    \item \textbf{Initialisation} : Au lancement, la classe \texttt{MainApp} construit la fenêtre \textit{Tkinter} et y place les différents boutons. Chaque bouton est immédiatement lié à une méthode de rappel (une "callback"). Par exemple, le bouton "Détendeur" est lié à une méthode comme \texttt{self.open\_expansion\_valve\_module}.

    \item \textbf{Action de l'utilisateur} : L'utilisateur clique sur l'un des boutons, par exemple "Détendeur".

    \item \textbf{Réponse du contrôleur} : L'événement de clic déclenche l'appel de la méthode \texttt{open\_expansion\_valve\_module}.

    \item \textbf{Instanciation du module} : Cette méthode a pour unique responsabilité d'importer et d'instancier le contrôleur spécifique au module du détendeur (c'est-à-dire \texttt{expansion\_valve.controller.Controller}).

    \item \textbf{Passage de relais} : Une fois le contrôleur du module instancié, celui-ci prend le relais. Il devient responsable de créer sa propre fenêtre de simulation (sa Vue), d'afficher ses propres widgets et de gérer ses propres calculs, comme décrit dans la section sur l'architecture modulaire.
\end{enumerate}

La fenêtre principale agit donc comme un "aiguilleur" : elle ne réalise aucun calcul thermodynamique mais se contente de déléguer la responsabilité au sous-système approprié en fonction de l'interaction de l'utilisateur.

\section{Validation par les Tests Unitaires}

Dans un projet de simulation numérique, la confiance dans les résultats produits est primordiale. Une simple erreur de code peut invalider l'ensemble d'une étude scientifique. Pour cette raison, une stratégie de validation rigoureuse basée sur des tests unitaires a été mise en place. L'objectif n'est pas de tester l'application dans son ensemble, mais de vérifier de manière chirurgicale et isolée que chaque brique de logique métier — chaque modèle physique — se comporte exactement comme attendu.

\subsection{Le choix de \texttt{pytest} et la philosophie de test}
Le framework \textbf{\texttt{pytest}} a été choisi pour sa simplicité et sa puissance. Il automatise la découverte et l'exécution des tests contenus dans le répertoire \texttt{test/}, en recherchant les fichiers nommés \texttt{test\_*.py}.

La philosophie de test adoptée est de se concentrer exclusivement sur la validation des \textbf{Modèles} de l'architecture MVC. L'interface graphique (la Vue) et la logique de coordination (le Contrôleur) sont volontairement exclues du périmètre de ces tests. En effet, la validité scientifique du simulateur réside dans la justesse des calculs thermodynamiques, qui sont entièrement contenus dans les modèles.

Le diagramme suivant illustre ce principe : le script de test interagit directement avec le Modèle, court-circuitant le reste de l'application (voir figure \ref{fig:test_unitaire}).

\begin{figure}[h]
    \centering
    \begin{tikzpicture}[node distance = 2.5cm, auto]
        % Les nœuds
        \node [test_script] (test) {Script de Test \\ (\texttt{test\_*.py})};
        \node [block, below=of test, fill=red!15] (model) {\textbf{Modèle}};

        % Les composants ignorés
        \node [block, right=of model, fill=green!20, opacity=0.3] (controller) {Contrôleur};
        \node [block, right=of controller, fill=yellow!20, opacity=0.3] (view) {Vue};

        % Les flèches
        \path [line] (test) -- node[midway, left] {1. Appelle le modèle} (model);
        \path [line] (model) -- node[midway, right] {2. Retourne le résultat} (test);
        \node at (test.south) [below, yshift=2.5cm, text width=10em, text centered] {3. \texttt{assert} \\ \small (Vérification du résultat)};
    \end{tikzpicture}
    \caption{Diagramme illustrant le principe de test unitaire}
    \label{fig:test_unitaire}
\end{figure}

\subsection{Structure et écriture d'un test : l'approche "Arrange-Act-Assert"}
Chaque test est écrit en suivant le patron de conception classique \textit{Arrange-Act-Assert} (AAA), qui structure le test en trois étapes claires. Prenons l'exemple d'un test pour le module du détendeur (\texttt{test/test\_expansion\_valve.py}).

\begin{enumerate}
    \item \textbf{Arrange (Préparer)} : La première étape consiste à mettre en place toutes les conditions initiales nécessaires pour le test. Cela inclut l'instanciation du modèle à tester et la création d'un objet \texttt{ThermoState} d'entrée avec des propriétés thermodynamiques connues et contrôlées.

          \begin{verbatim}
# ARRANGE
model = ExpansionValveModel()
state_in = ThermoState()
state_in.update(P=1e6, T=308.15) # P=10 bar, T=35°C
P_out = 1.227e6 # Pression de sortie cible
    \end{verbatim}

    \item \textbf{Act (Agir)} : Cette étape consiste à exécuter l'action que l'on souhaite tester. Dans notre cas, il s'agit d'appeler la méthode \texttt{simulate} du modèle avec les objets préparés à l'étape précédente.

          \begin{verbatim}
# ACT
state_out = model.simulate(state_in, P_out)
    \end{verbatim}

    \item \textbf{Assert (Vérifier)} : C'est l'étape de validation. On utilise une ou plusieurs instructions \texttt{assert} pour vérifier que le résultat obtenu est conforme à ce qui est attendu. Pour un détendeur, on doit vérifier la conservation de l'enthalpie, qui est la loi physique fondamentale de ce composant.

          \begin{verbatim}
# ASSERT
assert state_out.P == P_out
assert state_out.h == pytest.approx(state_in.h)
    \end{verbatim}
          L'utilisation de \texttt{pytest.approx()} est cruciale ici, car elle permet de comparer des nombres à virgule flottante avec une tolérance, évitant ainsi les échecs de test dus à des imprécisions de calcul infimes.
\end{enumerate}

Cette approche, appliquée à chaque module, permet de construire un filet de sécurité robuste. Avant chaque modification majeure du code, l'exécution de la suite de tests complète via la commande \texttt{pytest -v} garantit qu'aucune régression n'a été introduite et que les fondations scientifiques du simulateur restent intactes.

\section{Outils et Environnement de Développement}

La robustesse d'un projet logiciel ne dépend pas seulement de son architecture, mais aussi de la qualité et de la standardisation de son outillage. Ce projet s'appuie sur des standards modernes de l'écosystème Python pour définir sa configuration, gérer ses dépendances et assurer sa reproductibilité.

\subsection{Le fichier de configuration central : \texttt{pyproject.toml}}
Conformément aux recommandations actuelles de la communauté Python (PEP 518), le fichier \texttt{pyproject.toml} sert de point de configuration centralisé pour l'ensemble du projet. Il remplace les anciens fichiers de configuration dispersés (comme \texttt{setup.py} ou \texttt{requirements.txt} dans certains usages) et déclare de manière unifiée :
\begin{itemize}
    \item Les méta-informations du projet (nom, version, auteur).
    \item Le système de build utilisé.
    \item Les dépendances de production et de développement.
\end{itemize}

\subsection{Les dépendances du projet}
Le contenu du fichier \texttt{pyproject.toml} spécifie les bibliothèques externes sur lesquelles le simulateur s'appuie pour fonctionner.

\begin{lstlisting}[language=Python, caption={Extrait des dépendances dans \texttt{pyproject.toml}}]
[project]
dependencies = [
    "coolprop",
    "numpy",
    "scipy",
    "matplotlib",
]
\end{lstlisting}

Chaque dépendance joue un rôle critique :
\begin{itemize}
    \item \textbf{CoolProp} : La bibliothèque fondamentale pour tous les calculs de propriétés thermodynamiques. Elle garantit la précision scientifique des données pour le fluide R718.
    \item \textbf{NumPy} : Essentielle pour les calculs numériques, notamment la manipulation de tableaux et les opérations mathématiques vectorielles, offrant des performances élevées.
    \item \textbf{SciPy} : Utilisée pour des fonctions scientifiques plus avancées, comme la recherche de racines ou l'optimisation, qui peuvent être nécessaires dans les modèles physiques complexes.
    \item \textbf{Matplotlib} : La bibliothèque de visualisation de données, utilisée pour générer les diagrammes P-h et autres graphiques directement dans l'interface utilisateur.
\end{itemize}

L'interface graphique elle-même s'appuie sur \textbf{Tkinter}, qui a l'avantage d'être incluse dans la bibliothèque standard de Python et ne nécessite donc pas d'être déclarée comme une dépendance externe.

\subsection{Outils de développement}
Pour assurer la qualité du code, des outils spécifiques au développement sont également définis, notamment :
\begin{itemize}
    \item \textbf{Pytest} : Le framework de test utilisé pour la validation unitaire de chaque modèle, comme détaillé dans la section précédente.
\end{itemize}

Pour une analyse plus approfondie justifiant le choix de ces technologies et des patrons d'architecture, le lecteur est invité à consulter l'\textbf{Annexe B}, qui détaille les avantages et les compromis liés à l'utilisation de l'architecture MVC, des principes ACID, ainsi que des bibliothèques Matplotlib, CoolProp et Tkinter dans le contexte de ce projet.

% \hrulefill
\vspace{1cm}

\section*{Résumé du chapitre}
Ce chapitre a présenté en détail l'architecture logicielle et le fonctionnement interne du simulateur. Partant de l'\textbf{arborescence du projet}, qui sépare clairement le code source, les tests et la documentation, nous avons exploré l'\textbf{architecture modulaire} basée sur le patron de conception MVC. Cette approche garantit la séparation des responsabilités entre la logique métier (Modèle), l'interface (Vue) et la coordination (Contrôleur).

Nous avons ensuite analysé le \textbf{protocole d'échange}, un duo de classes composé de \texttt{ThermoState} et \texttt{PropsService}, qui agit comme un moteur de calcul fiable en standardisant la communication et en centralisant l'accès aux propriétés thermodynamiques. Les \textbf{points d'entrée} de l'application et le fonctionnement de l'interface d'accueil ont été décrits, montrant comment l'utilisateur est guidé vers les différents modules.

La stratégie de \textbf{validation par les tests unitaires} a été mise en avant, démontrant comment la fiabilité de chaque modèle physique est assurée de manière isolée via le framework \texttt{pytest}. Enfin, la section a conclu sur les \textbf{outils et dépendances} définis dans le fichier \texttt{pyproject.toml}, qui constituent le socle technique de l'environnement de développement et de production.

% ==========================================================================
% chapitre_III_modelisation.tex
% Chapitre III — Modélisation des composants
% À inclure via : \input{chapitre_III_modelisation}
% ==========================================================================

\chapter{Modélisation des composants}

Ce chapitre présente en détail la modélisation physique et mathématique de chaque composant du cycle frigorifique à éjecteur utilisant l'eau (R718) comme fluide de travail. Pour chaque composant, nous décrivons le rôle dans le système, le modèle mathématique retenu, les hypothèses simplificatrices, les équations implémentées, l'algorithme de résolution, la structure du code Python ainsi que les contrôles de cohérence associés.

La convention de numérotation des états thermodynamiques adoptée est la suivante (figure~\ref{fig:convention_etats}) :

\begin{itemize}
    \item $1 \rightarrow 2$ : Détendeur (détente isenthalpique)
    \item $2 \rightarrow 3$ : Évaporateur (évaporation)
    \item $3 \rightarrow 4$ : Chambre de mélange de l'éjecteur (aspiration secondaire)
    \item $4 \rightarrow 5$ : Diffuseur de l'éjecteur (compression subsonique)
    \item $5 \rightarrow 6$ : Condenseur (condensation)
    \item $1 \rightarrow 7$ : Pompe (compression du liquide)
    \item $7 \rightarrow 8$ : Chaudière solaire (chauffage et vaporisation)
    \item $8 \rightarrow 4$ : Tuyère de l'éjecteur (expansion supersonique du primaire)
\end{itemize}

\begin{figure}[h]
    \centering
    \begin{tikzpicture}[node distance=2cm, auto]
        % États du cycle
        \node (1) [block] {1: Liquide saturé \\ $P_\text{cond}$, $x=0$};
        \node (2) [block, right=of 1] {2: Détente \\ $P_\text{evap}$, $h=h_1$};
        \node (3) [block, right=of 2] {3: Vapeur évaporée \\ $P_\text{evap}$, $x=1$};
        \node (4) [block, below=of 3, xshift=-2cm] {4: Mélange éjecteur \\ $P_\text{mix}$};
        \node (5) [block, left=of 4] {5: Compression éjecteur \\ $P_\text{cond}$};
        \node (6) [block, left=of 5] {6: Liquide sous-refroidi \\ $P_\text{cond}$, $x=0$};
        \node (7) [block, above=of 6, xshift=2cm] {7: Compression pompe \\ $P_\text{gen}$};
        \node (8) [block, right=of 7] {8: Vapeur générée \\ $P_\text{gen}$, $x=1$};

        % Flèches
        \path [line] (1) -- node[above] {Détendeur} (2);
        \path [line] (2) -- node[above] {Évaporateur} (3);
        \path [line] (3) -- node[right] {Éjecteur} (4);
        \path [line] (4) -- node[below] {Éjecteur} (5);
        \path [line] (5) -- node[below] {Condenseur} (6);
        \path [line] (6) -- node[left] {Pompe} (7);
        \path [line] (7) -- node[above] {Générateur} (8);
        \path [line] (8) -- node[left] {Éjecteur} (4);
    \end{tikzpicture}
    \caption{Convention de numérotation des états thermodynamiques du cycle frigorifique à éjecteur}
    \label{fig:convention_etats}
\end{figure}

L'état 1 constitue le point de bifurcation du cycle : liquide saturé à la pression de condensation ($P_\text{cond}$, $x_1 = 0$). De ce point, le fluide se divise entre la branche froide basse pression ($1 \rightarrow 2 \rightarrow 3$) et la branche chaude haute pression ($1 \rightarrow 7 \rightarrow 8$). Les deux branches convergent dans l'éjecteur (état~4), puis le fluide traverse le condenseur ($5 \rightarrow 6$) pour revenir à l'état~1.

Toutes les propriétés thermodynamiques sont calculées par la bibliothèque \textit{CoolProp} via le service centralisé \texttt{PropsService} (patron Singleton). Les états sont transportés entre les modules sous forme d'objets \texttt{ThermoState} contenant les grandeurs $P$, $T$, $h$, $s$, $x$ et $\rho$.

\newpage
% ======================================================================
% SECTION 1 : DÉTENDEUR
% ======================================================================
\section{Le Détendeur (\textit{Expansion Valve})}

\subsection{Rôle dans le système}

Le détendeur réalise la transformation $1 \rightarrow 2$. Il abaisse la pression du fluide depuis la pression de condensation $P_\text{cond}$ jusqu'à la pression d'évaporation $P_\text{evap}$, sans échange de travail ni de chaleur avec l'extérieur. En sortie, le fluide se retrouve dans un état diphasique à basse pression, prêt à absorber de la chaleur dans l'évaporateur.

\subsection{Modèle mathématique retenu}

Le modèle retenu est la \textbf{détente isenthalpique}, complétée par un \textbf{modèle d'orifice optionnel} permettant d'estimer le débit massique.

\subsection{Hypothèses}

\begin{enumerate}
    \item Processus adiabatique : aucun échange de chaleur avec l'environnement.
    \item Transformation isenthalpique : l'enthalpie se conserve ($h_2 = h_1$).
    \item Aucun travail échangé.
    \item Régime permanent et écoulement unidimensionnel.
    \item Le modèle d'orifice suppose un fluide incompressible en amont.
\end{enumerate}

\subsection{Équations utilisées}

\textbf{Conservation de l'enthalpie :}
\begin{equation}
    h_2 = h_1
    \label{eq:detendeur_isenthalp}
\end{equation}

L'état de sortie est entièrement déterminé par le couple $(P_2, h_2) = (P_\text{evap}, h_1)$, à partir duquel \textit{CoolProp} calcule $T_2$, $s_2$, $x_2$ et $\rho_2$.

\textbf{Modèle d'orifice (optionnel) :}
\begin{equation}
    \dot{m} = C_d \cdot A \cdot \sqrt{2\,\rho_1 \cdot \Delta P}
    \label{eq:detendeur_orifice}
\end{equation}
où :
\begin{itemize}
    \item $C_d$ : coefficient de décharge (par défaut $C_d = 0{,}8$, $0 < C_d \leq 1$),
    \item $A$ : aire de l'orifice [$\text{m}^2$] (par défaut $A = 10^{-6}$~$\text{m}^2$),
    \item $\rho_1$ : masse volumique à l'entrée [$\text{kg/m}^3$],
    \item $\Delta P = P_1 - P_2$ : différence de pression [$\text{Pa}$].
\end{itemize}

\subsection{Algorithme d'implémentation}

\begin{enumerate}
    \item Vérifier que l'état d'entrée $\text{state}_1$ est initialisé.
    \item Initialiser les drapeaux de diagnostic.
    \item Vérifier si $P_\text{out} < 1100$~Pa (alerte vide poussé pour R718).
    \item Calculer $\Delta P = P_1 - P_\text{out}$ et vérifier $\Delta P > 0$.
    \item Créer l'état de sortie $\text{state}_2$ via \texttt{update\_from\_PH}$(P_\text{out}, h_1)$.
    \item Vérifier si l'état de sortie est diphasique ($x_2 \neq \texttt{None}$).
    \item Si le modèle d'orifice est activé, calculer $\dot{m}$ via l'équation~(\ref{eq:detendeur_orifice}).
    \item Retourner le résultat \texttt{ExpansionValveResult}.
\end{enumerate}

\subsection{Structure du code Python}
\begin{figure}[h]
    \centering
    \begin{tikzpicture}[node distance=0.6cm]

        % === ExpansionValveModel ===
        \node (title1) [umltitle] {ExpansionValveModel};
        \node (attr1) [umlsection, below=0cm of title1] {
            -- use\_orifice\_flow : bool \\
            -- Cd : float = 0.8 \\
            -- A\_orifice : float = 1e-6
        };
        \node (meth1) [umlsection, below=2cm of attr1] {
            + solve(state1, P\_out) \\
            -- \_calculate\_orifice\_flow() \\
            + set\_orifice\_parameters()
        };

        % === ExpansionValveResult ===
        \node (title2) [umldatatitle, right=3cm of title1] {$\ll$dataclass$\gg$ \\ ExpansionValveResult};
        \node (attr2) [umlsection, below=0cm of title2] {
            + state2 : ThermoState \\
            + m\_dot : Optional[float] \\
            + flags : Dict[str, bool]
        };

        % === ThermoState (simplifié) ===
        \node (title3) [umlcoretitle, below=2.5cm of title1] {ThermoState};
        \node (attr3) [umlsection, below=0cm of title3] {
            + P, T, h, s, x, rho \\
            + update\_from\_PH(P, h)
        };

        % Flèches
        \draw[uml_returns] (meth1.east) -- ++(0.5,0) |- node[near start, above, font=\tiny\sffamily] {retourne} (attr2.west);
        \draw[uml_uses] (attr1.south) -- node[midway, left, font=\tiny\sffamily] {utilise} (title3.north);

    \end{tikzpicture}
    \caption{Diagramme de classes du module Détendeur}
    \label{fig:uml_detendeur}
\end{figure}

Le module est situé dans \texttt{src/app\_r718/modules/expansion\_valve/}. La classe \texttt{ExpansionValveModel} est initialisée avec les paramètres du modèle d'orifice (\texttt{use\_orifice\_flow}, $C_d$, $A$). La méthode principale \texttt{solve(state1, P\_out)} retourne un objet \texttt{ExpansionValveResult} contenant :
\begin{itemize}
    \item \texttt{state2} : l'objet \texttt{ThermoState} de sortie,
    \item \texttt{m\_dot} : le débit massique (si le modèle d'orifice est activé),
    \item \texttt{flags} : dictionnaire de drapeaux diagnostiques.
\end{itemize}

\subsection{Flags et contrôles de cohérence associés}

\begin{table}[h]
    \centering
    \begin{tabular}{|l|p{9cm}|}
        \hline
        \textbf{Flag}                  & \textbf{Signification}                                    \\ \hline
        \texttt{deep\_vacuum\_warning} & $P_\text{out} < 1100$~Pa : pression proche du vide poussé \\ \hline
        \texttt{two\_phase\_outlet}    & L'état de sortie est diphasique ($x_2 \in [0, 1]$)        \\ \hline
        \texttt{invalid\_delta\_p}     & $\Delta P \leq 0$ : pas de chute de pression              \\ \hline
    \end{tabular}
    \caption{Drapeaux diagnostiques du module détendeur}
    \label{tab:flags_detendeur}
\end{table}

\subsection{Limites du modèle}

\begin{itemize}
    \item Le processus isenthalpique est une idéalisation. En pratique, de légères pertes de chaleur existent.
    \item Le modèle d'orifice suppose un écoulement incompressible monophasique en amont, ce qui n'est pas toujours vérifié pour R718 aux très basses pressions.
    \item Le comportement critique (cavitation, dégazage) n'est pas modélisé.
\end{itemize}


% ======================================================================
% SECTION 2 : ÉVAPORATEUR
% ======================================================================
\section{L'Évaporateur}

\subsection{Rôle dans le système}

L'évaporateur réalise la transformation $2 \rightarrow 3$. Il absorbe la chaleur du milieu à refroidir pour vaporiser le fluide diphasique provenant du détendeur. En sortie, le fluide est sous forme de vapeur saturée (ou légèrement surchauffée) à la pression d'évaporation $P_\text{evap}$.

\subsection{Modèle mathématique retenu}

Le modèle retenu couple un \textbf{bilan énergétique massique} à un \textbf{modèle d'échangeur de chaleur} basé sur la différence de température logarithmique moyenne (DTLM).

\subsection{Hypothèses}

\begin{enumerate}
    \item Évaporation à pression constante ($P_\text{evap}$).
    \item Régime permanent, écoulement unidimensionnel.
    \item L'état de sortie est vapeur saturée ($x_3 = 1$) ou vapeur surchauffée si une surchauffe $\Delta T_\text{sh}$ est spécifiée.
    \item La température de saturation $T_\text{sat}$ est supposée constante sur toute la surface d'échange côté réfrigérant.
    \item Le fluide externe (eau glacée) évolue de $T_\text{ext,in}$ à $T_\text{ext,out}$.
\end{enumerate}

\subsection{Équations utilisées}

\textbf{Bilan d'énergie massique :}
\begin{equation}
    Q_\text{mass} = \dot{m}_\text{sec} \cdot (h_3 - h_2)
    \label{eq:evap_Qmass}
\end{equation}

\textbf{Modèle d'échangeur de chaleur :}
\begin{equation}
    Q_{KA} = K \cdot A \cdot \Delta T_\text{lm}
    \label{eq:evap_QKA}
\end{equation}

\textbf{Différence de température logarithmique moyenne (DTLM) :}
\begin{equation}
    \Delta T_\text{lm} = \frac{\Delta T_1 - \Delta T_2}{\ln\left(\dfrac{\Delta T_1}{\Delta T_2}\right)}
    \label{eq:DTLM}
\end{equation}
avec :
\begin{align}
    \Delta T_1 & = T_\text{ext,in} - T_\text{sat}  \\
    \Delta T_2 & = T_\text{ext,out} - T_\text{sat}
\end{align}

Lorsque $\Delta T_1 \approx \Delta T_2$ ($|\Delta T_1 - \Delta T_2| < 10^{-6}$), l'expression dégénère et la DTLM est remplacée par $\Delta T_\text{lm} = \Delta T_1$.

\textbf{Écart relatif entre les deux méthodes :}
\begin{equation}
    \delta_\text{rel} = \frac{|Q_\text{mass} - Q_{KA}|}{\max(|Q_\text{mass}|,\, \varepsilon)}
    \label{eq:evap_delta}
\end{equation}
où $\varepsilon = 1{,}0$~W pour éviter la division par zéro.

\subsection{Algorithme d'implémentation}

\begin{enumerate}
    \item Calculer $T_\text{sat}$ à $P_\text{evap}$ via \texttt{PropsService.Tsat\_P}.
    \item Construire l'état de sortie $\text{state}_3$ :
          \begin{itemize}
              \item si surchauffe : \texttt{update\_from\_PT}$(P_\text{evap}, T_\text{sat} + \Delta T_\text{sh})$,
              \item sinon : \texttt{update\_from\_PX}$(P_\text{evap}, 1{,}0)$.
          \end{itemize}
    \item Vérifier $h_3 > h_2$ (évaporation complète).
    \item Calculer $Q_\text{mass}$ (éq.~\ref{eq:evap_Qmass}).
    \item Calculer la DTLM (éq.~\ref{eq:DTLM}) et $Q_{KA}$ (éq.~\ref{eq:evap_QKA}).
    \item Calculer l'écart relatif $\delta_\text{rel}$ (éq.~\ref{eq:evap_delta}).
    \item Retourner \texttt{EvaporatorResult}.
\end{enumerate}

\subsection{Structure du code Python}
\begin{figure}[h]
    \centering
    \begin{tikzpicture}[node distance=0.6cm]

        % === EvaporatorModel ===
        \node (title1) [umltitle] {EvaporatorModel};
        \node (attr1) [umlsection, below=0cm of title1] {
            -- props : PropsService
        };
        \node (meth1) [umlsection, below=0cm of attr1] {
            + solve(state2, m\_dot, P\_evap, \\
            \quad K, A, T\_ext\_in, T\_ext\_out, \\
            \quad superheat\_K)
        };

        % === EvaporatorResult ===
        \node (title2) [umldatatitle, right=3cm of title1] {$\ll$dataclass$\gg$ \\ EvaporatorResult};
        \node (attr2) [umlsection, below=0cm of title2] {
            + state3 : ThermoState \\
            + Q\_mass : float \\
            + Q\_KA : float \\
            + delta\_relative : float \\
            + flags : Dict[str, bool]
        };

        % === PropsService (simplifié) ===
        \node (title3) [umlcoretitle, below=6cm of title1] {$\ll$singleton$\gg$ \\ PropsService};
        \node (attr3) [umlsection, below=0cm of title3] {
            + Tsat\_P(P) \\
            + Psat\_T(T)
        };

        % Flèches
        \draw[uml_returns] (meth1.east) -- ++(0.5,0) |- node[near start, above, font=\tiny\sffamily] {retourne} (attr2.west);
        \draw[uml_uses] (attr1.south) -- node[midway, left, font=\tiny\sffamily] {délègue à} (title3.north);

    \end{tikzpicture}
    \caption{Diagramme de classes du module Évaporateur}
    \label{fig:uml_evaporateur}
\end{figure}

La classe \texttt{EvaporatorModel} est définie dans \texttt{src/app\_r718/modules/evaporator/model.py}. Elle utilise le service \texttt{PropsService} (obtenu via \texttt{get\_props\_service()}) pour les calculs de saturation. La méthode \texttt{solve} prend en paramètres l'état d'entrée $\text{state}_2$, le débit massique $\dot{m}$, la pression d'évaporation $P_\text{evap}$, les caractéristiques de l'échangeur ($K$, $A$), les températures du fluide externe ($T_\text{ext,in}$, $T_\text{ext,out}$) et la surchauffe optionnelle $\Delta T_\text{sh}$.

Le résultat est encapsulé dans un objet \texttt{EvaporatorResult} contenant : \texttt{state3}, $Q_\text{mass}$, $Q_{KA}$, $\delta_\text{rel}$ et les drapeaux.

\subsection{Flags et contrôles de cohérence associés}

\begin{table}[h]
    \centering
    \begin{tabular}{|l|p{9cm}|}
        \hline
        \textbf{Flag}                     & \textbf{Signification}                                                        \\ \hline
        \texttt{incomplete\_evaporation}  & $h_3 \leq h_2$ : l'énergie de sortie est insuffisante                         \\ \hline
        \texttt{negative\_heat\_transfer} & $Q_\text{mass} \leq 0$ ou $Q_{KA} \leq 0$                                     \\ \hline
        \texttt{invalid\_LMTD}            & $\Delta T_1 \leq 0$ ou $\Delta T_2 \leq 0$ : profil de température incohérent \\ \hline
        \texttt{thermal\_mismatch}        & $\delta_\text{rel} > 5\%$ : désaccord entre $Q_\text{mass}$ et $Q_{KA}$       \\ \hline
    \end{tabular}
    \caption{Drapeaux diagnostiques du module évaporateur}
    \label{tab:flags_evap}
\end{table}

\subsection{Limites du modèle}

\begin{itemize}
    \item La température de saturation est supposée uniforme côté réfrigérant, ce qui est une simplification pour un échangeur réel à changement de phase.
    \item Le coefficient d'échange global $K$ est un paramètre fixé et non calculé à partir de corrélations d'ébullition en film.
    \item Le modèle ne résout pas le profil de température le long de l'échangeur (modèle 0D).
\end{itemize}


% ======================================================================
% SECTION 3 : ÉJECTEUR
% ======================================================================
\section{L'Éjecteur}

L'éjecteur est le composant central et le plus complexe du cycle. Il réalise simultanément les transformations $8 \rightarrow 4$ (tuyère primaire), $3 \rightarrow 4$ (aspiration secondaire) et $4 \rightarrow 5$ (diffuseur). Deux modèles sont implémentés : un modèle thermodynamique simplifié (V1) et un modèle compressible avec onde de choc normale (V2). Le modèle V2 est utilisé par défaut.

\subsection{Rôle dans le système}

L'éjecteur utilise l'énergie cinétique du fluide primaire haute pression (vapeur issue de la chaudière, état~8) pour entraîner et comprimer le fluide secondaire basse pression (vapeur issue de l'évaporateur, état~3). Il se substituetypiquement au compresseur mécanique des cycles classiques. Le rapport d'entraînement $\mu$ est le paramètre clé :
\begin{equation}
    \mu = \frac{\dot{m}_\text{sec}}{\dot{m}_\text{pri}}
    \label{eq:mu}
\end{equation}

\subsection{Modèle mathématique retenu}

\subsubsection{Modèle V1 : Approche thermodynamique simplifiée}

Le modèle V1 (\texttt{model.py}) utilise une approche à rendements de composants (tuyère, diffuseur) avec estimation de la pression de mélange par moyenne géométrique et calcul du rapport d'entraînement par optimisation.

\subsubsection{Modèle V2 : Écoulement compressible 1D avec onde de choc (\texttt{model\_v2.py})}

Le modèle V2 est une extension majeure qui introduit :
\begin{itemize}
    \item Le calcul des nombres de Mach dans la tuyère et la chambre de mélange,
    \item La détection du blocage sonique (\textit{choking}),
    \item La modélisation de l'onde de choc normale via les \textbf{relations de Rankine-Hugoniot},
    \item Le calcul du saut d'entropie à travers le choc,
    \item L'identification du régime d'écoulement (subsonique, critique, supersonique).
\end{itemize}

\subsection{Hypothèses}

\begin{enumerate}
    \item Régime permanent, écoulement unidimensionnel (1D).
    \item Processus adiabatique (pas d'échange de chaleur avec l'extérieur).
    \item La vapeur d'eau est approximée comme un gaz parfait avec $\gamma = 1{,}33$ et $R_\text{sp} = 461{,}5$~J/(kg$\cdot$K).
    \item Les rendements isentropiques de la tuyère ($\eta_\text{tuy}$), du diffuseur ($\eta_\text{diff}$) et du mélange ($\eta_\text{mix}$) sont des paramètres imposés (défauts : $\eta_\text{tuy} = 0{,}85$, $\eta_\text{diff} = 0{,}85$, $\eta_\text{mix} = 1{,}0$).
    \item Le nombre de Mach maximal d'entrainement est limité à $\mu_\text{max} = 5{,}0$.
    \item La vitesse du fluide secondaire à l'entrée est supposée faible ($c_\text{sec} \approx 10$~m/s).
\end{enumerate}

\subsection{Équations utilisées}

\subsubsection{Tuyère primaire (état $8 \rightarrow$ sortie tuyère)}

\textbf{Vitesse du son (gaz parfait) :}
\begin{equation}
    a = \sqrt{\gamma \cdot R_\text{sp} \cdot T}
    \label{eq:sound_speed}
\end{equation}

\textbf{Nombre de Mach :}
\begin{equation}
    M = \frac{c}{a}
    \label{eq:mach}
\end{equation}

\textbf{Rapport de pression critique (blocage sonique) :}
\begin{equation}
    \frac{P^*}{P_0} = \left(\frac{2}{\gamma + 1}\right)^{\frac{\gamma}{\gamma - 1}}
    \label{eq:critical_ratio}
\end{equation}

Le blocage sonique est détecté lorsque $P_\text{mix}/P_{8} < P^*/P_0$.

\textbf{Nombre de Mach à partir du rapport de pressions isentropique :}
\begin{equation}
    M^2 = \frac{2}{\gamma - 1}\left[\left(\frac{P}{P_0}\right)^{-\frac{\gamma-1}{\gamma}} - 1\right]
    \label{eq:mach_from_P}
\end{equation}

\textbf{Détente réelle dans la tuyère :}
\begin{align}
    h_\text{tuy,is} & = h_\text{PS}(P_\text{mix},\, s_8) \label{eq:nozzle_is}                 \\
    h_\text{tuy}    & = h_8 - \eta_\text{tuy}\,(h_8 - h_\text{tuy,is}) \label{eq:nozzle_real}
\end{align}

\textbf{Vitesse en sortie de tuyère (conservation de l'énergie) :}
\begin{equation}
    c_\text{pri} = \sqrt{2\,(h_8 - h_\text{tuy})}
    \label{eq:velocity_nozzle}
\end{equation}

\subsubsection{Chambre de mélange (bilan de quantité de mouvement et d'énergie)}

\textbf{Conservation de la quantité de mouvement :}
\begin{equation}
    \dot{m}_\text{pri} \cdot c_\text{pri} + \dot{m}_\text{sec} \cdot c_\text{sec} = (\dot{m}_\text{pri} + \dot{m}_\text{sec}) \cdot c_\text{mix}
    \label{eq:momentum}
\end{equation}

\textbf{Conservation de l'énergie (enthalpie de stagnation) :}
\begin{equation}
    h_\text{mix,stag} = \frac{\dot{m}_\text{pri} \cdot h_\text{tuy} + \dot{m}_\text{sec} \cdot h_3}{\dot{m}_\text{pri} + \dot{m}_\text{sec}}
    \label{eq:energy_mix}
\end{equation}

\textbf{Enthalpie statique du mélange :}
\begin{equation}
    h_\text{mix,stat} = h_\text{mix,stag} - \frac{c_\text{mix}^2}{2}
    \label{eq:h_static_mix}
\end{equation}

La pression de mélange $P_\text{mix}$ est estimée par la moyenne géométrique :
\begin{equation}
    P_\text{mix} = \sqrt{P_\text{evap} \cdot P_\text{cond}}
    \label{eq:Pmix}
\end{equation}
bornée par $P_\text{evap} \cdot 1{,}01 \leq P_\text{mix} \leq P_\text{cond} \cdot 0{,}99$.

Le rapport d'entraînement $\mu$ est déterminé par optimisation numérique (méthode \texttt{minimize\_scalar} de \textit{SciPy}, bornes $[0;\, 3]$) en minimisant le travail de compression dans le diffuseur, ce qui correspond physiquement à la meilleure récupération de pression.

\subsubsection{Onde de choc normale (Rankine-Hugoniot)}

Si le nombre de Mach du mélange dépasse le seuil $M_\text{mix} > 1 + 10^{-6}$, une onde de choc normale se forme. Les relations de Rankine-Hugoniot sont appliquées :

\textbf{Rapport de pressions à travers le choc :}
\begin{equation}
    \frac{P_\text{ap}}{P_\text{av}} = 1 + \frac{2\gamma}{\gamma + 1}\,(M_\text{av}^2 - 1)
    \label{eq:shock_P}
\end{equation}

\textbf{Nombre de Mach en aval du choc :}
\begin{equation}
    M_\text{ap}^2 = \frac{1 + \dfrac{\gamma - 1}{2}\,M_\text{av}^2}{\gamma\,M_\text{av}^2 - \dfrac{\gamma-1}{2}}
    \label{eq:shock_M}
\end{equation}

\textbf{Rapport de masse volumique :}
\begin{equation}
    \frac{\rho_\text{ap}}{\rho_\text{av}} = \frac{(\gamma+1)\,M_\text{av}^2}{2 + (\gamma-1)\,M_\text{av}^2}
    \label{eq:shock_rho}
\end{equation}

\textbf{Rapport de températures :}
\begin{equation}
    \frac{T_\text{ap}}{T_\text{av}} = \frac{P_\text{ap}/P_\text{av}}{\rho_\text{ap}/\rho_\text{av}}
    \label{eq:shock_T}
\end{equation}

\textbf{Saut d'entropie (gaz parfait) :}
\begin{equation}
    \Delta s_\text{choc} = c_p \cdot \ln\!\left(\frac{T_\text{ap}}{T_\text{av}}\right) - R_\text{sp} \cdot \ln\!\left(\frac{P_\text{ap}}{P_\text{av}}\right)
    \label{eq:shock_ds}
\end{equation}
avec $c_p = \dfrac{\gamma \cdot R_\text{sp}}{\gamma - 1}$.

L'enthalpie de stagnation est conservée à travers le choc ($h_\text{ap} = h_\text{av}$). Après le calcul idéal (gaz parfait), le saut d'entropie est recalculé à partir des états \textit{CoolProp} réels pour assurer la cohérence thermodynamique. Le deuxième principe est vérifié : $\Delta s > 0$.

\subsubsection{Diffuseur (compression subsonique, état $4 \rightarrow 5$)}

\textbf{Compression isentropique :}
\begin{equation}
    h_\text{5,is} = h_\text{PS}(P_\text{cond},\, s_\text{mix})
    \label{eq:diff_is}
\end{equation}

\textbf{Compression réelle :}
\begin{equation}
    h_5 = h_\text{mix} + \frac{h_\text{5,is} - h_\text{mix}}{\eta_\text{diff}}
    \label{eq:diff_real}
\end{equation}

L'état de sortie est obtenu par \texttt{update\_from\_PH}$(P_\text{cond}, h_5)$.

\subsection{Algorithme d'implémentation (V2)}

\begin{enumerate}
    \item Valider les rendements ($0 < \eta \leq 1$) ; en cas de valeur invalide, utiliser les défauts.
    \item Valider la hiérarchie des pressions : $P_\text{sec} < P_\text{cond} < P_\text{pri}$.
    \item Estimer $P_\text{mix}$ par la moyenne géométrique.
    \item \textbf{Tuyère} : Expansion isentropique de l'état~8 à $P_\text{mix}$, appliquer $\eta_\text{tuy}$, calculer $c_\text{pri}$ et $M_\text{tuy}$.
    \item \textbf{Secondaire} : Initialiser $c_\text{sec} = 10$~m/s.
    \item \textbf{Mélange} : Optimiser $\mu$ via \texttt{minimize\_scalar} (bornes $[0;\, 3]$), bilans de quantité de mouvement et d'énergie.
    \item Calculer $c_\text{mix}$, $h_\text{mix,stag}$, $h_\text{mix,stat}$, $M_\text{mix}$.
    \item \textbf{Choc normal} : Si $M_\text{mix} > 1{,}000001$, appliquer Rankine-Hugoniot, calculer $\Delta s$, mettre à jour l'état post-choc via \textit{CoolProp}.
    \item \textbf{Diffuseur} : Compression à $P_\text{cond}$ avec $\eta_\text{diff}$.
    \item Diagnostics : vérifier la cohérence du mélange, le régime, la récupération de pression.
    \item Retourner \texttt{EjectorResultV2}.
\end{enumerate}

\subsection{Structure du code Python}

\begin{figure}[h]
    \centering
    \begin{tikzpicture}[node distance=0.6cm]

        % === EjectorModel (V1) ===
        \node (v1title) [umltitle] {EjectorModel \textit{(V1)}};
        \node (v1attr) [umlsection, below=0cm of v1title] {
            -- props : PropsService
        };
        \node (v1meth) [umlsection, below=0cm of v1attr] {
            + solve(state\_p\_in, state\_s\_in, \\
            \quad P\_out, m\_dot\_p, $\eta$...)
        };

        % === EjectorModelV2 ===
        \node (v2title) [umltitle, below=1.5cm of v1meth] {EjectorModelV2 \textit{(V2)}};
        \node (v2attr) [umlsection, below=0cm of v2title] {
            GAMMA = 1.33 \\
            R\_SPECIFIC = 461.5 \\
            MACH\_SHOCK\_THRESHOLD
        };
        \node (v2meth) [umlsection, below=0cm of v2attr] {
            + solve\_v2(...) \\
            + compute\_sound\_speed(T) \\
            + compute\_mach\_number(c, T) \\
            + is\_choked(P\_exit, P\_stag) \\
            + apply\_normal\_shock(M, P, T, h) \\
            + compute\_critical\_pressure\_ratio()
        };

        % === EjectorResult ===
        \node (r1title) [umldatatitle, right=3.5cm of v1title] {$\ll$dataclass$\gg$ \\ EjectorResult};
        \node (r1attr) [umlsection, below=0cm of r1title] {
            + mu, m\_dot\_p, m\_dot\_s \\
            + P\_mix : float \\
            + state\_p\_noz, state\_s\_adj \\
            + state\_mix, state\_out \\
            + flags, notes
        };

        % === EjectorResultV2 ===
        \node (r2title) [umldatatitle, below=1.2cm of r1attr] {$\ll$dataclass$\gg$ \\ EjectorResultV2};
        \node (r2attr) [umlsection, below=0cm of r2title] {
            + mach\_primary\_nozzle \\
            + mach\_before\_shock \\
            + mach\_after\_shock \\
            + shock\_location, regime \\
            + entropy\_jump \\
            + P\_before/after\_shock \\
            + compression\_ratio \\
            + state\_before/after\_shock
        };

        % Héritage V2 -> V1
        \draw[uml_inherits] (v2title.north) -- (v1meth.south);

        % Héritage ResultV2 -> Result
        \draw[uml_inherits] (r2title.north) -- (r1attr.south);

        % Retourne
        \draw[uml_returns] (v1meth.east) -- ++(0.3,0) |- node[near start, above, font=\tiny\sffamily] {retourne} (r1attr.west);
        \draw[uml_returns] (v2meth.east) -- ++(0.3,0) |- node[near start, above, font=\tiny\sffamily] {retourne} (r2attr.west);

    \end{tikzpicture}
    \caption{Diagramme de classes du module Éjecteur (V1 et V2 avec héritage)}
    \label{fig:uml_ejector}
\end{figure}

La classe \texttt{EjectorModelV2} hérite de \texttt{EjectorModel} (V1) et est définie dans \texttt{model\_v2.py}. Elle définit les constantes thermodynamiques ($\gamma = 1{,}33$, $R_\text{sp} = 461{,}5$~J/(kg$\cdot$K)), le seuil de Mach pour la détection du choc ($1 + 10^{-6}$) et le nombre maximal d'itérations ($100$).

Les méthodes auxiliaires sont :
\begin{itemize}
    \item \texttt{compute\_sound\_speed(T)} : calcule $a = \sqrt{\gamma R_\text{sp} T}$,
    \item \texttt{compute\_mach\_number(c, T)} : calcule $M = c / a$,
    \item \texttt{compute\_critical\_pressure\_ratio()} : calcule le rapport critique de blocage,
    \item \texttt{is\_choked(P\_exit, P\_stag)} : détecte le blocage sonique,
    \item \texttt{compute\_mach\_from\_pressure\_ratio(P/P\_0)} : inverse la relation isentropique,
    \item \texttt{compute\_velocity\_from\_enthalpy(h\_0, h)} : calcule $c = \sqrt{2(h_0 - h)}$,
    \item \texttt{apply\_normal\_shock(M\_1, P\_1, T\_1, h\_1)} : applique les relations de Rankine-Hugoniot.
\end{itemize}

% La méthode principale est \texttt{solve\_v2}, qui retourne un objet \texttt{EjectorResultV2} héritant d'\texttt{EjectorResult} avec des attributs additionnels : $M_\text{tuy}$, $M_\text{av}$, $M_\text{ap}$, localisation du choc, régime, $\Delta s$, $P_\text{av}$, $P_\text{ap}$, états avant et après choc, et la condition d'aspiration.
% \item \texttt{compute\_mach\_number(c, T)} : calcule $M = c / a$,
% \item \texttt{compute\_critical\_pressure\_ratio()} : calcule le rapport critique de blocage,
% \item \texttt{is\_choked(P\_exit, P\_stag)} : détecte le blocage sonique,
% \item \texttt{compute\_mach\_from\_pressure\_ratio(P/P\_0)} : inverse la relation isentropique,
% \item \texttt{compute\_velocity\_from\_enthalpy(h\_0, h)} : calcule $c = \sqrt{2(h_0 - h)}$,
% \item \texttt{apply\_normal\_shock(M\textsubscript{1}, P\textsubscript{1}, T\textsubscript{1}, h\textsubscript{1})} : applique les relations de Rankine-Hugoniot.
% \end{itemize}

% La méthode principale est \texttt{solve\_v2}, qui retourne un objet \texttt{EjectorResultV2} héritant d'\texttt{EjectorResult} avec des attributs additionnels : $M_\text{tuy}$, $M_\text{av}$, $M_\text{ap}$, localisation du choc, régime, $\Delta s$, $P_\text{av}$, $P_\text{ap}$, états avant et après choc, et la condition d'aspiration.

\subsection{Flags et contrôles de cohérence associés}

\begin{table}[h]
    \centering
    \begin{tabular}{|l|p{9cm}|}
        \hline
        \textbf{Flag}                      & \textbf{Signification}                                                          \\ \hline
        \texttt{invalid\_efficiency}       & Un des rendements est hors de l'intervalle $]0;\, 1]$                           \\ \hline
        \texttt{invalid\_pressure\_levels} & La hiérarchie $P_\text{sec} < P_\text{cond} < P_\text{pri}$ n'est pas respectée \\ \hline
        \texttt{unphysical\_state}         & Un calcul \textit{CoolProp} a échoué (état invalide)                            \\ \hline
        \texttt{two\_phase\_outlet}        & L'état de sortie du diffuseur est diphasique                                    \\ \hline
        \texttt{poor\_pressure\_recovery}  & $P_\text{mix} > 0{,}95 \cdot P_\text{cond}$                                     \\ \hline
        \texttt{solver\_no\_convergence}   & L'optimisation de $\mu$ n'a pas convergé                                        \\ \hline
    \end{tabular}
    \caption{Drapeaux diagnostiques du module éjecteur}
    \label{tab:flags_ejector}
\end{table}

Des diagnostics supplémentaires sont calculés :
\begin{itemize}
    \item \texttt{regime\_type} : \texttt{"non-entraining"} ($\mu < 0{,}01$), \texttt{"critical"} ou \texttt{"entraining-supersonic"} ($M_\text{av} > 1$),
    \item \texttt{compression\_ratio} : $P_\text{cond} / P_\text{evap}$,
    \item \texttt{pressure\_lift} : $P_\text{cond} - P_\text{evap}$ [Pa],
    \item \texttt{static\_suction\_check} : $P_\text{tuy,sortie} < P_\text{sec}$ (critère d'aspiration),
    \item \texttt{dynamic\_entrainment} : $\mu > 0{,}01$ et $M_\text{tuy} > 1$,
    \item \texttt{physically\_consistent\_mixture} : vérification que $h_\text{mix}$ est dans le dôme de saturation à $P_\text{mix}$,
    \item \texttt{entropy\_jump\_suspect} : alerte si $\Delta s > 0{,}2$~kJ/(kg$\cdot$K) pour un choc faible ($M < 1{,}1$).
\end{itemize}

\subsection{Limites du modèle}

\begin{itemize}
    \item L'approximation gaz parfait ($\gamma = 1{,}33$) pour la vapeur d'eau est une simplification ; le comportement réel de R718 à basse pression peut s'en écarter.
    \item La vitesse secondaire est fixée à $c_\text{sec} = 10$~m/s au lieu d'être calculée de manière auto-consistante.
    \item La pression de mélange est estimée par une moyenne géométrique plutôt que résolue itérativement par le bilan de quantité de mouvement complet.
    \item Le modèle ne distingue pas explicitement les sections constante et convergente de la géométrie de l'éjecteur ; il n'y a pas de modèle de section à aire constante.
    \item Le choc est supposé normal (perpendiculaire à l'écoulement) ; les chocs obliques ne sont pas modélisés.
\end{itemize}


% ======================================================================
% SECTION 4 : CONDENSEUR
% ======================================================================
\section{Le Condenseur}

\subsection{Rôle dans le système}

Le condenseur réalise la transformation $5 \rightarrow 6$. Il rejette la chaleur du fluide de travail vers l'air ambiant par convection naturelle, provoquant la condensation de la vapeur (ou du mélange diphasique) sortant de l'éjecteur en liquide saturé (ou sous-refroidi).

\subsection{Modèle mathématique retenu}

Comme l'évaporateur, le modèle couple un \textbf{bilan énergétique massique} à un \textbf{modèle d'échangeur de chaleur} via la DTLM.

\subsection{Hypothèses}

\begin{enumerate}
    \item Condensation à température constante $T_\text{cond}$ (pression $P_\text{cond} = P_\text{sat}(T_\text{cond})$).
    \item Régime permanent, écoulement unidimensionnel.
    \item L'état de sortie est liquide saturé ($x_6 = 0$) ou liquide sous-refroidi si un sous-refroidissement $\Delta T_\text{sc}$ est spécifié.
    \item L'air de refroidissement évolue de $T_\text{air,in}$ à $T_\text{air,out}$.
    \item Le coefficient d'échange global $K$ est un paramètre imposé.
\end{enumerate}

\subsection{Équations utilisées}

\textbf{Pression de condensation :}
\begin{equation}
    P_\text{cond} = P_\text{sat}(T_\text{cond})
    \label{eq:cond_P}
\end{equation}

\textbf{Bilan énergétique massique :}
\begin{equation}
    Q_\text{mass} = \dot{m}_\text{tot} \cdot (h_5 - h_6)
    \label{eq:cond_Qmass}
\end{equation}
où $\dot{m}_\text{tot} = \dot{m}_\text{pri} + \dot{m}_\text{sec}$.

\textbf{Modèle d'échangeur de chaleur :}
\begin{equation}
    Q_{KA} = K \cdot A \cdot \Delta T_\text{lm}
    \label{eq:cond_QKA}
\end{equation}

\textbf{DTLM côté condenseur :}
\begin{align}
    \Delta T_1 & = T_\text{sat} - T_\text{air,in}  \\
    \Delta T_2 & = T_\text{sat} - T_\text{air,out}
\end{align}

L'écart relatif $\delta_\text{rel}$ est calculé de manière identique à l'évaporateur (éq.~\ref{eq:evap_delta}), avec un seuil de mismatch thermique fixé à 5\%.

\subsection{Algorithme d'implémentation}

\begin{enumerate}
    \item Calculer $P_\text{cond} = P_\text{sat}(T_\text{cond})$ via \texttt{PropsService.Psat\_T}.
    \item Construire l'état de sortie :
          \begin{itemize}
              \item si sous-refroidissement : \texttt{update\_from\_PT}$(P_\text{cond}, T_\text{sat} - \Delta T_\text{sc})$,
              \item sinon : \texttt{update\_from\_PX}$(P_\text{cond}, 0{,}0)$.
          \end{itemize}
    \item Vérifier $h_\text{in} > h_\text{out}$ (condensation effective).
    \item Calculer $Q_\text{mass}$, la DTLM et $Q_{KA}$.
    \item Calculer $\delta_\text{rel}$ et vérifier les drapeaux.
    \item Retourner \texttt{CondenserResult}.
\end{enumerate}

\subsection{Structure du code Python}
La classe \texttt{CondenserModel} est définie dans \texttt{src/app\_r718/modules/condenser/model.py}. La méthode \texttt{solve} prend en paramètres l'état d'entrée, le débit total, la température de condensation, les caractéristiques de l'échangeur ($K$, $A$) et les températures de l'air. Le résultat \texttt{CondenserResult} contient : \texttt{state\_out}, $P_\text{cond}$, $T_\text{sat}$, $Q_\text{mass}$, $Q_{KA}$, $\delta_\text{rel}$ et les drapeaux.
\begin{figure}[h]
    \centering
    \begin{tikzpicture}[node distance=0.6cm]

        % === CondenserModel ===
        \node (title1) [umltitle] {CondenserModel};
        \node (attr1) [umlsection, below=0cm of title1] {
            -- props : PropsService
        };
        \node (meth1) [umlsection, below=0cm of attr1] {
            + solve(state\_in, m\_dot, \\
            \quad T\_cond, K, A, \\
            \quad T\_air\_in, T\_air\_out, \\
            \quad subcool\_K)
        };

        % === CondenserResult ===
        \node (title2) [umldatatitle, right=3cm of title1] {$\ll$dataclass$\gg$ \\ CondenserResult};
        \node (attr2) [umlsection, below=0cm of title2] {
            + state\_out : ThermoState \\
            + P\_cond : float \\
            + T\_sat : float \\
            + Q\_mass : float \\
            + Q\_KA : float \\
            + delta\_relative : float \\
            + flags : Dict[str, bool]
        };

        % === PropsService ===
        \node (title3) [umlcoretitle, below=4.5cm of title1] {$\ll$singleton$\gg$ \\ PropsService};
        \node (attr3) [umlsection, below=0cm of title3] {
            + Psat\_T(T) \\
            + Tsat\_P(P)
        };

        % Flèches
        \draw[uml_returns] (meth1.east) -- ++(0.5,0) |- node[near start, above, font=\tiny\sffamily] {retourne} (attr2.west);
        \draw[uml_uses] (attr1.south) -- node[midway, left, font=\tiny\sffamily] {délègue à} (title3.north);

    \end{tikzpicture}
    \caption{Diagramme de classes du module Condenseur}
    \label{fig:uml_condenseur}
\end{figure}

\subsection{Flags et contrôles de cohérence associés}

\begin{table}[h]
    \centering
    \begin{tabular}{|l|p{9cm}|}
        \hline
        \textbf{Flag}                      & \textbf{Signification}                     \\ \hline
        \texttt{incomplete\_condensation}  & $h_\text{in} \leq h_\text{out}$            \\ \hline
        \texttt{negative\_heat\_rejection} & $Q_\text{mass} \leq 0$ ou $Q_{KA} \leq 0$  \\ \hline
        \texttt{invalid\_LMTD}             & $\Delta T_1 \leq 0$ ou $\Delta T_2 \leq 0$ \\ \hline
        \texttt{thermal\_mismatch}         & $\delta_\text{rel} > 5\%$                  \\ \hline
    \end{tabular}
    \caption{Drapeaux diagnostiques du module condenseur}
    \label{tab:flags_cond}
\end{table}

\subsection{Limites du modèle}

\begin{itemize}
    \item Le coefficient d'échange $K$ est imposé et non calculé par corrélation de condensation en film (Nusselt) ou de convection naturelle côté air.
    \item La désurchauffe de la vapeur en entrée (si $T_5 > T_\text{sat}$) n'est pas modélisée séparément.
    \item Le modèle est 0D (pas de discrétisation spatiale de l'échangeur).
\end{itemize}


% ======================================================================
% SECTION 5 : POMPE
% ======================================================================
\section{La Pompe}

\subsection{Rôle dans le système}

La pompe réalise la transformation $1 \rightarrow 7$. Elle comprime le liquide saturé issu du condenseur (état~1, $P_\text{cond}$) jusqu'à la pression du générateur ($P_\text{gen}$). C'est le seul composant du cycle qui consomme un travail mécanique externe.

\subsection{Modèle mathématique retenu}

Le modèle retenu est une \textbf{compression à rendement isentropique} $\eta_\text{is}$.

\subsection{Hypothèses}

\begin{enumerate}
    \item Compression adiabatique du liquide.
    \item Le rendement isentropique $\eta_\text{is}$ est un paramètre imposé ($0 < \eta_\text{is} \leq 1$, défaut : $\eta_\text{is} = 0{,}7$).
    \item Le fluide en entrée est en phase liquide (monophasique).
    \item Régime permanent.
\end{enumerate}

\subsection{Équations utilisées}

\textbf{État isentropique (compression idéale) :}
\begin{equation}
    (P_7^\text{is},\, s_7^\text{is}) = (P_\text{gen},\, s_1) \quad \Rightarrow \quad h_7^\text{is} = h_\text{PS}(P_\text{gen},\, s_1)
    \label{eq:pump_is}
\end{equation}

\textbf{Enthalpie de sortie réelle :}
\begin{equation}
    h_7 = h_1 + \frac{h_7^\text{is} - h_1}{\eta_\text{is}}
    \label{eq:pump_real}
\end{equation}

\textbf{Puissance consommée par la pompe :}
\begin{equation}
    W_\text{pompe} = \dot{m}_\text{pri} \cdot (h_7 - h_1)
    \label{eq:pump_power}
\end{equation}

\subsection{Algorithme d'implémentation}

\begin{enumerate}
    \item Vérifier $P_\text{out} > P_\text{in}$ (élévation de pression).
    \item Vérifier que l'entrée est monophasique ($x_1 = 0$ ou $x_1 = \texttt{None}$).
    \item Vérifier le risque de cavitation ($P_\text{in} < 1500$~Pa).
    \item Calculer l'état isentropique via \texttt{update\_from\_PS}$(P_\text{gen}, s_1)$.
    \item Calculer $h_7$ par l'équation~(\ref{eq:pump_real}).
    \item Construire l'état de sortie réel via \texttt{update\_from\_PH}$(P_\text{gen}, h_7)$.
    \item Calculer $W_\text{pompe}$ (éq.~\ref{eq:pump_power}).
    \item Retourner \texttt{PumpResult}.
\end{enumerate}

\subsection{Structure du code Python}

\begin{figure}[h]
    \centering
    \begin{tikzpicture}[node distance=0.6cm]

        % === PumpModel ===
        \node (title1) [umltitle] {PumpModel};
        \node (attr1) [umlsection, below=0cm of title1] {
            -- props : PropsService
        };
        \node (meth1) [umlsection, below=0cm of attr1] {
            + solve(state\_in, P\_out, \\
            \quad eta\_is, m\_dot)
        };

        % === PumpResult ===
        \node (title2) [umldatatitle, right=3cm of title1] {$\ll$dataclass$\gg$ \\ PumpResult};
        \node (attr2) [umlsection, below=0cm of title2] {
            + state\_out : ThermoState \\
            + state\_is : ThermoState \\
            + W\_pump : float \\
            + delta\_h : float \\
            + flags : Dict[str, bool]
        };

        % === ThermoState ===
        \node (title3) [umlcoretitle, below=2.5cm of title1] {ThermoState};
        \node (attr3) [umlsection, below=0cm of title3] {
            + update\_from\_PS(P, s) \\
            + update\_from\_PH(P, h)
        };

        % Flèches
        \draw[uml_returns] (meth1.east) -- ++(0.5,0) |- node[near start, above, font=\tiny\sffamily] {retourne} (attr2.west);
        \draw[uml_uses] (attr1.south) -- node[midway, left, font=\tiny\sffamily] {utilise} (title3.north);

    \end{tikzpicture}
    \caption{Diagramme de classes du module Pompe}
    \label{fig:uml_pompe}
\end{figure}
La classe \texttt{PumpModel} est définie dans \texttt{src/app\_r718/modules/pump/model.py}. La méthode \texttt{solve} prend en entrée l'état $\text{state}_\text{in}$, la pression de sortie $P_\text{out}$, le rendement isentropique $\eta_\text{is}$ et le débit massique $\dot{m}$. Le résultat \texttt{PumpResult} contient : \texttt{state\_out}, \texttt{state\_is} (état idéal), $W_\text{pompe}$, $\Delta h$ et les drapeaux.

\subsection{Flags et contrôles de cohérence associés}

\begin{table}[h]
    \centering
    \begin{tabular}{|l|p{9cm}|}
        \hline
        \textbf{Flag}                    & \textbf{Signification}                       \\ \hline
        \texttt{invalid\_pressure\_rise} & $P_\text{out} \leq P_\text{in}$              \\ \hline
        \texttt{two\_phase\_inlet}       & Présence de vapeur en entrée ($0 < x_1 < 1$) \\ \hline
        \texttt{cavitation\_risk}        & $P_\text{in} < 1500$~Pa (NPSH insuffisant)   \\ \hline
        \texttt{unphysical\_state}       & Échec du calcul \textit{CoolProp}            \\ \hline
    \end{tabular}
    \caption{Drapeaux diagnostiques du module pompe}
    \label{tab:flags_pump}
\end{table}

\subsection{Limites du modèle}

\begin{itemize}
    \item Le rendement isentropique est constant, indépendant du point de fonctionnement.
    \item Le critère de cavitation est simplifié (seuil de pression fixe à 1500~Pa) au lieu d'un calcul de NPSH complet.
    \item Les pertes mécaniques (paliers, joints) ne sont pas modélisées.
    \item Le modèle n'intègre pas de courbe caractéristique $H$-$Q$ de la pompe.
\end{itemize}


% ======================================================================
% SECTION 6 : CHAUDIÈRE SOLAIRE (GÉNÉRATEUR)
% ======================================================================
\section{La Chaudière Solaire (Générateur)}

\subsection{Rôle dans le système}

Le générateur réalise la transformation $7 \rightarrow 8$. Il reçoit le liquide comprimé issu de la pompe (état~7) et le chauffe puis le vaporise à pression constante ($P_\text{gen}$) jusqu'à l'état de vapeur saturée ($x_8 = 1$), éventuellement avec une surchauffe $\Delta T_\text{sh}$. La source de chaleur est un fluide caloporteur chaud provenant d'un capteur solaire.

\subsection{Modèle mathématique retenu}

Le modèle couple un \textbf{bilan énergétique massique} à un \textbf{modèle d'échangeur de chaleur} (DTLM) entre le fluide caloporteur (HTF) et le réfrigérant R718.

\subsection{Hypothèses}

\begin{enumerate}
    \item Chauffage et vaporisation à pression constante $P_\text{gen} = P_\text{sat}(T_\text{gen})$.
    \item Régime permanent.
    \item L'état de sortie est imposé : vapeur saturée ($x_8 = 1$) ou surchauffée ($T_8 = T_\text{sat} + \Delta T_\text{sh}$).
    \item La majeure partie du transfert de chaleur a lieu durant le changement de phase ; la DTLM est donc calculée avec $T_\text{sat}$ comme température effective du réfrigérant (simplification 0D).
    \item Le fluide caloporteur (HTF) évolue de $T_\text{HTF,in}$ à $T_\text{HTF,out}$.
\end{enumerate}

\subsection{Équations utilisées}

\textbf{Pression du générateur :}
\begin{equation}
    P_\text{gen} = P_\text{sat}(T_\text{gen})
    \label{eq:gen_P}
\end{equation}

\textbf{Bilan énergétique massique :}
\begin{equation}
    Q_\text{mass} = \dot{m}_\text{pri} \cdot (h_8 - h_7)
    \label{eq:gen_Qmass}
\end{equation}

\textbf{Modèle d'échangeur de chaleur :}
\begin{equation}
    Q_{KA} = K \cdot A \cdot \Delta T_\text{lm}
    \label{eq:gen_QKA}
\end{equation}

\textbf{DTLM :}
\begin{align}
    \Delta T_1 & = T_\text{HTF,in} - T_\text{sat}  \\
    \Delta T_2 & = T_\text{HTF,out} - T_\text{sat}
\end{align}

L'écart relatif $\delta_\text{rel}$ est calculé comme :
\begin{equation}
    \delta_\text{rel} = \frac{|Q_\text{mass} - Q_{KA}|}{\max(|Q_\text{mass}|,\, |Q_{KA}|,\, \varepsilon)}
    \label{eq:gen_delta}
\end{equation}
avec $\varepsilon = 10^{-6}$. Le seuil de mismatch thermique est fixé à 20\% dans l'implémentation, ce qui permet de signaler un sous-dimensionnement de l'échangeur sans pour autant invalider la simulation.

\subsection{Algorithme d'implémentation}

\begin{enumerate}
    \item Calculer $P_\text{gen} = P_\text{sat}(T_\text{gen})$ via \texttt{PropsService.Psat\_T}.
    \item Construire l'état de sortie $\text{state}_8$ :
          \begin{itemize}
              \item si pas de surchauffe : \texttt{update\_from\_PX}$(P_\text{gen}, 1{,}0)$,
              \item sinon : \texttt{update\_from\_PT}$(P_\text{gen}, T_\text{sat} + \Delta T_\text{sh})$.
          \end{itemize}
    \item Vérifier $h_8 > h_7$ (apport de chaleur positif).
    \item Calculer $Q_\text{mass}$ (éq.~\ref{eq:gen_Qmass}).
    \item Calculer la DTLM et $Q_{KA}$ (éq.~\ref{eq:gen_QKA}).
    \item Calculer $\delta_\text{rel}$ (éq.~\ref{eq:gen_delta}).
    \item Retourner \texttt{GeneratorResult}.
\end{enumerate}

\subsection{Structure du code Python}

\begin{figure}[h]
    \centering
    \begin{tikzpicture}[node distance=0.6cm]

        % === GeneratorModel ===
        \node (title1) [umltitle] {GeneratorModel};
        \node (attr1) [umlsection, below=0cm of title1] {
            -- props : PropsService
        };
        \node (meth1) [umlsection, below=0cm of attr1] {
            + solve(state\_in, m\_dot, \\
            \quad T\_gen\_target, K, A, \\
            \quad T\_htf\_in, T\_htf\_out, \\
            \quad superheat\_K)
        };

        % === GeneratorResult ===
        \node (title2) [umldatatitle, right=3cm of title1] {$\ll$dataclass$\gg$ \\ GeneratorResult};
        \node (attr2) [umlsection, below=0cm of title2] {
            + state\_out : ThermoState \\
            + P\_gen : float \\
            + Q\_mass : float \\
            + Q\_KA : float \\
            + delta\_relative : float \\
            + delta\_T\_lm : float \\
            + flags : Dict[str, bool]
        };

        % === PropsService ===
        \node (title3) [umlcoretitle, below=5.5cm of title1] {$\ll$singleton$\gg$ \\ PropsService};
        \node (attr3) [umlsection, below=0cm of title3] {
            + Psat\_T(T) \\
            + h\_PX(P, x) \\
            + s\_PX(P, x)
        };

        % Flèches
        \draw[uml_returns] (meth1.east) -- ++(0.5,0) |- node[near start, above, font=\tiny\sffamily] {retourne} (attr2.west);
        \draw[uml_uses] (attr1.south) -- node[midway, left, font=\tiny\sffamily] {délègue à} (title3.north);

    \end{tikzpicture}
    \caption{Diagramme de classes du module Chaudière Solaire (Générateur)}
    \label{fig:uml_generateur}
\end{figure}
La classe \texttt{GeneratorModel} est définie dans \texttt{src/app\_r718/modules/generator/model.py}. La méthode \texttt{solve} prend en paramètres l'état d'entrée (état~7), le débit primaire $\dot{m}_\text{pri}$, la température cible du générateur $T_\text{gen}$, les paramètres de l'échangeur ($K$, $A$), les températures du HTF et la surchauffe optionnelle. Le résultat \texttt{GeneratorResult} contient : \texttt{state\_out}, $P_\text{gen}$, $Q_\text{mass}$, $Q_{KA}$, $\delta_\text{rel}$, $\Delta T_\text{lm}$ et les drapeaux.

\subsection{Flags et contrôles de cohérence associés}

\begin{table}[h]
    \centering
    \begin{tabular}{|l|p{9cm}|}
        \hline
        \textbf{Flag}                  & \textbf{Signification}                                  \\ \hline
        \texttt{invalid\_LMTD}         & $\Delta T_1 \leq 0$ ou $\Delta T_2 \leq 0$              \\ \hline
        \texttt{negative\_heat\_input} & $Q_\text{mass} < 0$ ou $h_8 \leq h_7$                   \\ \hline
        \texttt{thermal\_mismatch}     & $\delta_\text{rel} > 20\%$ : échangeur sous-dimensionné \\ \hline
        \texttt{two\_phase\_outlet}    & État de sortie inattendu en zone diphasique             \\ \hline
    \end{tabular}
    \caption{Drapeaux diagnostiques du module générateur}
    \label{tab:flags_gen}
\end{table}

\subsection{Limites du modèle}

\begin{itemize}
    \item L'état de sortie est imposé (vapeur saturée ou surchauffée) et non résolu par un bilan couplé. Le flag \texttt{thermal\_mismatch} indique si l'échangeur peut ou non fournir la puissance requise.
    \item La modélisation solaire est simplifiée : pas de rayonnement direct $G_b$, pas de rendement optique, pas de pertes radiatives. Le chauffage est transmis par un HTF intermédiaire.
    \item Le modèle ne discrétise pas les zones de préchauffage, d'ébullition et de surchauffe dans l'échangeur.
    \item Le coefficient $K$ est un paramètre fixe et non calculé par corrélation.
\end{itemize}


% ======================================================================
% CONCLUSION DU CHAPITRE
% ======================================================================
\section*{Résumé du chapitre}

Ce chapitre a présenté la modélisation mathématique et l'implémentation numérique des six composants du cycle frigorifique à éjecteur R718. Chaque composant a été décrit selon une structure uniforme : rôle, modèle retenu, hypothèses, équations, algorithme, structure du code et contrôles de cohérence.

Le \textbf{détendeur} réalise une détente isenthalpique ($h_2 = h_1$). L'\textbf{évaporateur} et le \textbf{condenseur} couplent un bilan énergétique massique à un modèle d'échangeur via la DTLM.  La \textbf{pompe} utilise un modèle à rendement isentropique avec détection de cavitation. La \textbf{chaudière solaire} impose un état de sortie et vérifie le dimensionnement de l'échangeur.

L'\textbf{éjecteur} est le composant le plus complexe. Le modèle V2 implémenté inclut le calcul des nombres de Mach, la détection du blocage sonique, les bilans de quantité de mouvement et d'énergie dans la chambre de mélange, la modélisation de l'onde de choc normale via les relations de Rankine-Hugoniot et le calcul du saut d'entropie. Le rapport d'entraînement $\mu$ est déterminé par optimisation numérique.

L'ensemble des modules s'appuie sur le service \texttt{PropsService} (CoolProp) pour garantir la précision des propriétés thermodynamiques et sur les objets \texttt{ThermoState} pour assurer la cohérence des échanges entre composants. Les drapeaux diagnostiques implémentés dans chaque module constituent un filet de sécurité qui permet de détecter automatiquement les incohérences physiques et les limites de validité des calculs.

\chapter{Analyse des performances}

Ce chapitre présente l’analyse des performances du cycle frigorifique à éjecteur utilisant le fluide R718, à partir du simulateur développé avec Python.
L’objectif est d’évaluer :

\begin{itemize}
    \item la cohérence thermodynamique du modèle,
    \item la stabilité du fonctionnement sous vide profond,
    \item le coefficient de performance (COP),
    \item la sensibilité du système aux paramètres thermiques,
    \item les limitations structurelles du cycle.
\end{itemize}

L’analyse repose sur le cas nominal issu du dimensionnement inverse, correspondant à une puissance frigorifique cible :

\[
    \dot{Q}_{evap} = 12~\mathrm{kW}
\]

avec :

\[
    T_{evap} = 10^\circ C, \quad
    T_{cond} = 35^\circ C, \quad
    T_{gen} = 100^\circ C
\]

Le solveur global a convergé en deux itérations avec une erreur relative inférieure à $10^{-4}$, sans activation de flag d'incohérence.
\newpage
%% Ajouter ici le schéma du cycle avec numérotation (1 à 8)

% ------------------------------------------------------------
\section{Cas nominal}

\subsection{États thermodynamiques}

Les états calculés sont résumés ci-dessous :

\begin{itemize}
    \item État 1 : sortie condenseur
          \(P_1 = 5.629~\mathrm{kPa}, \quad T_1 = 35^\circ C, \quad x_1 = 0\)

    \item État 2 : sortie détendeur
          \(P_2 = 1.228~\mathrm{kPa}, \quad T_2 = 10^\circ C, \quad x_2 = 0.04223\)

    \item État 3 : sortie évaporateur
          \(P_3 = 1.228~\mathrm{kPa}, \quad x_3 = 1\)

    \item État 4 : sortie mélange éjecteur
          \(P_4 = 2.828~\mathrm{kPa}\)

    \item État 5 : sortie diffuseur
          \(P_5 = 5.629~\mathrm{kPa}\)

    \item État 7 : sortie pompe
          \(P_7 = 101.418~\mathrm{kPa}\)

    \item État 8 : sortie chaudière
          \(P_8 = 101.418~\mathrm{kPa}, \quad T_8 = 100^\circ C, \quad x_8 = 1\)
\end{itemize}

On observe que le cycle fonctionne intégralement sous vide côté évaporateur, avec :

\[
    P_{evap} \approx 1.228~\mathrm{kPa}
\]

soit près de 800 fois inférieur à la pression atmosphérique.

% ------------------------------------------------------------
\subsection{Débits massiques}

Les débits obtenus sont :

\[
    \dot{m}_{pri} = 0.004423~\mathrm{kg/s}
\]

\[
    \dot{m}_{sec} = 0.005058~\mathrm{kg/s}
\]

\[
    \dot{m}_{tot} = 0.009481~\mathrm{kg/s}
\]

Le rapport d’entraînement est :

\[
    \mu = \frac{\dot{m}_{sec}}{\dot{m}_{pri}} = 1.1435
\]

Cette valeur indique un fonctionnement stable en régime supersonique avec choc interne dans la chambre de mélange.

% ------------------------------------------------------------
\subsection{Bilans énergétiques}

Les puissances calculées sont :

\[
    \dot{Q}_{evap} = 12~\mathrm{kW}
\]

\[
    \dot{Q}_{gen} = 11.185~\mathrm{kW}
\]

\[
    \dot{Q}_{cond} = 23.954~\mathrm{kW}
\]

Le bilan énergétique global vérifie :

\[
    \dot{Q}_{cond} = \dot{Q}_{evap} + \dot{Q}_{gen} + \dot{W}_p
\]

avec :

\[
    \dot{W}_p = 0.000609~\mathrm{kW}
\]

confirmant le caractère négligeable du travail de pompe.

% ------------------------------------------------------------
\subsection{Coefficient de performance}

Le COP thermique du cycle est :

\[
    COP = \frac{\dot{Q}_{evap}}{\dot{Q}_{gen}} = 1.0728
\]

Cette valeur est élevée pour un cycle à éjection vapeur à 100°C, ce qui s’explique par :

\begin{itemize}
    \item le fort effet frigorifique spécifique du R718,
    \item la bonne récupération de pression dans le diffuseur,
    \item l’optimisation du rapport d’entraînement.
\end{itemize}

%% Ajouter ici le diagramme P-h nominal
%% Ajouter ici le diagramme T-s nominal

% ------------------------------------------------------------
\section{Analyse de sensibilité}

\subsection{Influence de la température de condensation}

Une augmentation de \(T_{cond}\) entraîne :

\begin{itemize}
    \item une augmentation de \(P_{cond}\),
    \item un déplacement du choc dans l’éjecteur,
    \item une diminution du rapport d’entraînement \(\mu\),
    \item une réduction du COP.
\end{itemize}

Le cycle devient instable lorsque :

\[
    P_{cond} > P_{cond,crit}
\]

correspondant au décrochage de l’éjecteur.

%% Ajouter ici le graphique COP = f(T_cond)

% ------------------------------------------------------------
\subsection{Influence de la température générateur}

L’augmentation de \(T_{gen}\) :

\begin{itemize}
    \item augmente la pression motrice,
    \item améliore l’aspiration secondaire,
    \item accroît \(\mu\),
    \item mais augmente également les pertes thermiques solaires.
\end{itemize}

Un optimum thermodynamique existe entre performance éjecteur et rendement solaire.

%% Ajouter ici le graphique mu = f(T_gen)

% ------------------------------------------------------------
\section{Analyse exergétique du système}

L’exergie détruite dans le cycle provient principalement de :

\begin{itemize}
    \item l’irréversibilité du choc dans l’éjecteur (\(\Delta s > 0\)),
    \item le laminage isoenthalpique du détendeur,
    \item les transferts thermiques à faible gradient.
\end{itemize}

La production d’entropie mesurée au niveau du choc est :

\[
    \Delta s = 0.8385~\mathrm{kJ/kg.K}
\]

indiquant une dissipation modérée compatible avec un régime stable.

% ------------------------------------------------------------
\section{Limites du modèle}

Le modèle actuel présente les limitations suivantes :

\begin{itemize}
    \item hypothèse de détente strictement isoenthalpique,
    \item condensation modélisée en régime laminaire,
    \item absence de modélisation transitoire,
    \item absence de non-condensables,
    \item pertes de charge simplifiées.
\end{itemize}

% ------------------------------------------------------------
\section{Recommandations conceptuelles}

Les résultats obtenus suggèrent :

\begin{itemize}
    \item un dimensionnement précis du condenseur pour limiter \(P_{cond}\),
    \item une maîtrise rigoureuse des pertes de charge sous vide,
    \item une optimisation conjointe \(T_{gen}\)/surface solaire,
    \item une étude future incluant exergie complète et dynamique transitoire.
\end{itemize}

% ------------------------------------------------------------
\section*{Résumé du chapitre}

L’analyse des performances confirme la viabilité thermodynamique du cycle R718 à éjecteur sous vide profond pour une puissance frigorifique de 12 kW.

Le COP obtenu (1.0728) et le rapport d’entraînement (1.1435) traduisent un fonctionnement stable et cohérent avec la modélisation implémentée.

La stabilité du cycle demeure fortement conditionnée par la pression de condensation et par la maîtrise des irréversibilités dans l’éjecteur.

Le chapitre suivant discutera les perspectives d’optimisation et d’industrialisation du système.

% ------------------------
% Annexes
% ------------------------
\appendix
\part{Annexes}

% =========================
% Fichier : appendices/annexeA.tex
% ANNEXE A — Fondements théoriques de l’écoulement compressible dans l’éjecteur
% Convention : vitesse d'écoulement = c (pas V)
% =========================

\chapter{Fondements théoriques de l’écoulement compressible dans l’éjecteur}
\label{ann:compressible}

\section{Hypothèses fondamentales}
\label{ann:compressible:hypotheses}

Dans cette annexe, on présente les bases analytiques utiles à la compréhension de l’écoulement compressible dans l’éjecteur. Sauf mention contraire, les hypothèses suivantes sont adoptées :

\begin{itemize}
    \item écoulement stationnaire ;
    \item écoulement quasi-unidimensionnel (quasi-1D) ;
    \item fluide compressible ;
    \item transformation adiabatique (hors pertes internes et chocs) ;
    \item absence de travail mécanique externe sur le volume de contrôle ;
    \item propriétés \emph{gaz parfait} utilisées comme cadre théorique, avec adaptation au fluide réel via propriétés thermodynamiques (p.\ ex. CoolProp) lorsque nécessaire \cite{Anderson2016,Bell2014}.
\end{itemize}

\section{Équations fondamentales (volume de contrôle 1D)}
\label{ann:compressible:eq_fond}

\subsection{Conservation de la masse}
\label{ann:compressible:mass}

Pour un écoulement 1D dans une conduite de section \(A\), la conservation de la masse s’écrit :
\begin{equation}
    \dot{m} = \rho\,c\,A,
    \label{eq:ann_mdot}
\end{equation}
où \(\dot{m}\) est le débit massique, \(\rho\) la masse volumique et \(c\) la vitesse moyenne d’écoulement.

\subsection{Conservation de la quantité de mouvement}
\label{ann:compressible:momentum}

En régime stationnaire 1D, en négligeant les effets de gravité et en regroupant les effets visqueux dans une force équivalente \(F_{\mathrm{pertes}}\), on peut écrire sous forme intégrale simplifiée :
\begin{equation}
    P\,A + \rho\,c^{2}A = \text{constante} \quad (\text{à pertes négligeables}),
    \label{eq:ann_momentum_simpl}
\end{equation}
et, de manière plus générale pour un volume de contrôle délimité par une entrée (1) et une sortie (2) :
\begin{equation}
    \dot{m}\,(c_2 - c_1) = (P_1A_1 - P_2A_2) - F_{\mathrm{pertes}}.
    \label{eq:ann_momentum_general}
\end{equation}

\subsection{Conservation de l’énergie (enthalpie totale)}
\label{ann:compressible:energy}

Sous hypothèse adiabatique et sans travail de paroi, l’énergie spécifique totale (ou enthalpie de stagnation) est conservée :
\begin{equation}
    h + \frac{c^2}{2} = h_0 = \text{constante}.
    \label{eq:ann_energy_total}
\end{equation}
Cette relation est fondamentale pour décrire l’accélération dans une tuyère : une baisse d’enthalpie statique \(h\) se traduit par une augmentation de la vitesse \(c\).

\section{Écoulement isentropique dans une tuyère}
\label{ann:compressible:isentropic}

\subsection{Définition et relations de base}
\label{ann:compressible:isentropic_def}

Dans une tuyère idéale, l’écoulement est souvent assimilé à une transformation isentropique :
\begin{equation}
    ds = 0.
    \label{eq:ann_ds0}
\end{equation}

Pour un gaz parfait, cela implique :
\begin{equation}
    \frac{P}{\rho^\gamma} = \text{constante},
    \label{eq:ann_isentropic_Prho}
\end{equation}
où \(\gamma\) est le rapport des chaleurs spécifiques.

On définit le nombre de Mach :
\begin{equation}
    M = \frac{c}{a},
    \label{eq:ann_mach}
\end{equation}
où \(a\) est la célérité du son.

\subsection{Relations stagnation–statique (gaz parfait)}
\label{ann:compressible:stagnation}

Pour un gaz parfait, les relations reliant grandeurs statiques et stagnation sont \cite{Anderson2016} :
\begin{align}
    \frac{T_0}{T}       & = 1 + \frac{\gamma-1}{2}M^2,
    \label{eq:ann_T0_T}
    \\
    \frac{P_0}{P}       & = \left(1 + \frac{\gamma-1}{2}M^2\right)^{\frac{\gamma}{\gamma-1}},
    \label{eq:ann_P0_P}
    \\
    \frac{\rho_0}{\rho} & = \left(1 + \frac{\gamma-1}{2}M^2\right)^{\frac{1}{\gamma-1}}.
    \label{eq:ann_rho0_rho}
\end{align}

\section{Condition d’étranglement (Mach = 1) et relation aire–Mach}
\label{ann:compressible:choking}

\subsection{Critère d’étranglement}
\label{ann:compressible:choking_crit}

L’étranglement (\emph{choking}) correspond à l’atteinte du régime sonique \(M=1\) au col de la tuyère. À partir de ce point, le débit massique devient peu sensible à la pression aval. La relation différentielle issue de la conservation de masse et de l’équation d’énergie mène à la relation (gaz parfait) :
\begin{equation}
    \frac{dA}{A} = (M^2 - 1)\,\frac{dc}{c}.
    \label{eq:ann_dA_relation}
\end{equation}
Ainsi :
\begin{itemize}
    \item si \(M<1\) (subsonique), une section convergente (\(dA<0\)) accélère l’écoulement ;
    \item si \(M>1\) (supersonique), une section divergente (\(dA>0\)) accélère l’écoulement.
\end{itemize}

\subsection{Relation aire–Mach}
\label{ann:compressible:area_mach}

Pour un écoulement isentropique de gaz parfait, la relation aire–Mach s’écrit \cite{Anderson2016} :
\begin{equation}
    \frac{A}{A^\star} =
    \frac{1}{M}
    \left[
        \frac{2}{\gamma+1}
        \left(1+\frac{\gamma-1}{2}M^2\right)
        \right]^{\frac{\gamma+1}{2(\gamma-1)}},
    \label{eq:ann_area_mach}
\end{equation}
où \(A^\star\) est la section critique associée à \(M=1\).

\section{Choc normal : équations de Rankine–Hugoniot}
\label{ann:compressible:shock}

\subsection{Conservation à travers le choc}
\label{ann:compressible:shock_cons}

On considère un choc normal stationnaire. Pour éviter toute confusion avec la numérotation des états du cycle (1 à 8), on note ici les grandeurs amont/aval du choc par \(\mathrm{am}\) et \(\mathrm{av}\). Les équations de conservation s’écrivent :

\paragraph{Conservation de la masse}
\begin{equation}
    \rho_{\mathrm{am}}\,c_{\mathrm{am}} = \rho_{\mathrm{av}}\,c_{\mathrm{av}}.
    \label{eq:ann_shock_mass}
\end{equation}

\paragraph{Conservation de la quantité de mouvement}
\begin{equation}
    P_{\mathrm{am}} + \rho_{\mathrm{am}}c_{\mathrm{am}}^2
    =
    P_{\mathrm{av}} + \rho_{\mathrm{av}}c_{\mathrm{av}}^2.
    \label{eq:ann_shock_momentum}
\end{equation}

\paragraph{Conservation de l’énergie}
\begin{equation}
    h_{\mathrm{am}} + \frac{c_{\mathrm{am}}^2}{2}
    =
    h_{\mathrm{av}} + \frac{c_{\mathrm{av}}^2}{2}.
    \label{eq:ann_shock_energy}
\end{equation}

Ces trois équations définissent la discontinuité associée au choc normal.

\subsection{Relation de Hugoniot}
\label{ann:compressible:hugoniot}

En combinant les équations de conservation, on obtient la relation de Hugoniot, qui relie l’état amont et l’état aval compatibles avec un choc adiabatique \cite{Anderson2016} :
\begin{equation}
    h_{\mathrm{av}} - h_{\mathrm{am}}
    =
    \frac{1}{2}\left(P_{\mathrm{av}} - P_{\mathrm{am}}\right)\left(v_{\mathrm{am}} + v_{\mathrm{av}}\right),
    \label{eq:ann_hugoniot}
\end{equation}
où \(v = 1/\rho\) est le volume spécifique.

Cette expression met en évidence que le choc est adiabatique mais généralement non isentropique.

\section{Variation d’entropie à travers le choc}
\label{ann:compressible:entropy}

Un choc normal est une transformation irréversible : l’entropie augmente :
\begin{equation}
    s_{\mathrm{av}} > s_{\mathrm{am}}.
    \label{eq:ann_entropy_increase}
\end{equation}

Cette augmentation d’entropie est associée à une perte d’énergie disponible (exergie) et contribue à limiter la récupération de pression dans le diffuseur, ce qui pénalise la performance globale (baisse de \(\mu\) et du \(COP\)) \cite{Anderson2016,Eames1995}.

\section{Relations Mach amont/aval (gaz parfait)}
\label{ann:compressible:mach_relations}

Dans le cadre gaz parfait, les relations usuelles du choc normal donnent \cite{Anderson2016} :
\begin{align}
    \frac{P_{\mathrm{av}}}{P_{\mathrm{am}}}
     & =
    1+\frac{2\gamma}{\gamma+1}\left(M_{\mathrm{am}}^2-1\right),
    \label{eq:ann_P_ratio_shock}
    \\
    M_{\mathrm{av}}^2
     & =
    \frac{1+\frac{\gamma-1}{2}M_{\mathrm{am}}^2}
    {\gamma M_{\mathrm{am}}^2-\frac{\gamma-1}{2}}.
    \label{eq:ann_M2_shock}
\end{align}

Ces relations montrent que si \(M_{\mathrm{am}}>1\), alors \(M_{\mathrm{av}}<1\) : le choc transforme un régime supersonique en régime subsonique.

\section{Adaptation au cas du R718 (fluide réel sous vide)}
\label{ann:compressible:r718}

La vapeur d’eau sous vide profond et proche de la saturation n’est pas strictement représentable par un gaz parfait, et \(\gamma\) n’est ni constant ni toujours pertinent. Dans la simulation du présent travail, l’approche retenue est dite \emph{hybride} :

\begin{itemize}
    \item la structure des équations (conservation de masse, quantité de mouvement, énergie) est conservée (Eqs.~\ref{eq:ann_shock_mass}--\ref{eq:ann_shock_energy}) ;
    \item les propriétés thermodynamiques sont calculées via une bibliothèque de référence (p.\ ex. CoolProp) : \(h(P,T)\), \(s(P,T)\), \(\rho(P,T)\), etc. \cite{Bell2014}.
\end{itemize}

Cette stratégie vise à garantir :
\begin{itemize}
    \item le respect des lois fondamentales ;
    \item la cohérence thermodynamique en régime réel ;
    \item la compatibilité avec les diagrammes \(P\text{-}h\) et \(T\text{-}s\) utilisés en validation qualitative.
\end{itemize}

\section{Portée et limites}
\label{ann:compressible:limits}

Cette annexe fournit les bases analytiques et la justification des hypothèses utilisées dans un modèle 1D. Elle ne remplace pas :
\begin{itemize}
    \item une simulation CFD 2D/3D détaillée ;
    \item une validation expérimentale sur banc d’essai.
\end{itemize}

Néanmoins, elle constitue un socle mathématique indispensable pour comprendre l’influence du régime supersonique, de l’étranglement et des chocs sur la stabilité et la performance d’un éjecteur intégré dans un cycle frigorifique à R718.


\newpage
\chapter{Outils et choix technologiques du simulateur}

Le développement du simulateur thermodynamique de la machine frigorifique à éjecteur R718 ne repose pas uniquement sur des modèles physiques rigoureux, mais également sur une architecture logicielle cohérente et des outils adaptés aux exigences scientifiques.

Cet appendice présente les principaux outils et concepts technologiques ayant structuré l’implémentation, ainsi que la justification de leur choix.

\section{Architecture logicielle : Design Pattern MVC}

\subsection{Principe}

Le modèle \textbf{MVC (Model–View–Controller)} est un patron d’architecture logicielle séparant :

\begin{itemize}
    \item \textbf{Model} : logique métier et modèles thermodynamiques,
    \item \textbf{View} : interface graphique utilisateur,
    \item \textbf{Controller} : gestion des interactions et orchestration des calculs.
\end{itemize}

\subsection{Application au simulateur}

Dans le cadre du simulateur :

\begin{itemize}
    \item Les composants thermodynamiques (détendeur, éjecteur, condenseur, etc.) constituent le \textbf{Model}.
    \item L’interface graphique développée avec Tkinter constitue la \textbf{View}.
    \item Le module principal d’exécution et de gestion des paramètres joue le rôle de \textbf{Controller}.
\end{itemize}

\subsection{Justification du choix}

Ce choix permet :

\begin{itemize}
    \item Une séparation claire entre physique et interface.
    \item Une meilleure maintenabilité.
    \item Une évolutivité vers une interface Web ou industrielle.
    \item Une traçabilité scientifique des modèles indépendamment de l’UI.
\end{itemize}

L’adoption d’une architecture MVC renforce la robustesse du simulateur et facilite son exploitation future comme outil d’aide à la décision.

\section{Principe ACID appliqué à la cohérence des calculs}

\subsection{Rappel du principe ACID}

Le principe ACID (Atomicité, Cohérence, Isolation, Durabilité), issu des systèmes transactionnels, garantit la fiabilité des opérations.

\begin{itemize}
    \item \textbf{Atomicité} : une opération est exécutée entièrement ou annulée.
    \item \textbf{Cohérence} : l’état final respecte les règles du système.
    \item \textbf{Isolation} : les opérations intermédiaires n’altèrent pas l’intégrité globale.
    \item \textbf{Durabilité} : les résultats validés sont conservés.
\end{itemize}

\subsection{Application au simulateur thermodynamique}

Bien que le simulateur ne soit pas une base de données, ces principes ont été transposés :

\begin{itemize}
    \item Atomicité : une simulation converge entièrement ou retourne un flag d’échec.
    \item Cohérence : validation via flags thermodynamiques (mismatch, pression, régime).
    \item Isolation : chaque composant est calculé indépendamment avant couplage.
    \item Durabilité : résultats stockés sous forme de structures persistantes (JSON).
\end{itemize}

Cette approche améliore la fiabilité numérique et évite les incohérences physiques.

\section{Bibliothèque thermophysique : CoolProp}

\subsection{Présentation}

CoolProp est une bibliothèque open-source de propriétés thermodynamiques permettant le calcul précis de :

\begin{itemize}
    \item enthalpie,
    \item entropie,
    \item densité,
    \item pression de saturation,
    \item chaleur latente,
    \item propriétés de transport.
\end{itemize}

\subsection{Justification du choix}

Le fluide R718 (eau) fonctionne sous vide profond, dans des conditions où :

\begin{itemize}
    \item l’approximation gaz parfait est invalide,
    \item les variations de propriétés sont fortement non linéaires,
    \item la précision thermodynamique est critique.
\end{itemize}

CoolProp permet :

\begin{itemize}
    \item l’accès à des équations d’état multiparamètres,
    \item une cohérence thermodynamique stricte,
    \item une compatibilité directe avec Python.
\end{itemize}

Son intégration garantit la validité physique des résultats.

\section{Visualisation scientifique : Matplotlib}

\subsection{Rôle}

La bibliothèque Matplotlib est utilisée pour :

\begin{itemize}
    \item tracer les diagrammes $P$–$h$,
    \item tracer les diagrammes $T$–$s$,
    \item visualiser les évolutions d’états thermodynamiques,
    \item analyser les sensibilités paramétriques.
\end{itemize}

\subsection{Justification}

Les diagrammes thermodynamiques constituent un outil fondamental d’analyse énergétique.
Matplotlib offre :

\begin{itemize}
    \item un contrôle précis des axes,
    \item une intégration native avec NumPy,
    \item une exportation haute résolution compatible avec \LaTeX.
\end{itemize}

Cela permet une validation visuelle du cycle simulé.

\section{Interface graphique : Tkinter}

\subsection{Présentation}

Tkinter est la bibliothèque standard d’interface graphique de Python.

Elle permet :

\begin{itemize}
    \item la création de tableaux de bord dynamiques,
    \item l’ajustement interactif des paramètres,
    \item le déclenchement des simulations,
    \item l’affichage en temps réel des résultats.
\end{itemize}

\subsection{Justification du choix}

Le choix de Tkinter repose sur :

\begin{itemize}
    \item sa simplicité d’intégration,
    \item sa légèreté,
    \item sa compatibilité multiplateforme,
    \item l’absence de dépendances lourdes.
\end{itemize}

Il constitue une solution adaptée pour un prototype scientifique et pédagogique.

\section{Synthèse}

Les choix technologiques réalisés dans ce projet ne relèvent pas d’un simple confort de développement, mais d’une volonté d’assurer :

\begin{itemize}
    \item rigueur scientifique,
    \item cohérence thermodynamique,
    \item stabilité numérique,
    \item traçabilité des résultats,
    \item évolutivité du simulateur.
\end{itemize}

L’association d’une architecture MVC, d’une bibliothèque thermophysique de référence (CoolProp), d’outils de visualisation scientifique (Matplotlib) et d’une interface interactive (Tkinter) constitue un socle robuste pour le développement d’un simulateur thermodynamique fiable et extensible.

\newpage
% Trigger bibliography processing
\cite{cengel2015thermodynamics}

% ------------------------
% Bibliographie
% ------------------------
\nocite{*}  % Affiche toutes les références même sans citation
\bibliography{references}

\end{document}
