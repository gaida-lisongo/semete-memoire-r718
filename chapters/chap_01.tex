\chapter{Introduction}

This chapter introduces the context and motivation of the work. It provides a general overview of the problem being addressed, explains its relevance within the broader field of study, and outlines the objectives pursued in this document. Additionally, the structure of the work is briefly described to guide the reader through the subsequent chapters.

\section{Problem Statement and Motivation}

The problem considered in this work arises in the context of \emph{[area or application domain]}. Recent developments in this field have highlighted the need for efficient and reliable approaches to address \emph{[specific challenge or limitation]}. Understanding and solving this problem is important due to its impact on \emph{[theoretical relevance, practical applications, or both]}. This work is motivated by the need to explore methods that can improve performance, accuracy, or robustness in this setting.

\section{Objectives and Organization of the Document}

The main objective of this work is to study and develop \emph{[methods, models, or techniques]} to address the problem described above. Specific goals include \emph{[key objectives or tasks]}.  
The remainder of this document is organized as follows: Chapter~2 presents \emph{[related work or theoretical background]}; Chapter~3 describes \emph{[methodology or model]}; Chapter~4 discusses the results obtained; and Chapter~5 concludes the work and outlines possible directions for future research. Bibliography usage example \cite{ho2020denoising}.





