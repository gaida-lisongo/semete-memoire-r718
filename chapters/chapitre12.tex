\chapter{Analyse des performances}

Ce chapitre présente l’analyse des performances du cycle frigorifique à éjecteur utilisant le fluide R718, à partir du simulateur développé avec Python.
L’objectif est d’évaluer :

\begin{itemize}
    \item la cohérence thermodynamique du modèle,
    \item la stabilité du fonctionnement sous vide profond,
    \item le coefficient de performance (COP),
    \item la sensibilité du système aux paramètres thermiques,
    \item les limitations structurelles du cycle.
\end{itemize}

L’analyse repose sur le cas nominal issu du dimensionnement inverse, correspondant à une puissance frigorifique cible :

\[
    \dot{Q}_{evap} = 12~\mathrm{kW}
\]

avec :

\[
    T_{evap} = 10^\circ C, \quad
    T_{cond} = 35^\circ C, \quad
    T_{gen} = 100^\circ C
\]

Le solveur global a convergé en deux itérations avec une erreur relative inférieure à $10^{-4}$, sans activation de flag d'incohérence.
\newpage
%% Ajouter ici le schéma du cycle avec numérotation (1 à 8)

% ------------------------------------------------------------
\section{Cas nominal}

\subsection{États thermodynamiques}

Les états calculés sont résumés ci-dessous :

\begin{itemize}
    \item État 1 : sortie condenseur
          \(P_1 = 5.629~\mathrm{kPa}, \quad T_1 = 35^\circ C, \quad x_1 = 0\)

    \item État 2 : sortie détendeur
          \(P_2 = 1.228~\mathrm{kPa}, \quad T_2 = 10^\circ C, \quad x_2 = 0.04223\)

    \item État 3 : sortie évaporateur
          \(P_3 = 1.228~\mathrm{kPa}, \quad x_3 = 1\)

    \item État 4 : sortie mélange éjecteur
          \(P_4 = 2.828~\mathrm{kPa}\)

    \item État 5 : sortie diffuseur
          \(P_5 = 5.629~\mathrm{kPa}\)

    \item État 7 : sortie pompe
          \(P_7 = 101.418~\mathrm{kPa}\)

    \item État 8 : sortie chaudière
          \(P_8 = 101.418~\mathrm{kPa}, \quad T_8 = 100^\circ C, \quad x_8 = 1\)
\end{itemize}

On observe que le cycle fonctionne intégralement sous vide côté évaporateur, avec :

\[
    P_{evap} \approx 1.228~\mathrm{kPa}
\]

soit près de 800 fois inférieur à la pression atmosphérique.

% ------------------------------------------------------------
\subsection{Débits massiques}

Les débits obtenus sont :

\[
    \dot{m}_{pri} = 0.004423~\mathrm{kg/s}
\]

\[
    \dot{m}_{sec} = 0.005058~\mathrm{kg/s}
\]

\[
    \dot{m}_{tot} = 0.009481~\mathrm{kg/s}
\]

Le rapport d’entraînement est :

\[
    \mu = \frac{\dot{m}_{sec}}{\dot{m}_{pri}} = 1.1435
\]

Cette valeur indique un fonctionnement stable en régime supersonique avec choc interne dans la chambre de mélange.

% ------------------------------------------------------------
\subsection{Bilans énergétiques}

Les puissances calculées sont :

\[
    \dot{Q}_{evap} = 12~\mathrm{kW}
\]

\[
    \dot{Q}_{gen} = 11.185~\mathrm{kW}
\]

\[
    \dot{Q}_{cond} = 23.954~\mathrm{kW}
\]

Le bilan énergétique global vérifie :

\[
    \dot{Q}_{cond} = \dot{Q}_{evap} + \dot{Q}_{gen} + \dot{W}_p
\]

avec :

\[
    \dot{W}_p = 0.000609~\mathrm{kW}
\]

confirmant le caractère négligeable du travail de pompe.

% ------------------------------------------------------------
\subsection{Coefficient de performance}

Le COP thermique du cycle est :

\[
    COP = \frac{\dot{Q}_{evap}}{\dot{Q}_{gen}} = 1.0728
\]

Cette valeur est élevée pour un cycle à éjection vapeur à 100°C, ce qui s’explique par :

\begin{itemize}
    \item le fort effet frigorifique spécifique du R718,
    \item la bonne récupération de pression dans le diffuseur,
    \item l’optimisation du rapport d’entraînement.
\end{itemize}

%% Ajouter ici le diagramme P-h nominal
%% Ajouter ici le diagramme T-s nominal

% ------------------------------------------------------------
\section{Analyse de sensibilité}

\subsection{Influence de la température de condensation}

Une augmentation de \(T_{cond}\) entraîne :

\begin{itemize}
    \item une augmentation de \(P_{cond}\),
    \item un déplacement du choc dans l’éjecteur,
    \item une diminution du rapport d’entraînement \(\mu\),
    \item une réduction du COP.
\end{itemize}

Le cycle devient instable lorsque :

\[
    P_{cond} > P_{cond,crit}
\]

correspondant au décrochage de l’éjecteur.

%% Ajouter ici le graphique COP = f(T_cond)

% ------------------------------------------------------------
\subsection{Influence de la température générateur}

L’augmentation de \(T_{gen}\) :

\begin{itemize}
    \item augmente la pression motrice,
    \item améliore l’aspiration secondaire,
    \item accroît \(\mu\),
    \item mais augmente également les pertes thermiques solaires.
\end{itemize}

Un optimum thermodynamique existe entre performance éjecteur et rendement solaire.

%% Ajouter ici le graphique mu = f(T_gen)

% ------------------------------------------------------------
\section{Analyse exergétique du système}

L’exergie détruite dans le cycle provient principalement de :

\begin{itemize}
    \item l’irréversibilité du choc dans l’éjecteur (\(\Delta s > 0\)),
    \item le laminage isoenthalpique du détendeur,
    \item les transferts thermiques à faible gradient.
\end{itemize}

La production d’entropie mesurée au niveau du choc est :

\[
    \Delta s = 0.8385~\mathrm{kJ/kg.K}
\]

indiquant une dissipation modérée compatible avec un régime stable.

% ------------------------------------------------------------
\section{Limites du modèle}

Le modèle actuel présente les limitations suivantes :

\begin{itemize}
    \item hypothèse de détente strictement isoenthalpique,
    \item condensation modélisée en régime laminaire,
    \item absence de modélisation transitoire,
    \item absence de non-condensables,
    \item pertes de charge simplifiées.
\end{itemize}

% ------------------------------------------------------------
\section{Recommandations conceptuelles}

Les résultats obtenus suggèrent :

\begin{itemize}
    \item un dimensionnement précis du condenseur pour limiter \(P_{cond}\),
    \item une maîtrise rigoureuse des pertes de charge sous vide,
    \item une optimisation conjointe \(T_{gen}\)/surface solaire,
    \item une étude future incluant exergie complète et dynamique transitoire.
\end{itemize}

% ------------------------------------------------------------
\section*{Résumé du chapitre}

L’analyse des performances confirme la viabilité thermodynamique du cycle R718 à éjecteur sous vide profond pour une puissance frigorifique de 12 kW.

Le COP obtenu (1.0728) et le rapport d’entraînement (1.1435) traduisent un fonctionnement stable et cohérent avec la modélisation implémentée.

La stabilité du cycle demeure fortement conditionnée par la pression de condensation et par la maîtrise des irréversibilités dans l’éjecteur.

Le chapitre suivant discutera les perspectives d’optimisation et d’industrialisation du système.
