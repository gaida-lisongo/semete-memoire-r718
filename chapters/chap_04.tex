\chapter{Results and Discussion}

This chapter presents the results obtained using the methodology described in Chapter~3. The performance of the proposed approach is evaluated, and the results are analyzed and discussed in the context of the objectives stated in Chapter~1.

\section{Experimental Setup}

This section describes the experimental or evaluation setup used to obtain the results. It includes information about the data or test cases considered, parameter settings, evaluation metrics, and any assumptions made during the experiments. These details ensure that the results can be properly interpreted and reproduced.

\subsection{Results}

This section reports the main results obtained from the experiments or analyses. The outcomes are presented using \emph{tables, figures, or quantitative measures}, as appropriate. Key observations are highlighted, and the behavior of the proposed method is examined under different conditions or parameter choices.

\begin{table}[ht!]
\centering
\caption{Summary of results obtained for different test cases.}
\label{tab:results}
\begin{tabular}{c c c c}
\hline
\textbf{Test Case} & \textbf{Parameter Setting} & \textbf{Metric 1} & \textbf{Metric 2} \\
\hline
Case 1 & Setting A & Value 1 & Value 2 \\
Case 2 & Setting B & Value 3 & Value 4 \\
Case 3 & Setting C & Value 5 & Value 6 \\
\hline
\end{tabular}
\end{table}

\section{Discussion}

This section provides an interpretation of the results. The findings are discussed in relation to the objectives of the work and compared, when relevant, with existing methods or theoretical expectations. Strengths and limitations of the proposed approach are identified, and possible explanations for the observed behavior are provided.