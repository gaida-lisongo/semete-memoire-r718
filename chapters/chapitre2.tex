\chapter{Le fluide frigorigène : l’eau (R718)}

\section{Introduction partielle}

Le choix du fluide frigorigène constitue l’un des déterminants majeurs des performances thermodynamiques, environnementales et technologiques d’un système frigorifique.
Historiquement, les premiers cycles frigorifiques exploitaient des fluides naturels tels que l’ammoniac (\(\mathrm{NH_3}\)), le dioxyde de carbone (\(\mathrm{CO_2}\)) ou l’eau, avant que les chlorofluorocarbures (CFC) ne dominent le marché au milieu du XX\textsuperscript{e} siècle.
Toutefois, les impacts environnementaux associés aux CFC puis aux hydrofluorocarbures (HFC) ont conduit à une transition progressive vers des fluides à faible potentiel de réchauffement global (GWP) \cite{Herold2016}.

Dans ce contexte, l’eau — référencée \textbf{R718} selon la nomenclature ASHRAE — présente un intérêt renouvelé.
Elle possède un GWP nul, un ODP nul, une non-toxicité, une ininflammabilité et une disponibilité universelle.
Cependant, son utilisation comme fluide frigorigène impose des contraintes physiques majeures liées à ses propriétés thermodynamiques intrinsèques, notamment une pression de saturation très faible aux températures usuelles de production de froid.

L’analyse approfondie de ces caractéristiques est indispensable pour comprendre les défis liés au fonctionnement sous vide poussé de la machine étudiée.

\section{Conditions thermodynamiques requises pour un fluide frigorigène}

Un fluide frigorigène adapté doit satisfaire plusieurs critères thermodynamiques fondamentaux, parmi lesquels :

\begin{itemize}
    \item une pression d’évaporation modérée (limitant le vide poussé) ;
    \item une chaleur latente élevée ;
    \item un faible volume spécifique en phase vapeur (limitant les débits volumiques) ;
    \item une température critique suffisamment élevée (élargissant le domaine opératoire).
\end{itemize}

D’un point de vue énergétique, l’effet frigorifique spécifique associé à un changement de phase à l’évaporateur peut s’exprimer, en première approximation, par :

\begin{equation}
    q_{\mathrm{evap}} = h_v - h_l,
    \label{eq:q_evap_def}
\end{equation}

où \(h_v\) est l’enthalpie de la vapeur saturée et \(h_l\) celle du liquide saturé.
Plus la différence \((h_v - h_l)\) est grande, plus la puissance frigorifique spécifique est élevée pour un débit massique donné.

L’eau présente une chaleur latente de vaporisation particulièrement élevée (de l’ordre de \(\approx 2400~\mathrm{kJ\,kg^{-1}}\) à \(10^\circ\mathrm{C}\)), ce qui constitue un avantage énergétique majeur \cite{Moran2014}.
La contrepartie se manifeste néanmoins par la relation pression--température, particulièrement défavorable au voisinage de \(10^\circ\mathrm{C}\).

\section{Relation pression--température et fonctionnement sous vide}

La pression de saturation d’un fluide pur est régie par l’équation de Clausius--Clapeyron :

\begin{equation}
    \frac{dP}{dT} = \frac{L}{T\left(v_v - v_l\right)},
    \label{eq:clausius_clapeyron}
\end{equation}

où \(L\) est la chaleur latente de changement de phase, et \(v_v\) et \(v_l\) sont respectivement les volumes spécifiques de la vapeur et du liquide.

Pour l’eau à \(10^\circ\mathrm{C}\), la pression de saturation est typiquement :

\begin{equation}
    P_{\mathrm{sat}}(10^\circ\mathrm{C}) \approx 1.23~\mathrm{kPa}.
    \label{eq:psat_10C}
\end{equation}

Ce niveau de pression est environ \(800\) fois inférieur à la pression atmosphérique standard, ce qui implique :

\begin{itemize}
    \item un fonctionnement sous vide profond ;
    \item un volume spécifique vapeur très élevé ;
    \item des vitesses d’écoulement \(c\) potentiellement élevées dans les conduites et l’éjecteur ;
    \item une sensibilité accrue aux pertes de charge et aux infiltrations d’air.
\end{itemize}

Ainsi, la principale difficulté associée au R718 en réfrigération n’est pas la capacité frigorifique intrinsèque, mais les contraintes mécaniques, hydrauliques et d’étanchéité qu’impose le vide poussé.

\section{Analyse du volume spécifique vapeur}

En première approche, le volume spécifique vapeur peut être estimé par le modèle du gaz parfait :

\begin{equation}
    v \approx \frac{R\,T}{P},
    \label{eq:ideal_gas_specific_volume}
\end{equation}

où \(R\) est la constante spécifique du gaz et \(T\) la température absolue.
Pour la vapeur d’eau, \(R \approx 461~\mathrm{J\,kg^{-1}\,K^{-1}}\).

À \(T \approx 283~\mathrm{K}\) et \(P \approx 1.23\times 10^3~\mathrm{Pa}\), on obtient :

\begin{equation}
    v \approx \frac{461 \times 283}{1230} \approx 106~\mathrm{m^3\,kg^{-1}}.
    \label{eq:vapor_specific_volume_approx}
\end{equation}

Cet ordre de grandeur met en évidence un débit volumique important pour des débits massiques pourtant modérés.
Les conséquences directes incluent :

\begin{itemize}
    \item un dimensionnement critique des sections d’écoulement ;
    \item un accroissement du risque de pertes de charge (et donc de dégradation des performances) ;
    \item un comportement fortement non linéaire de l’éjecteur vis-à-vis des variations de pression secondaire.
\end{itemize}

Des travaux de référence montrent en particulier que la performance d’un éjecteur devient très sensible aux variations de conditions lorsqu’on opère avec de grands volumes spécifiques vapeur \cite{Huang1999}.

\section{Avantages thermodynamiques du R718}

Malgré les contraintes associées au vide poussé, l’eau présente plusieurs avantages thermodynamiques et environnementaux :

\begin{enumerate}
    \item \textbf{Chaleur latente élevée :} elle permet de réduire le débit massique nécessaire pour une puissance frigorifique donnée.
    \item \textbf{Température critique élevée :} \(T_c \approx 374^\circ\mathrm{C}\), autorisant un large domaine d’exploitation thermique.
    \item \textbf{Capacité calorifique liquide élevée :} favorisant une stabilité thermique des phases liquides.
    \item \textbf{Compatibilité environnementale :} GWP = 0 et ODP = 0.
\end{enumerate}

Selon \citeauthor{Chunnanond2004} (\citeyear{Chunnanond2004}), les cycles à éjection vapeur utilisant l’eau sont particulièrement adaptés aux applications de froid solaire, notamment en raison de la compatibilité entre chaleur disponible à moyenne température et génération de vapeur motrice.

\section{Contraintes spécifiques du fonctionnement sous vide}

Le fonctionnement sous vide impose plusieurs exigences de conception et d’exploitation :

\begin{itemize}
    \item étanchéité renforcée de l’ensemble des composants ;
    \item gestion des infiltrations d’air et des gaz non condensables ;
    \item sélection de matériaux compatibles avec le vide et les gradients thermiques ;
    \item surveillance du risque de cavitation et des conditions d’aspiration de la pompe (NPSH).
\end{itemize}

À très basse pression, les problèmes de désamorçage, d’instabilité hydraulique et de cavitation peuvent compromettre la continuité de fonctionnement.
Ces contraintes justifient une modélisation rigoureuse des pertes de charge et des bilans énergétiques, ainsi qu’un dimensionnement attentif des conduites et des échangeurs.

\section{Comparaison avec un fluide HFC : analyse critique}

Le tableau~\ref{tab:r718_vs_r134a} présente une comparaison qualitative entre l’eau (R718) et un HFC courant (R134a), à titre illustratif.

\begin{table}[H]
    \centering
    \caption{Comparaison qualitative entre l’eau (R718) et le R134a.}
    \label{tab:r718_vs_r134a}
    \begin{tabular}{@{}p{4cm}p{4cm}p{4cm}@{}}
        \toprule
        \textbf{Propriété}                              & \textbf{Eau (R718)}           & \textbf{R134a}               \\ \midrule
        GWP                                             & 0                             & \(\approx 1430\)             \\
        Pression de saturation à \(10^\circ\mathrm{C}\) & \(\approx 1.23~\mathrm{kPa}\) & \(\approx 300~\mathrm{kPa}\) \\
        Chaleur latente                                 & Très élevée                   & Moyenne                      \\
        Volume spécifique vapeur                        & Très élevé                    & Faible                       \\ \bottomrule
    \end{tabular}
\end{table}

Il apparaît que le principal handicap du R718 est principalement \textbf{mécanique et hydraulique} (vide poussé et forts débits volumiques) plutôt qu’énergétique.
Ainsi, son adoption dépend davantage de la conception du système (étanchéité, dimensionnement, pertes de charge) que de ses performances thermiques intrinsèques.

\section{Conclusion partielle}

L’eau (R718) constitue un fluide frigorigène écologiquement irréprochable et thermodynamiquement performant en termes d’effet frigorifique spécifique.
Cependant, son utilisation à \(10^\circ\mathrm{C}\) impose un fonctionnement sous vide profond, entraînant des volumes spécifiques élevés et une forte sensibilité aux pertes de charge.

Ces particularités rendent indispensable :

\begin{itemize}
    \item une modélisation fine de l’éjecteur ;
    \item une attention particulière au dimensionnement hydraulique ;
    \item une analyse détaillée des phénomènes d’évaporation et de condensation.
\end{itemize}

La section suivante abordera le cœur dynamique du système : l’éjecteur de fluide, dont le comportement conditionne directement la stabilité et le \(COP\) de la machine.
