\chapter{Description de la machine frigorifique à eau de 12 kW}

\section{Évolution historique et positionnement scientifique}

L’histoire de la réfrigération thermodynamique débute au XIX\textsuperscript{e} siècle avec les travaux fondateurs de Carnot sur les cycles idéaux.
Par la suite, les premières machines à compression mécanique utilisant l’ammoniac et le dioxyde de carbone ont été développées pour des applications industrielles.

Parallèlement aux cycles à compression, des solutions alternatives utilisant la chaleur comme énergie motrice ont émergé, notamment les cycles à absorption et les cycles à éjection.
Les premiers systèmes à jet de vapeur (\textit{steam-jet refrigeration}) apparaissent à la fin du XIX\textsuperscript{e} siècle et trouvent des applications industrielles majeures au XX\textsuperscript{e} siècle \cite{Sokolov1990}.

Avec les restrictions environnementales issues du Protocole de Montréal puis de l’amendement de Kigali, les fluides frigorigènes synthétiques ont progressivement été remplacés par des fluides naturels \cite{Herold2016}.
L’eau (R718) présente l’avantage d’être non toxique, non inflammable et universellement disponible.

La valorisation de l’énergie solaire thermique dans les régions intertropicales a renforcé l’intérêt pour les cycles frigorifiques trithermes \cite{Kalogirou2014}.

Le présent travail étudie une machine frigorifique simple effet, dimensionnée pour une puissance frigorifique nominale de :

\begin{itemize}
    \item Température d’évaporation : $T_{evap} = 10^\circ C$
    \item Température de condensation : $T_{cond} = 35^\circ C$
    \item Température du générateur : $T_{gen} = 100^\circ C$
    \item Puissance frigorifique nominale : $\dot{Q}_{evap} = 12~\mathrm{kW}$
\end{itemize}

Cette configuration est adaptée aux conditions climatiques de la région intertropicale.

\section{Organisation générale de la machine}

La machine étudiée comprend :

\begin{itemize}
    \item Une pompe d’alimentation
    \item Une chaudière solaire
    \item Un éjecteur (tuyère + chambre de mélange + diffuseur)
    \item Un condenseur
    \item Un détendeur
    \item Un évaporateur
\end{itemize}

Le cycle thermodynamique suit la convention de numérotation suivante :

\begin{align*}
    1 & \rightarrow 2 &  & \text{Détendeur}          \\
    2 & \rightarrow 3 &  & \text{Évaporateur}        \\
    3 & \rightarrow 4 &  & \text{Chambre de mélange} \\
    4 & \rightarrow 5 &  & \text{Diffuseur}          \\
    5 & \rightarrow 6 &  & \text{Condenseur}         \\
    1 & \rightarrow 7 &  & \text{Pompe}              \\
    7 & \rightarrow 8 &  & \text{Chaudière}          \\
    8 & \rightarrow 4 &  & \text{Tuyère primaire}
\end{align*}

\section{Fonctionnement détaillé du cycle}

\subsection{Détente isoenthalpique (1 → 2)}

La détente dans le détendeur est modélisée comme isoenthalpique :

\begin{equation}
    h_1 = h_2
    \label{eq:detente_isoenthalpique}
\end{equation}

La pression chute de $P_{cond}$ à $P_{evap}$, générant un mélange diphasique.

\subsection{Évaporation (2 → 3)}

Dans l’évaporateur, le fluide absorbe la puissance frigorifique :

\begin{equation}
    \dot{Q}_{evap} = \dot{m}_{sec} \left( h_3 - h_2 \right)
    \label{eq:evaporation}
\end{equation}

À $10^\circ C$, la pression de saturation est de l’ordre du kPa, impliquant un volume spécifique élevé \cite{Moran2014}.

\subsection{Pompage (1 → 7)}

La pompe élève la pression du liquide :

\begin{equation}
    W_p = \dot{m}_{pri} \left( h_7 - h_1 \right)
    \label{eq:pompe}
\end{equation}

Le travail de la pompe reste négligeable devant la puissance thermique.

\subsection{Génération vapeur motrice (7 → 8)}

La chaudière solaire fournit la chaleur :

\begin{equation}
    \dot{Q}_{gen} = \dot{m}_{pri} \left( h_8 - h_7 \right)
    \label{eq:generateur}
\end{equation}

La vapeur atteint l’état saturé à $T_{gen} = 100^\circ C$.

\subsection{Éjecteur (8 → 4 → 5)}

L’éjecteur réalise trois transformations :

\begin{enumerate}
    \item Accélération dans la tuyère primaire (8 → 4)
    \item Mélange adiabatique (3 → 4)
    \item Recompression dans le diffuseur (4 → 5)
\end{enumerate}

Le taux d’entraînement est défini par :

\begin{equation}
    \mu = \frac{\dot{m}_{sec}}{\dot{m}_{pri}}
    \label{eq:mu}
\end{equation}

Les équations de conservation utilisées sont celles de la masse, de la quantité de mouvement et de l’énergie \cite{Huang1999}.

\subsection{Condensation (5 → 6)}

Dans le condenseur, le fluide rejette la chaleur :

\begin{equation}
    \dot{Q}_{cond} = \dot{m}_{tot} \left( h_5 - h_6 \right)
    \label{eq:condensation}
\end{equation}

Le fluide redevient liquide saturé à la pression $P_{cond}$.

\section{Indicateur de performance énergétique}

Le coefficient de performance énergétique est défini par :

\begin{equation}
    COP = \frac{\dot{Q}_{evap}}{\dot{Q}_{gen}}
    \label{eq:COP}
\end{equation}

Contrairement aux cycles à compression mécanique, le travail de la pompe est négligeable.

Les études antérieures rapportent des valeurs typiques de $COP$ comprises entre 0.2 et 0.5 pour les cycles à éjection vapeur \cite{Chunnanond2004}.
Le modèle développé dans ce travail permet d’atteindre des performances supérieures dans des conditions favorables.

\section{Résumé du chapitre}

Ce chapitre a présenté une description détaillée de la machine frigorifique à éjecteur fonctionnant au R718, en mettant l’accent sur les aspects historiques, scientifiques et technologiques. Les éléments clés abordés comprennent :

\begin{itemize}
    \item L’évolution historique des cycles à éjection
    \item Le positionnement environnemental du R718
    \item L’architecture détaillée de la machine
    \item La convention de numérotation adoptée
    \item Les indicateurs énergétiques fondamentaux
\end{itemize}

Elle constitue le socle de la modélisation mathématique détaillée développée dans les chapitres suivants.
